\section{Some Differentials on Steenrod Squares in the Adams Spectral Sequence}

For the rest of this paper we will set $Y=S$, so that the spectral sequence in question will converge to the 2-component of $\pi_*$ (for $X$ a suspension spectrum, this is $\pi_*^s$).  In this special case, we have that if $X\from \{X_i\}$ is an Adams Resolution with $K_i$ the cofibers, then 
\[\pi_{t-s}(K_s)\cong \pi_{t-s}(X_s,X_{s+1}) = [(D^{t-s},S^{t-s-1}),(X_s,X_{s+1})]_0\]
The differential is of a map $f\in \pi_{t-s}(K_s)$ is given by restricting 
\[f|_{S^{t-s-1}}:S^{t-s-1}\to X_{s+1}\]
 and compressing the image to $X_{s+r-1}$, and consider this as an element of $\pi_{t-s-1}(X_{i+r-1},X_{i+r})$. 


\subsection{Topology of Stunted Projective Spectra and the J Homomorphism}

It will happen that the elements of the form $Sq^ix$ in the Adams Spectral Sequence will factor through various quadratic constructions, and these are stunted projective spectra, so we need to know a bit about their topology.
A stunted projective spectra,  $\Sigma^nP^{n+k}_n$, has one cell in each dimension from $2n$ to $2n+k$.  
We say that such a spectrum is reducible if
\[\Sigma^n P^{n+k}_n \cong \Sigma^n(P^{n+k-1}_n\wedge S^{n+k})\]
\begin{Def}
  \[v(n) = \max\{v|P^n_{n-v+1}\mbox{ is reducible}\}\]
  Let $a(n) \in \pi_{n-1}(S^{n-v(n)})=\pi_{v(n)-1}(S)$ be the composite
  \[S^{n-1}\to P^{n-v(n)}_0\to P^{n-v(n)}_{n-v(n)}\cong S^{n-v(n)}\]
  where the first map is the attaching map of the top cell of $P^n_0$.
  By the definition of $v(n)$ the attaching map of this cell lies in $P_0^{n-v(n)}$, since $P^n_{n-v(n)+1}$ is reducible.  
\end{Def}

There is a map, known as $J:\pi_r(SO)\to \pi_r^s(S)$, from the unstable homotopy of the Lie Group $SO$ to the stable homotopy groups of spheres.  This map was studied by Mahowald in \cite{imJ}.  We state the following theorem without proof.
\begin{Theorem}
  \label{sec:jthm}
  There is a map 
  \[J:\pi_r(SO)\to \pi_r^s(S)\]
  such that
  \begin{enumerate}
  \item The image of $J$ in $\pi_r(S)$ is a cyclic summand.
  \item The image is trivial if $r\not\equiv$ 0,1 or 2 (mod 8).  
  \item The order of the image is 2 if $r>0$ is 0 or 1 mod 8.  
  \item If $r\equiv 3$ (mod 4), the order is given by the denominator of $B_{2n}/4n$, where $B_{2n}$ is a Bernoulli number.  If $r\equiv 3$ (mod 8), the two component of the image of $J$ is $2^3$, so this image is detected by three elements in the Adams Spectral Sequence of spheres.  
  \item In the Adams Spectral Sequence in topological degree $t-s\equiv0,3$ or 7 (mod 8), the image of the $J$ is detected by the ``tower'' of the appropriate size ending in as high a filtration as possible.  In topological degree $t-s\equiv 1$, the image of $J$ is detected by $h_1$ times the image of $J$ when $t-s\equiv 0$ (mod 8).  
    There is never any ambiguity in determining which elements detect the $\im J$. 
  \end{enumerate}
\end{Theorem}


One might consider this cheating, but we can actually infer some differentials from this.
In particular, we know that all differentials of elements detecting $\im J$ and no differential can hit them.  However, we can also infer some non-zero differentials in the Adams Spectral Sequence for spheres when $r\equiv 7$ (mod 8).  
If $h_0x$ detects the image of $J$ but $x$ does not, we know $x$ must either not be a cycle or be a boundary.  Thus we can infer, for instance, in $t-s=15$, that
\[d_3(h_0^2h_4) = h_0d_3(h_0h_4)\ne 0\]


We will need he following theorem, which we will state without proof.  It can be found in \cite{milgramGroupReps} and \cite[V.2.14-V.2.17]{H00RingSpectra}.


\begin{Theorem}
  \label{sec:stuntthm}
  Let $\varphi(k)$ be the number of integers congruent to $0,1,2$ or 4 mod 8 less than or equal to $k$.  
  The following are true of $\Sigma^n P^{n+k}_n$
  \begin{enumerate}
  \item $P^{n+k}_n$ is reducible iff $n+k+1\equiv 0$ (mod $2^{\varphi(k)}$).
    Equivalently, if $\epsilon$ is the number of factors of 2 in $n+1$ and $\epsilon = 4a+b$ for $0\le b < 4$, then
    \[v(n)=8a+2^b\]
  \item If $n\equiv n'$ (mod $2^{\varphi(k)}$), then $P^{n+k}_n\equiv \Sigma^{n-n'}P^{n'+k}_{n'}$
  \item If $v(n)=1$, then $a(n)=2\in \pi_0(S)$, that is, $a(n)$ is a degree two map of spheres.  
  \item If $v(n)>1$, then $a(n)$ generates $\im J$ in $\pi_{v-1}(S)$.  
  \end{enumerate}
\end{Theorem}

\subsection{Setting Up The Cells}
\label{sec:setup}

Fix $n,s,k,r$ and an element $x\in E_r^{s,n}(S,X)$, represented by a map
\[x: (D^n,S^{n-1})\to (X_s,X_{s+r})\]
We are now ready to begin the long hard computation of $d_rSq^{k+n}x$.  

Give $D^n$ the cell structure of one $(n-1)$-cell and one $n$-cell, respectively called $dx$ and $x$.
This should be suggestive: we want to mentally connect these cells with the map $x$ and its boundary.  
Let the following spaces inherit this cell structure
\[\Gamma_0:=D^{2n}=D^n\wedge D^n\]
\[\Gamma_1:=S^{2n-1}=D^n\wedge S^{n-1}\cup S^{n-1}\wedge D^n\]
\[\Gamma_2:=S^{2n-2}=S^{n-1}\wedge S^{n-1}\]
$\Z/2$ acts on coherently on the $\Gamma_i$ by swapping the wedge factors.  

The cellular chains in each of these spaces are the obvious subspaces of
\[C_*D^{2n} = \langle x\otimes x,x\otimes dx,dx\otimes x,dx\otimes dx\rangle\]

Give $S^\infty$ the cell structure with two $k$-cells in dimension $k$, called $e_k$ and $\rho e_k$.
Let $\Z/2=\langle \rho \rangle$ act on this CW complex antipodally, by swapping the cells in each dimension.  

Similar to $Q$, we define the functor $P^j$ as
\[P^j(\Gamma_i)=S^j\ltimes_{\Z/2}\Gamma_i\subset Q^jD^n=P^j\Gamma_0\]
so that
We define this so that we can more easily refer to $P^j\Gamma_i$ as a filtration of $Q^jD^n$.  


Consider the inclusion
\[P^{k-1}\Gamma_1 \to P^k\Gamma_1\]
The cofiber is
\[P^k\Gamma_1\cup C(P^{k-1}\Gamma_1)
\cong P^k\Gamma_1\cup P^{k-1}\Gamma_0\]
where we recall the reduced cone of a spectrum $X$ is $CX=I\wedge X$ and use the properties mentioned in \ref{sec:quadConst}.  
However, the inclusion is equivalent to the inclusion of the $2n+k-2$ skeleton (that is, all but the top dimension cell) of 
\[\Sigma^{n-1}P^{n+k-2}_{n-1}\to \Sigma^{n-1}P^{n+k-1}_{n-1}\]
and so the cofiber is isomorphic to $S^{2n+k-1}$.  
Of course, $P^k\Gamma_0\cong D^{2n+k}$ with boundary $P^k\Gamma_1\cup P^{k-1}\Gamma_0$.  
Thus we have that maps in $\pi_{t-s}(Z_s,Z_{s+1})$ (for some Adams resolution $Z_*$) are represented by maps of pairs
\[(P^k\Gamma_0,P^k\Gamma_1\cup P^{k-1}\Gamma_0)\to (Z_s,Z_{s+1})\]


Our strategy for computing $d_*Sq^jx$ will be as follows.
We will show that the boundary of the map $Sq^{k+n}x$, as a relative homotopy class,
factors through $P^{k}\Gamma_1\cup P^{k-1}\Gamma_0$, which are stunted projective spaces.
This boundary can be recognized as the top cell of $P^{k+1}\Gamma_2$, which we shoehorn into $P^k\Gamma_1$, and glue together in $P^{k}\Gamma_1\cap P^{k-1}\Gamma_0$ with some other homotopy class which depends on $v(k+n)$.  



\subsection{Identification of the Boundary and some Chain Calculations}

Recall the diagram in Corollary \ref{sec:relgeomsteen} constructing $Sq^{k+n}x$.  
Using the notation of this section, it looks like
\begin{RefDiagram}
  \label{sec:steengam}
  \begin{diagram}
    (D^{2n+k},S^{2n+k-1}) && \rTo^{Sq^{n+k}x} & & (X_{2s-k},X_{2s-k+1})\\
    \dTo^\approx   &           &    &    & \uTo_{\Theta_{k,s}}\\ 
    (D^k,S^{k-1})\ltimes(\Gamma_0,\Gamma_1) & \rTo^{1\ltimes x\wedge x} & 
    (D^k,S^{k-1})\ltimes(Z_{2s},Z_{2s+1})  & \rTo^{\overline{\psi}_+^k} &  (P^kZ_{2s},P^{k-1}Z_{2s}\cup P^kZ_{2s+r})\\
    \dTo^{i_+} & & & & \uTo^{1\ltimes x\wedge x}\\
    (S^k,S^{k-1})\ltimes (\Gamma_0,\Gamma_1) && \rTo &&    (P^k\Gamma_0,P^k\Gamma_1\cup P^{k-1}\Gamma_0)
  \end{diagram}
\end{RefDiagram}

There are a few things to note.  
First note that we lifted the boundary of $x$, which by abuse of notation we call $dx$, to $X_{s+r}$, so $d(1\ltimes x\wedge x)$ is lifted to $P^{k-1}Z_{2s}\cup P^kZ_{2s+r}$.
Second, note that the outside composite
\[(D^{2n+k},S^{2n+k-1})\to   (P^k\Gamma_0,P^k\Gamma_1\cup P^{k-1}\Gamma_0)\]
is a homotopy equivalence, by the argument in Section \ref{sec:setup}.
Finally, call the right side vertical composite 
\[\xi : (P^k\Gamma_0,P^k\Gamma_1\cup P^{k-1}\Gamma_0) \to (X_{2s-k},X_{2s-k+1})\]
More generally, we will use $\xi$ to represent any composite of $\Theta_{*,*}$ and $1\ltimes x\wedge x$, that is, $\xi$ is a map
\[\xi : P^k\Gamma_i\to X_{2s+ir-k}\]
This indexing makes sense if we recall the definition of $\Gamma_i$ and that $dx$ lands in $X_{s+r}$, so 
\[x\wedge x: (\Gamma_0,\Gamma_1,\Gamma_2)\to (Z_{2s},Z_{2s+r},Z_{2s+2r})\]

\begin{Remark}
  \label{sec:xirem}
  Notice that $\xi$ depends on $x$.  The map induced on homotopy should be thought of as a conversion between the world of cells of stunted projective spectra and the world of homotopy classes in the Adams Spectral Sequence.  
  For instance, if $f\in \pi_{k+2n}(P^k\Gamma_0,P^k\Gamma_1\cup P^{k-1}\Gamma_0)$ is the inclusion of the top cell, then clearly $\xi_*f$ represents $Sq^{k+n}x$, and the Hurewicz image of $f$ is $e_k\otimes x\otimes x$.  Similarly, the inclusion of the cell $e_{k}\otimes dx\otimes dx$ is sent to $Sq^{n+k-1}d_r x$.  The inclusion of $e_0\otimes x\otimes dx$ goes to $xd_r x$ for some $r$.  
\end{Remark}


Let 
\[\partial:S^{2n+k-1}\cong P^k\Gamma_1\cup P^{k-1}\Gamma_0 \to X_{2s-k+1}\]
denote the boundary.  
We want to find conditions on an element in $\pi_{2n+k-1}X_{2s-k+1}$ to recognize it.  

Recall that
\[C^*(P^k\Gamma_i)=C^*(S^k)\otimes_{\Z/2} C^*\Gamma_i\]
Here is an easy but important calculation
\begin{Lemma}
  \label{sec:elemma}
  In the integral homology group, $H_*(P^{i+1}\Gamma_1)$, if $n\not\equiv i$ (mod 2), then
  \[e_{i+1}\otimes dx\otimes dx = (-1)^ie_i\otimes d(x\otimes x)\]
  and if $n\equiv i$ (mod 2)
  \[e_{i+1}\otimes dx\otimes dx = (-1)^ie_i\otimes d(x\otimes x)-2e_i\otimes x\otimes dx\]
\end{Lemma}

\begin{proof}
  \begin{eqnarray*}
    d(e_{i+1}\otimes x\otimes dx)  &=& 
    (\rho+(-1)^{i+1})e_i\otimes x\otimes dx + (-1)^{i+1}e_{i+1}\otimes dx\otimes dx
  \end{eqnarray*}
  But since the tensor product is over $\Z/2$, we have that this is equal to
  \[e_i\otimes dx \otimes x + (-1)^{i+1}e_i\otimes x\otimes dx + (-1)^{i+1}e_{i+1}\otimes dx\otimes dx\]
  Using that $d(x\otimes x)=dx\otimes x + (-1)^n x\otimes dx$, we have in homology that
  \[e_{i+1}\otimes dx\otimes dx = (-1)^ie_i\otimes d(x\otimes x) - (1+(-1)^{i+n}e_i\otimes x\otimes dx\]
\end{proof}

\begin{Remark}
  \label{sec:assgradrem}
  The introduction of integral homology, non unit coefficients and signs may be confusing.
  Just recall that the situation in the Adams Spectral Sequence is actually an associated-graded version of the situation in actual homotopy or homology.
  Multiplication by 2 in homotopy corresponds to a filtration shift in the Adams Spectral Sequence.  
\end{Remark}

Consider the inclusion
\[P^{i+1}\Gamma_2\to P^{i+1}\Gamma_1\]
Since  $P^{i+1}\Gamma_1=\Sigma^{n-1}P^{n+i}_{n-1}$ is $2n+i-1$ dimensional and 
$P^{i+1}\Gamma_1/P^{i}\Gamma_1\cong S^{2n+i}$ is $2n+i$ dimensional, the inclusion factors through $P^i\Gamma_1$.  
Define that map to be 
\[e:P^{i+1}\Gamma_2\to P^i\Gamma_1\]
\begin{Lemma}
  \label{sec:e_hom_lem}
  if $i\not\equiv n$ (mod 2), then
  \[e_*(e_{i+1}\otimes dx\otimes dx) = (-1)^ie_i\otimes d(x\otimes x)\]
\end{Lemma}

\begin{proof}
  The proof is given by chasing through the various homology groups and using the equivalences previously given.  
\end{proof}

We can now recognize $\partial$.
\begin{Lemma}
  \label{sec:ilem}
  Let $\mathfrak{i}\in \pi_{2n+k-1}(P^k\Gamma_1\cup P^{k-1}\Gamma_0)$ be the Hurewicz preimage of the homology class

  \[(-1)^ke_k\otimes d(x\otimes x) + \left\{\begin{array}{cc} 
  0 & k\not\equiv n\mbox{ (mod 2)}\\
  (-1)^k2e_{k-1}\otimes x\otimes x & k\equiv n\mbox{ (mod 2)}\end{array}\right.\]

  Since Hurewicz is an isomorphism in this dimension, this defines $\mathfrak{i}$.  We have

  \[\xi_*(\mathfrak{i}) = \partial\]
\end{Lemma}

\begin{proof}
  Hurewicz is an isomorphism in this dimension since $P^k\Gamma_1\cup P^{k-1}\Gamma_0\cong S^{2n+k-1}$, so we need only notice that this is the boundary of $e_k\otimes x\otimes x$, which is the Hurewicz image of that which $\xi_*$ sends to $Sq^{k+n}x$.  
\end{proof}


\subsection{Differentials When $v$ is Large}

\begin{Theorem}
  If $v(k+n)>k+1$ then 
  \[d_{2r-1}Sq^{k+n}x=Sq^{k+n}d_r x\]
\end{Theorem}

\begin{proof}
  Consider the diagram
  \begin{diagram}
                    &        &             &      &        S^{2n+k-1}\\
                    &        &      & \ldDashto(4,2)^C & \dTo^{\mathfrak{i}}\\
    P^{k+1}\Gamma_2 & \rTo^e & P^k\Gamma_1 & \rTo^i & P^k\Gamma_1\cup P^{k-1}\Gamma_0\\
    \dTo_\xi       &         & \dTo^\xi    &        & \dTo^\xi\\
    X_{2s+2r-k+1}  & \rTo    & X_{2s+r-k+1}& \rTo   & X_{2s-k+1} 
  \end{diagram}
  Note that the middle $\xi$ technically has codomain $X_{2s+r-k}$, but we want to diagram to still make sense when $r=2$, so we include $X_{2s+r-k}\subset X_{2s+r-k+1}$.  
  We have
  \[P^{k+1}\Gamma_2\cong \Sigma^{n-1}P^{n+k}_{n-1}\cong S^{2n+k-1}\vee \Sigma^{n-1}P^{n+k-1}_{n-1}\]
  where the second equivalence comes from the fact that $P^{n+k}_{n+k-m+1}$ is reducible for $m\le k+2 \le v(k+n)$.  
  Thus the homology of $P^{k+1}\Gamma_2$ is generated by $e_{k+1}\otimes dx\otimes dx$, so there is an $C\in \pi_{2n+k-1}(P^{k+1}\Gamma_2)$ (for instance the inclusion of the top cell) which has this as Hurewicz image.  
  If the diagram commutes, $\xi C$ is a lift of $\xi\mathfrak{i}=\partial$ and we are done, since, by Diagram \ref{sec:steengam} and Remark \ref{sec:xirem}, 
  \[\xi C =Sq^{k+n}d_rx\]
  By Theorem \ref{sec:stuntthm}, $k+n$ is odd, 
  so $ie C$ into has, by Lemma \ref{sec:e_hom_lem}, Hurewicz image $(-1)^ke_k\otimes d(x\otimes x)$, which is the Hurewicz image of $\mathfrak{i}$, and since Hurewicz is an isomorphism the relevant dimensions, this means $ieC=\mathfrak{i}$, so the diagram commutes and we are done.
\end{proof}

\begin{Theorem}
  Let $v(k+n)=k+1$ and let $f$ be the (readily computable by Theorem \ref{sec:stuntthm}) filtration of the image of $J$ in $\pi_{k}S$.  Let $a$ detect the generator of $im(J)$ in $\pi_k$.  Then

  \[  \begin{array}{llccr}
    d_{2r-1}Sq^{n+k}x &=& Sq^{n+k}d_rx & \mbox{if } & 2r-1 < r + f + k\\
    d_{2r-1}Sq^{n+k}x &=& Sq^{n+k}d_rx + \overline{a}xd_rx & \mbox{if } &2r-1 = r + f + k\\
    d_{r + f + k}Sq^{n+k}x &=& \overline{a}xd_rx & \mbox{if }& 2r-1 > r + f + k
  \end{array}\]
  
  where $\overline{a}$ is the detector of $a$ in the Ext.  

\end{Theorem}

\begin{proof}

  Let $C\in \pi_{2n+k-1}(P^{k+1}\Gamma_2)$ be the attaching map of the top cell.  
  Since $v(k+n)=k+1$, $\partial C$ can be compressed into the $(2n-2)$-skeleton of $P^{k+1}\Gamma_2$, which is equivalent to $P^0\Gamma_2\cong \Gamma_2$.  
  By Theorem \ref{sec:stuntthm} $a=\partial C$, and $C$ has Hurewicz image $e_{k+1}\otimes dx\otimes dx$.  
  Let
  \[R:(D^{2n-1},S^{2n-2})\to (P^0\Gamma_1,P^0\Gamma_2)\cong (\Gamma_1,\Gamma_2)\]
  bet the inclusion of the upper hemisphere of $\Gamma_1$, with Hurewicz image $x\otimes dx$ (which we can call $e_0\otimes x\otimes dx$).  
  Of course, $\partial R$ has Hurewicz image $e_0\otimes dx\otimes dx$ and is an isomorphism.  
  
  With this notation, there are two cases we need to check: $k=0$ and $k\ne 0$.  Let $Ca$ be the cone on $a$.  Assume first that $k\ne 0$ and get the following diagram:
  \begin{diagram}
    S^{2n+k-2} & & \rTo & & D^{2n+k-1} & & \\
    & \rdTo^{(\partial R)a} & & & \vLine & \rdTo^{R(Ca)} & \\
    \dTo&&\Gamma_2 & & \rTo & & \Gamma_1 & & \\
    &&&  &\dTo & & & \rdTo_i & \\
    D^{2n+k-1}&\hLine&\dTo &\rTo & S^{2n+k-1} &  & \dTo  &&  P^{k-1}\Gamma_0\\
    &\rdTo_C&& &  &\rdTo_{R(Ca)\cup C} &  & & \\
    &&P^{k+1}\Gamma_2 & \rTo &  & & \Gamma_1\cup P^{k+1}\Gamma_2 & & \dTo \\
    &&& \rdTo_{e} & & & & \rdTo^{i\cup e} & \\
    &&& & P^k\Gamma_1 & & \rTo & &  P^{k-1}\Gamma_0\cup P^k\Gamma_1
  \end{diagram}
  
  Recall that $P^{k-1}\Gamma_0$ is the cone on $P^{k-1}\Gamma_1$, so there is a homotopy equivalence
  \[\pi : P^{k-1}\Gamma_0\cup P^k\Gamma_1\to P^k\Gamma_1/P^{k-1}\Gamma_1\]
  and, since $[eC\cap iR(Ca)]=[a]=[0]$ mod $P^{k-1}\Gamma_1= P^{k-1}\Gamma_0\cap P^k\Gamma_1$, we have
  \[\pi(iR(Ca)\cup eC) = [eC]-[iR(Ca)]\]
  where $[\alpha]$ denote the equivalence class of such a map $\alpha$ mod $P^{k-1}\Gamma_1$.  
  Of course, by the diagram, $iR(Ca)\equiv 0$ mod $P^{k-1}\Gamma_1$.
  Thus the Hurewicz image of $iR(Ca)\cup eC$ is the Hurewicz image of $eC$, which by Lemma \ref{sec:e_hom_lem} is
  \[(-1)^ke_k\otimes d(x\otimes x)\]
  and so, by Lemma \ref{sec:ilem}
  \[iR(Ca)\cup eC=\mathfrak{i}\]
  
  Of course, we have, by Diagram \ref{sec:steengam}, we have
  \[\xi C = Sq^{n+k}d_r x\in \pi_*(X_{2s-k+2r-1},X_{2s+2r})\]
  for $r$ small enough for this to make sense.  
  Of course, recalling from  Theorem \ref{sec:thetathm} the base case in the inductive definition of
  \[\xi_*:\pi_*(\Gamma_0,\Gamma_1)\to \pi_*(X_{2s+r},X_{2s+2r})\]
  we see that $\xi_*$ sends the Hurewicz preimage of $x\otimes dx (= e_0\otimes x\otimes dx)$ to $xd_rx$.  
  We thus have in $\pi_*(X_{2s-k+1},X_{2s+2r})$, using that $\xi_*$ commutes with $e$ and $i$, 
  \begin{eqnarray*}
    \partial &=& \xi_*(eC\cup iR(Ca))\\
    &=& \xi_*(eC)-\xi_*(iR(Ca))\\
    &=& \xi_*(C)-\overline{a}\xi_*(R)
  \end{eqnarray*}
  Noting that this represents $d_{2r-1}x$ and sorting through the filtrations, the theorem is proved for $k>0$
  
  
  If $k=0$, one gets the diagram
  \begin{diagram}
    S^{2n-2} & & \rTo & & D^{2n-1} & & \\
    & \rdTo^{(\partial R)a} & & & \vLine & \rdTo^{R(Ca)} & \\
    \dTo&&\Gamma_2 & & \rTo & & \Gamma_1 & & \\
    &&&  &\dTo & & & \rdTo & \\
    D^{2n-1}&\hLine&\dTo &\rTo & S^{2n-1} &  & \dTo  &&  \Gamma_1\\
    &\rdTo_C&& &  &\rdTo_{R(Ca)\cup C} &  & & \\
    &&P^1\Gamma_2 & \rTo &  & & \Gamma_1\cup P^{1}\Gamma_2 & & \dTo \\
    &&& \rdTo_{e} & & & & \rdTo^{1\cup e} & \\
    &&& & \Gamma_1 & & \rTo & &  \Gamma_1
  \end{diagram}
  Note that $\pi: \Gamma_1\to \Gamma_1/\Gamma_2$ is an injection on cohomology (this is just a quotient of $S^{2n-1}$ by the equator, so the map in $H_{2n-1}$ is the diagonal $\F_2\to \F_2^2$).  The intersection $eC\cap R(Ca)$ is zero mod $\Gamma_2$, so the Hurewicz image of $\pi(eC\cap R(Ca))=[eC]-[R(Ca)]$ is, using $k+n=n$ is even,
  \begin{eqnarray*}
    e_*(e_1\otimes dx\otimes dx + 2e_0\otimes x\otimes dx) 
    &=& e_0\otimes dx\otimes x -e_0\otimes x\otimes dx + 2e_0\otimes x\otimes dx \\
    &=& e_0\otimes dx\otimes x +e_0\otimes x\otimes dx \\
    &=& e_0\otimes d(x\otimes x)
  \end{eqnarray*}
  by Lemma \ref{sec:elemma}and thus 
  \[eC\cap R(Ca)=\mathfrak{i}\]

  Like above, we have
  \begin{eqnarray*}
    \partial &=& \xi(eC\cup R(Ca))\\
    &=& \xi(eC)-\xi(R(Ca))\\
    &=& \xi(C)-\overline{a}\xi(R)
  \end{eqnarray*}
  and the theorem is proved.  

\end{proof}

\subsection{Differentials on the Hopf Invariant One Elements}

We would now like to give a differential formula which puts the Hopf Invariant question to rest once and for all.
\begin{Theorem}
  \label{sec:hdiffthm}
  If $k>0$ and $k+n$ is even (so $v(k+n)=1$)
  \[d_{2}Sq^{k+n}x = h_0Sq^{n+k-1}x\]
\end{Theorem}


\begin{proof}
  Like before, let $C$ be the top cell in $\pi_{k+2n-1}(P^k\Gamma_1,P^{k-1}\Gamma_1)$, so that it has Hurewicz image $(-1)^ke_k\otimes d(x\otimes x)$ (we can pick either sign, so pick this one).  
  Since $P^{k-1}\Gamma_0$ is contractible, there is an isomorphism 
  \[\partial : \pi_{2n+k-1}(P^{k-1}\Gamma_0,P^{k-1}\Gamma_1)\to \pi_{2n+k-2}(P^{k-1}\Gamma_1)\]
  coming from the fibration long exact sequence.  Of course the boundary of $C$, abusively denoted $\partial C$, is in the codomain, so define $A\in \pi_{2n+k-1}(P^{k-1}\Gamma_0,P^{k-1}\Gamma_1)$ by 
  \[A=\partial^{i-1}(\partial C)\]
  and once can easily check, using the corresponding isomorphism in integral homology and the fact that the Hurewicz map on $\pi_{k+2n-1}(P^k\Gamma_1,P^{k-1}\Gamma_1)$ is an isomorphism, that the Hurewicz image of $A$ is
  \[(-1)^{k-1}2e_{k-1}\otimes x\otimes x\]
  
  Consider now the map 
  \[C\cup A \in \pi_{2n+k-1}(P^k\Gamma_1\cup P^{k-1}\Gamma_0)\]
  Since the boundary is only $2n+k-2$ dimensional (and thus has zero $2n+k-1$ homology),
  we can use the same trick as before to see that this map actually splits as 
  \[(-1)^k(e_k\otimes d(x\otimes x) + 2e_{k-1}\otimes x \otimes x)\in H_{2n+k-1}(P^k\Gamma_1\cup P^{k-1}\Gamma_0,P^{k-1}\Gamma_1)\]
  Thus we have that, by Lemma \ref{sec:ilem}, 
  \[\mathfrak{i} \equiv C\cup A \mbox{ (mod $P^{k-1}\Gamma_1$)}\]
  However, since $H_{2n+k-1}P^{k-1}\Gamma_1$ is zero for dimension reasons, the long exact sequence in homology says that the inclusion
  \[H_{2n+k-1}(P^k\Gamma_1\cup P^{k-1}\Gamma_0)\to H_{2n+k-1}(P^k\Gamma_1\cup P^{k-1}\Gamma_0,P^{k-1}\Gamma_1)\]
  is an injection so in fact, 
  \[\mathfrak{i} = C\cup A\]
  
  Now, $A$ is divisible by 2, since it is divisible by 2 in homology and Hurewicz is a surjection.
  We have
  \[\xi_*\left(\frac{(-1)^{k-1}}{2}A\right) = Sq^{n+k-1}x \in \pi_*(X_{2s-k+1},X_{2s+2r-k+1})\]
  Since multiplying by 2 in homology is the same as pre-composing with the map detected by $h_0$, we have
  \[\xi_*A = (-1)^{k-1}h_0Sq^{n+k-1}x \in \pi_*(X_{2s-k+2},X_{2s+2r-k+1})\]
  Note that we can push the image of $A$ from $X_{2s-k+1}$ to $X_{2s-k+2}$ since it is twice a map into $(X_{2s-k+1},X_{2s+2r-k+1})$.
  Thus we have, mod $X_{2s+2r-k+1}$ (where $\xi$ sends $P^{k-1}\Gamma_1$, the domain of $C\cap A$), that
  \[\partial = \xi(C\cup A) = \xi C - \xi A = \xi C - (-1)^{k-1}h_0Sq^{n+k-1}x\]
  so if we can just push $\xi C$ up farther than $X_{2s-k+2}$, we will be done.  


  We have the following relative diagram
  \begin{diagram}
    (X_{2s-k+2},X_{2s+r-k+1}) & \lTo & (X_{2s-k+1},X_{2s+r-k+1}) & \lTo^\xi & (P^k\Gamma_1\cup P^{k-1}\Gamma_0,P^{k-1}\Gamma_1)\\
    \uTo & & &\ruTo \\
    (X_{2s+r-k},X_{2s+r-k+1}) & \lTo^\xi & (P^k\Gamma_1,P^{k-1}\Gamma_1) & \rTo & (P^{k+1}\Gamma_1,P^{k-1}\Gamma_1\cup P^{k+1}\Gamma_2)\\
    \dEq& &   \dTo                          &      \ruTo\\
    (X_{2s+r-k},X_{2s+r-k+1}) & \lTo^\xi & (P^k\Gamma_1,P^{k-1}\Gamma_1\cup P^k\Gamma_2) 
  \end{diagram}
  Now, consider the image of $C$ in $\pi_{2n+k-1}(P^{k+1}\Gamma_1,P^{k-1}\Gamma_1\cup P^{k+1}\Gamma_2)$.
  By Lemma \ref{sec:elemma}, $C$ has Hurewicz image $2e_k\otimes x\otimes dx$ here.  
  Since Hurewicz is still an isomorphism, this means there is an element 
  \[\frac{1}{2}C\in \pi_{2n+k-1}(P^{k+1}\Gamma_1,P^{k-1}\Gamma_1\cup P^{k+1}\Gamma_2)\]
  We can uniquely (Hurewicz is an isomorphism) pull this back to 
  \[\frac{1}{2}C\in \pi_{2n+k-1}(P^{k}\Gamma_1,P^{k-1}\Gamma_1\cup P^{k}\Gamma_2)\]
  since $\partial( e_k\otimes x\otimes dx)\in H_{2n+k-2}(P^{k-1}\Gamma_1\cup P^{k}\Gamma_2)$, making $\frac{1}{2}C$ still a cycle here.
  Thus $2\cdot\frac{1}{2}C$ is the image of $C$ in $\pi_{2n+k-1}(P^k\Gamma_1,P^{k-1}\Gamma_1\cup P^k\Gamma_2)$.  
  But $\pi_*(X_{2s+r-k},X_{2s+r-k+1})$ is an $\F_2$-vector space, so since $\xi C$ is divisible by 2 there, it must be zero, and so zero in 
  $\pi_*(X_{2s-k+2},X_{2s+r-k+1})$.  
  Thus, in the actual Adams Spectral Sequence, we get
  \[\partial = h_0Sq^{n+k-1}x\]
  
\end{proof}

To see that this answers the Hopf Invariant question, consider this formula when $x=h_i$ and $k=1$, so that $n=2^i-1$ and $k+n=2^i$.
Since $Sq^{2^i}h_i=h_{i+1}$, this formula says
\[d_2h_{i+1} = h_0Sq^{2^i-1}h_i = h_0h_i^2\]
Thus the formula says that whenever $h_0h_i^2\ne 0$ in $E_2$, there is no element of Hopf Invariant 1 in $\pi_{2^i-1}^s(S^{2^{i-1}})$.  

\begin{Lemma}[\cite{Wang}]
  In $\Ext^3_{\A}(\Z/2,\Z/2)$, the only relationships in elements of the form $h_ih_jh_k$ are
  \[\begin{array}{lcr} h_ih_{i+1}=0 & h_1h_{i+2}^2=0 & h_i^2h_{i+2}=h_{i+1}^3\end{array}\]
  In particular, for $i\ge 3$,
  \[h_0h_i\ne 0\]
\end{Lemma}
Thus $h_i$ does not make it past $E_2$ for $i>3$.  

\begin{Cor}
  The only $n$ such that $\pi_{2n-1}(S^n)$ contains a map of Hopf Invariant One are $n=1,2,4$ and $8$.  
\end{Cor}

\begin{Cor}
  The only finite dimensional division algebras over $\R$ are the complex numbers, quaternions and octonions.  
\end{Cor}


