\section{Setting up the Adams Spectral Sequence}

The Adams Spectral Sequence is a rather heavy-duty machine for computing $2$-component of the homotopy groups $[Y,X]_*$ for spectra $X$ and $Y$.  
If $X=S$, this is the 2-component of the stable homotopy groups $\pi^s_*(X)\otimes \F_2$.  
While this is twice removed from the original goal of computing $\pi_*(X)$, but at least in this case we have a fighting chance at doing the computations.
Since there are $p$-component analogues, these methods can be combined to compute all of $\pi_*^s(X)$.  
However, as we are about to find, even once the Adam's Spectral Sequence is set up, actually using it is not so easy.  

Morally, the Adams Spectral Sequence works as follows.  
We start with a ``geometric'' resolution of the space $Y$
\[X\to K_1\to K_2\to K_3\to ...\]
such that each $K_i$ is a wedge sum of Eilenberg-Maclane spaces and the sequence induced in mod-2 cohomology is exact.  
Thus we get an $\A$-free resolution of $H^*(Y)$, and we consider 
\[[Y,X]\from [Y,K_1]\from [Y,K_2]\from ...\]
Since $K_i$ is a wedge sum of $H\F_2$, we have for $H^*$ being mod-2 cohomology
\[[Y,K_i]\cong \bigoplus_i [Y,H\F_2]\cong \bigoplus_i H^*(Y) \cong \bigoplus_i Hom_\A(\A,H^*(Y))\cong Hom_\A(H^*(K_i),H^*(Y))\]
The reason we needed to develop spectra is now clear: A space $X$ cannot have $\A$-free cohomology.  
We can take homology of this chain complex and get $Ext_\A(H^*(X),H^*(Y))$, the starting point or so-called ``$E_2$'' term of the Adams Spectral Sequence.  
We will show this is a sort of over-approximation of $[Y,X]_*\otimes \F_2$, and the Adams Spectral Sequence ``converges'' to this.  
In the literature of spectral sequences, one might write
\[Ext_\A(H^*(X),H^*(Y))\implies [Y,X]\otimes \F_2\]
Computing this Ext group can often be tough, but is algebraic in nature and usually more-or-less mechanical.  
However, we are not done yet.  


The second step in running the Adams Spectral Sequence is to figure out exactly how the convergence works.
When we construct $Ext_\A(H^*(X),H^*(Y))$, we construct in it a way such that it is still a chain complex.
Thus we can take homology and get what we call the $E_3$ page.
Some elements of $E_2$ will not be cycles, and thus will disappear forever.  
Others will be related by boundaries and become equal.  
Thus $E_3$ is a sub-quotient of $E_2$, just as $E_2$ is a sub-quotient of $Hom_\A(H^*(X),H^*(Y))$.  
We can continue this process until all the differentials become zero, which we call $E_\infty$.  
It can be shown that the process does stabilize and the final answer, $[Y,X]\otimes \F_2$, can be read off the $E_\infty$ page. 
However, computing the differentials on each $E_r$ is notoriously difficult, in fact, it will be the goal of most of the rest of this paper.  
The differential computations are often geometric in nature; this is not surprising since the algebra of Ext cannot possibly be enough to determine all the homotopy groups.  

\subsection{The Adams Resolution}

Let $X$ and $Y$ be spectra, $X$ connective and of finite type, and let $H^*(?)=(H\F_2)^*(?)$ be mod-2 cohomology.  
We want to start by creating the geometric resolution of $Y$ described above.  
We construct the spaces of the resolution as follows.  
Consider $H^*(X)$, and assume it is finitely generated over $\A$.  
Call those generators $u_i\in [X,H\F_2]_{n_i}$.
We can then form 
\[\bigvee_i u_i:X\to \bigvee_i \Sigma^{n_i}H\F_2\] 
to be a degree 0 map to a wedge sum of suspensions of Eilenberg-Maclane spaces.  
We can then take the fiber of this map and call it $X_1$.  
Repeating this process, we get the following diagram.

\begin{diagram}
  X & \lTo & X_1 & \lTo & X_2 & \lTo & X_3 & \lTo &...\\ 
  \dTo & & \dTo && \dTo && \dTo\\
  K_0 && K_1 && K_2 && K_3
\end{diagram}


We call this the Adams Resolution of $X$.  
Now, notice that the fiber of $X_i \to X_{i-1}$ is $\Sigma^{-1}K_{i-1}$.  
Thus we get a diagram of spaces
\begin{diagram}
  X & \rTo&  K_0 & &\rTo && \Sigma K_1 && \rTo && \Sigma^2 K_2 && \rTo && \Sigma^3 K_3 && \rTo &...\\
    &   &  &    \rdTo  & & \ruTo && \rdTo && \ruTo&& \rdTo && \ruTo\\
    &      &  &   &  \Sigma X_1 & &&& \Sigma^2 X_2 & &&& \Sigma^3 X_3
\end{diagram}

In cohomology we get:

\begin{diagram}
  H^*X & \lTo&  H^*K_0 & &\lTo && H^*\Sigma K_1 && \lTo && H^*\Sigma^2 K_2 && \lTo && H^*\Sigma^3 K_3 && \lTo &...\\
    &   &  &    \luTo  & & \ldTo && \luTo && \ldTo&& \luTo && \ldTo\\
    &      &  &   &  H^*\Sigma X_1 & &&& H^*\Sigma^2 X_2 & &&& H^*\Sigma^3 X_3
\end{diagram}

By exactness of 
\[H^*\Sigma^i X_i\to H^*\Sigma^{i-1}K_{i-1}\to H^*\Sigma^{i-1}X_{i-1}\]
this gives an $\A$-free resolution of $H^*(X)$.  

Next, we construct what is known in the world of spectral sequences as an ``exact couple''.  
Let $Y$ be a spectra and 
\[E_1=\bigoplus_i [Y,K_i]_*\]
and
\[A_1=\bigoplus_i [Y,X_i]_*\]

Consider the cofibration sequence
\[X_{i+1}\xrightarrow{i} X_i\xrightarrow{p} K_i\xrightarrow{\partial} \Sigma X_{i+1}\]

This induces a diagram
\begin{diagram}
  A_1 & &\rTo^{i_1} & & A_1 \\
  &\luTo^{\partial_1} & & \ldTo^{p_1}\\
  && E_1
\end{diagram}
which is exact at each node.  

Now, consider
\[d_1=\partial_1p_1\]
and note that this is a differential on $E_1$ since $d_1^2=\partial_1(p_1\partial_1)p_1=0$.  
Thus, we can let $E_2$ be the cohomology of $(E_1,d_1)$ and $A_2$ be the image of $i_1$.  
Let $i_2$ be the restriction, $\partial_2$ be the quotient of $\partial_1$ and define $p_2$ to the image of $i_1$, $p_2$ defined by 
\[p_2(ia)=[p_1(a)]\]
It is an easy exercise to check that this all makes sense, is well defined and that the diagram
\begin{diagram}
  A_2 & &\rTo^{i_2} & & A_2 \\
  &\luTo^{\partial_2} & & \ldTo^{p_2}\\
  && E_2
\end{diagram}
is exact at each point.  This is done in, for instance \cite[Ch~1]{HatcherSS}.

Thus we can iterate this construction and get a sequence of modules $E_r$ and $A_r$.  
We will see later that the limit, which we call $E_\infty$, is closely related to $[Y,X]\otimes \F_2$
We denote
\[E_1^{s,t}=[Y,K_s]_t\cong Hom_\A^t(H^*(K_s),H^*(Y))\]
where the second equivalence comes from the argument in the introduction.  
Since the $H^*(K_s)$ make an $\A$-free resolution of $H^*(X)$, we have
\[E_2^{s,t}=Hom_\A^{s,t}(H^*(X),H^*(Y))\]


Let us unravel what this means.  
Our exact couple unravels into the following diagram where the ``staircases'' are exact
\begin{diagram}
  \vdots &&&& \vdots  &&&& \vdots \\
  \dTo & &&& \dTo  & &&& \dTo &&& &  \\
  [Y,X_s]_{t-s+1} & \rTo & [Y,K_{s}]_{t-s+1} & \rTo & [Y,X_{s+1}]_{t-s} & \rTo & [Y,K_{s+1}]_{t-s} & \rTo & [Y,X_{s+2}]_{t-s-1}\\
  \dTo & &&& \dTo  & &&& \dTo &&& &  \\
  [Y,X_{s-1}]_{t-s+1} & \rTo & [Y,K_{s-1}]_{t-s+1} & \rTo & [Y,X_{s}]_{t-s} & \rTo & [Y,K_{s}]_{t-s} & \rTo & [Y,X_{s+1}]_{t-s-1}\\
  \dTo & &&& \dTo  & &&& \dTo &&& &  \\
  [Y,X_{s-2}]_{t-s+1} & \rTo & [Y,K_{s-2}]_{t-s+1} & \rTo & [Y,X_{s-1}]_{t-s} & \rTo & [Y,K_{s-1}]_{t-s} & \rTo & [Y,X_{s}]_{t-s-1}\\
  \dTo & &&& \dTo  & &&& \dTo &&& &  \\
  \vdots &&&& \vdots  &&&& \vdots 
\end{diagram}

How do we calculate the differentials.  
Put your pencil on the module in the center row: $[Y,K_{s-1}]_{t-s+1}$.
Pretend your pencil tip is $x\in[Y,K_{s-1}]_{t-s+1}$
By definition, the map $d_1$ is obtained by going straight across.  
If $d_1(x)=0$, then $x$ represents an element in $E_2$ and so $d_2$ is expected to be defined.  
To calculate $d_2$, move your pencil to $\partial_1(x)\in [Y,X_s]_{t-s}$.  
By exactness, since $x$ is zero in $[Y,K_s]_{t-s}$, there is a preimage of $x$ above in $[Y,X_{s+1}]_{t-s}$.  
Applying $p_1$ to this will give an element of $[Y,K_{s+1}]_{t-s}$, and this is $d_2([x])$.  
In general, a differential is calculated by pushing $\partial_1(x)$ ``up'' as far as you can before applying $p_1$.  

For spheres, we can do a bit better.  
Exactly like with spaces (you should check this), there is a relative homotopy group
\[\pi_i(A,X)=[(D,S^{-1}),(X,A)]_i\]
where $D=\Sigma^\infty D^0$ and the homotopies are to leave the boundary of $D$ in $A$.  
Rewriting our diagram like this for $Y=S$, we have

\begin{diagram}
  \vdots &&&& \vdots  &&&& \vdots \\
  \dTo & &&& \dTo  & &&& \dTo &&& &  \\
  \pi_{t-s+1}X_s & \rTo & \pi_{t-s+1}(X_s,X_{s+1}) & \rTo & \pi_{t-s}X_{s+1} & \rTo & \pi_{t-s}(X_{s},X_{s+1}) & \rTo & \pi_{t-s-1}X_{s+2}\\
  \dTo & &&& \dTo  & &&& \dTo &&& &  \\
  \pi_{t-s+1}X_{s-1} & \rTo & \pi_{t-s+1}(X_{s-1},X_{s}) & \rTo & \pi_{t-s}X_{s} & \rTo & \pi_{t-s}(X_{s-1},X_{s}) & \rTo & \pi_{t-s-1}X_{s+1}\\
  \dTo & &&& \dTo  & &&& \dTo &&& &  \\
  \pi_{t-s+1}X_{s-2} & \rTo & \pi_{t-s+1}(X_{s-2},X_{s-1}) & \rTo & \pi_{t-s}X_{s-1} & \rTo & \pi_{t-s}(X_{s-2},X_{s-1}) & \rTo & \pi_{t-s-1}X_{s}\\
  \dTo & &&& \dTo  & &&& \dTo &&& &  \\
  \vdots &&&& \vdots  &&&& \vdots 
\end{diagram}

You can, without loss of generality (use a mapping-cylinder construction) assume all the maps $X_i\to X_{i-1}$ are injections.  
Thus, let
\[f:(D^{t-s+1},S^{t-s})\to (X_s,X_{s+1})\]
be an element of $\pi_{t-s+1}(X_{s-1},X_{s})$.  
Then $\partial_1(f)$ is the boundary of $f$, that is, $f|_{S^{t-s}}$.  
The image of $\partial_1(f)$ is in $X_{s}$, but you may be able to find a homotopy compressing the image to $X_{s+r-1}$.
Thus $d_r(f)$ is the inclusion of this map into $\pi_{t-s}(X_{s+r-1},X_{s+r})$.  
This is the strategy we will use to compute the vast majority of the differentials.  

\subsection{Convergence of the Adams Spectral Sequence}

Before we go any further in discussing how to calculate $E_\infty$, let us prove the following result
\begin{Theorem}
  Use the notation of above.  Let 
  \[F^{s,t}=Im([Y,X_s]_{t-s}\to [Y,X]_{t-s}\]
  Then 
  \[\bigcap_n F^{s+n,t+n}=[Y,X]_{t-s}\otimes \F_2\]
  and for each $(s,t)$ there is an $R$ such that
  \[E_R^{s,t}=\frac{F^{s,t}}{F^{s+1,t+1}}\]
  We call $E_R^{s,t}$ by $E_\infty^{s,t}$.  
\end{Theorem}


