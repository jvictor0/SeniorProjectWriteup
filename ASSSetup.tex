\section{Setting up the Adams Spectral Sequence}

The Adams Spectral Sequence is a rather heavy-duty machine for computing $2$-component of the homotopy groups $[Y,X]_*$ for spectra $X$ and $Y$.  
If $Y=S$, this is the 2-component of the stable homotopy groups $\pi^s_*(X)$, which, if you'd like, can be written $\pi^s(X)\otimes \Z_2$, where $\Z_2$ is the 2-adic integers.  
While this is twice removed from the original goal of computing $\pi_*(X)$, but at least in this case we have a fighting chance at doing the computations.
Since there are $p$-component analogues, these methods can be combined to compute all of $\pi_*^s(X)$.  
However, as we are about to find, even once the Adam's Spectral Sequence is set up, actually using it is not so easy.  

Morally, the Adams Spectral Sequence works as follows.  
We start with a ``geometric'' resolution of the space $Y$
\[X\to K_1\to K_2\to K_3\to ...\]
such that each $K_i$ is a wedge sum of Eilenberg-Maclane spaces and the sequence induced in mod-2 cohomology is exact.  
Thus we get an $\A$-free resolution of $H^*(Y)$, and we consider 
\[[Y,X]\from [Y,K_1]\from [Y,K_2]\from ...\]
Since $K_i$ is a wedge sum of $H\F_2$, we have for $H^*$ being mod-2 cohomology
\[[Y,K_i]\cong \bigoplus_i [Y,H\F_2]\cong \bigoplus_i H^*(Y) \cong \bigoplus_i Hom_\A(\A,H^*(Y))\cong Hom_\A(H^*(K_i),H^*(Y))\]
The reason we needed to develop spectra is now clear: A space $X$ cannot have $\A$-free cohomology.  
We can take homology of this chain complex and get $Ext_\A(H^*(X),H^*(Y))$, the starting point or so-called ``$E_2$'' term of the Adams Spectral Sequence.  
We will show this is a sort of over-approximation of $[Y,X]_*\otimes \F_2$, and the Adams Spectral Sequence ``converges'' to this.  
In the literature of spectral sequences, one might write
\[Ext_\A(H^*(X),H^*(Y))\implies [Y,X]\otimes \F_2\]
Computing this Ext group can often be tough, but is algebraic in nature and usually more-or-less mechanical.  
However, we are not done yet.  


The second step in running the Adams Spectral Sequence is to figure out exactly how the convergence works.
When we construct $Ext_\A(H^*(X),H^*(Y))$, we construct in it a way such that it is still a chain complex.
Thus we can take homology and get what we call the $E_3$ page.
Some elements of $E_2$ will not be cycles, and thus will disappear forever.  
Others will be related by boundaries and become equal.  
Thus $E_3$ is a sub-quotient of $E_2$, just as $E_2$ is a sub-quotient of $Hom_\A(H^*(X),H^*(Y))$.  
We can continue this process until all the differentials become zero, which we call $E_\infty$.  
It can be shown that the process does stabilize and the final answer, $[Y,X]\otimes \F_2$, can be read off the $E_\infty$ page. 
However, computing the differentials on each $E_r$ is notoriously difficult, in fact, it will be the goal of most of the rest of this paper.  
The differential computations are often geometric in nature; this is not surprising since the algebra of Ext cannot possibly be enough to determine all the homotopy groups.  

\subsection{The Adams Resolution}

Let $X$ and $Y$ be spectra, $X$ connective and of finite type, and let $H^*(?)=(H\F_2)^*(?)$ be mod-2 cohomology.  
We want to start by creating the geometric resolution of $Y$ described above.  
We construct the spaces of the resolution as follows.  
Consider $H^*(X)$, and assume it is finitely generated over $\A$.  
Call those generators $u_i\in [X,H\F_2]_{n_i}$.
We can then form 
\[\bigvee_i u_i:X\to \bigvee_i \Sigma^{n_i}H\F_2\] 
to be a degree 0 map to a wedge sum of suspensions of Eilenberg-Maclane spaces.  
Call the codomain above $K_0$.  
We can then take the fiber of this map and call it $X_1$.  
Repeating this process, we get the following diagram. 

\begin{diagram}
  X & \lTo & X_1 & \lTo & X_2 & \lTo & X_3 & \lTo &...\\ 
  \dTo & & \dTo && \dTo && \dTo\\
  K_0 && K_1 && K_2 && K_3
\end{diagram}

This is one specific construction of the following definition:
\begin{Def}
    An Adams Complex of a spectrum $X$ is a diagram, as above, where $K_0$ is a wedge sum of suspensions of $H\F_2$'s, $X_{i+1}$ is the fiber of $X_i\to K_i$.   

    An Adams Resolution of a spectrum $X$ is an Adams Complex of $X$ with $X_{i}\to K_i$ inducing a surjection in cohomology.  
\end{Def}


Consider an Adams Resolution of $X$.  
Now, notice that the fiber of $X_i \to X_{i-1}$ is $\Sigma^{-1}K_{i-1}$.  
Thus we get a diagram of spaces
\begin{diagram}
  X & \rTo&  K_0 & &\rTo && \Sigma K_1 && \rTo && \Sigma^2 K_2 && \rTo && \Sigma^3 K_3 && \rTo &...\\
    &   &  &    \rdTo  & & \ruTo && \rdTo && \ruTo&& \rdTo && \ruTo\\
    &      &  &   &  \Sigma X_1 & &&& \Sigma^2 X_2 & &&& \Sigma^3 X_3
\end{diagram}

In cohomology we get:

\begin{diagram}
  H^*X & \lTo&  H^*K_0 & &\lTo && H^*\Sigma K_1 && \lTo && H^*\Sigma^2 K_2 && \lTo && H^*\Sigma^3 K_3 && \lTo &...\\
    &   &  &    \luTo  & & \ldTo && \luTo && \ldTo&& \luTo && \ldTo\\
    &      &  &   &  H^*\Sigma X_1 & &&& H^*\Sigma^2 X_2 & &&& H^*\Sigma^3 X_3
\end{diagram}

By exactness of 
\[H^*\Sigma^i X_i\to H^*\Sigma^{i-1}K_{i-1}\to H^*\Sigma^{i-1}X_{i-1}\]
this gives an $\A$-free resolution of $H^*(X)$.  

(Notice that if this was just a complex, the sequence of maps in cohomology would still be a complex).   

Next, we construct what is known in the world of spectral sequences as an ``exact couple''.  
Let $Y$ be a spectra and 
\[E_1=\bigoplus_i [Y,K_i]_*\]
and
\[A_1=\bigoplus_i [Y,X_i]_*\]

Consider the cofibration sequence
\[X_{i+1}\xrightarrow{i} X_i\xrightarrow{p} K_i\xrightarrow{\partial} \Sigma X_{i+1}\]

This induces a diagram
\begin{diagram}
  A_1 & &\rTo^{i_1} & & A_1 \\
  &\luTo^{\partial_1} & & \ldTo^{p_1}\\
  && E_1
\end{diagram}
which is exact at each node.  

Now, consider
\[d_1=\partial_1p_1\]
and note that this is a differential on $E_1$ since $d_1^2=\partial_1(p_1\partial_1)p_1=0$.  
Thus, we can let $E_2$ be the cohomology of $(E_1,d_1)$ and $A_2$ be the image of $i_1$.  
Let $i_2$ be the restriction, $\partial_2$ be the quotient of $\partial_1$ and define $p_2$ to the image of $i_1$, $p_2$ defined by 
\[p_2(ia)=[p_1(a)]\]
It is an easy exercise to check that this all makes sense, is well defined and that the diagram
\begin{diagram}
  A_2 & &\rTo^{i_2} & & A_2 \\
  &\luTo^{\partial_2} & & \ldTo^{p_2}\\
  && E_2
\end{diagram}
is exact at each point.  This is done in, for instance \cite[Ch~1]{HatcherSS}.

Thus we can iterate this construction and get a sequence of modules $E_r$ and $A_r$ (when $X$ and $Y$ are ambiguous, we write $E_r(Y,X)$ and $A_r(Y,X)$)
\begin{Def}
  We call the sequence of groups $E_r$ the Adams Spectral Sequence
\end{Def}
We will see later that the limit, which we call $E_\infty$, is closely related to $[Y,X]\otimes \F_2$


This construction, at first, can be rather disorienting.  
The issue is that $E_{r+1}$ is a sub-quotient (a quotient of a subgroup) of $E_r$.
That means when going from $E_r$ to $E_{r+1}$, some elements will become equal to others, while some elements will cease to exist.
If $x\in E_r$ has $d_r(x)=0$, then $x$ has some image in $E_{r+1}$, so we say that $x$ survives to the $r+1$ page, and we use the same symbol to denote it on the $E_{r+1}$ page.  
This is not as confusing as it might sound, because geometrically $x$ is represented by the same map on the $E_r$ page and the $E_{r+1}$ page.  
If $d_r(x)\ne 0$, we say the differential kills $x$.  
Notice that we say the $x$ survives even if there is some other element $y$ with $d_r(y)=x$, that is, even if $x$ is a boundary and the image of $x=0$ in $E_{r+1}$.


We recall
\[E_1^{s,t}=[Y,K_s]_t\cong Hom_\A^t(H^*(K_s),H^*(Y))\]
where the second equivalence comes from the argument in the introduction.  
Since the $H^*(K_s)$ make an $\A$-free resolution of $H^*(X)$, we have
\[E_2^{s,t}=Hom_\A^{s,t}(H^*(X),H^*(Y))\]


Let us unravel what this means.  
Our exact couple unravels into the following diagram where the ``staircases'' are exact
\begin{diagram}
  \label{fig:exactCouple}
  \vdots &&&& \vdots  &&&& \vdots \\
  \dTo & &&& \dTo  & &&& \dTo &&& &  \\
  [Y,X_s]_{t-s+1} & \rTo & [Y,K_{s}]_{t-s+1} & \rTo & [Y,X_{s+1}]_{t-s} & \rTo & [Y,K_{s+1}]_{t-s} & \rTo & [Y,X_{s+2}]_{t-s-1}\\
  \dTo & &&& \dTo  & &&& \dTo &&& &  \\
  [Y,X_{s-1}]_{t-s+1} & \rTo & [Y,K_{s-1}]_{t-s+1} & \rTo & [Y,X_{s}]_{t-s} & \rTo & [Y,K_{s}]_{t-s} & \rTo & [Y,X_{s+1}]_{t-s-1}\\
  \dTo & &&& \dTo  & &&& \dTo &&& &  \\
  [Y,X_{s-2}]_{t-s+1} & \rTo & [Y,K_{s-2}]_{t-s+1} & \rTo & [Y,X_{s-1}]_{t-s} & \rTo & [Y,K_{s-1}]_{t-s} & \rTo & [Y,X_{s}]_{t-s-1}\\
  \dTo & &&& \dTo  & &&& \dTo &&& &  \\
  \vdots &&&& \vdots  &&&& \vdots 
\end{diagram}

How do we calculate the differentials.  
Put your pencil on the module in the center row: $[Y,K_{s-1}]_{t-s+1}$.
Pretend your pencil tip is $x\in[Y,K_{s-1}]_{t-s+1}$
By definition, the map $d_1$ is obtained by going straight across.  
If $d_1(x)=0$, then $x$ represents an element in $E_2$ and so $d_2$ is expected to be defined.  
To calculate $d_2$, move your pencil to $\partial_1(x)\in [Y,X_s]_{t-s}$.  
By exactness, since $x$ is zero in $[Y,K_s]_{t-s}$, there is a preimage of $x$ above in $[Y,X_{s+1}]_{t-s}$.  
Applying $p_1$ to this will give an element of $[Y,K_{s+1}]_{t-s}$, and this is $d_2([x])$.  
In general, a differential is calculated by pushing $\partial_1(x)$ ``up'' as far as you can before applying $p_1$.  

Notice that if $f\in [Y,X]$ is detected by an element in filtration $s$, we can factor it as the composite of $s$ maps which induce 0 in cohomology.  
To see this, see that $f$ is detected by a map $[K_s,X]$, which can be pushed to a map $[Y,X_s]$ which is a lift of $f$.  
Thus $f$ factors as the $s-1$ maps in $[X_{l}, X_{l-1}]$ for $l<s$ (these are 0 in cohomology) and the composite of $Y\to X_s\to X_{s-1}$.
We state this easy fact as a lemma.
\begin{Lemma}
  \label{sec:filtlemma}
  If $f\in [Y,X]$ is detected by an element in filtration $s$, then it can factor as the composite of $s$ maps, each of which induce the zero map in cohomology.
\end{Lemma}


For spheres, we can do a bit better.  
Exactly like with spaces (you should check this), there is a relative homotopy group
\[\pi_i(A,X)=[(D,S^{-1}),(X,A)]_i\]
where $D=\Sigma^\infty D^0$ and the homotopies are to leave the boundary of $D$ in $A$.  
Rewriting our diagram like this for $Y=S$, we have

\begin{diagram}
  \vdots &&&& \vdots  &&&& \vdots \\
  \dTo & &&& \dTo  & &&& \dTo &&& &  \\
  \pi_{t-s+1}X_s & \rTo & \pi_{t-s+1}(X_s,X_{s+1}) & \rTo & \pi_{t-s}X_{s+1} & \rTo & \pi_{t-s}(X_{s},X_{s+1}) & \rTo & \pi_{t-s-1}X_{s+2}\\
  \dTo & &&& \dTo  & &&& \dTo &&& &  \\
  \pi_{t-s+1}X_{s-1} & \rTo & \pi_{t-s+1}(X_{s-1},X_{s}) & \rTo & \pi_{t-s}X_{s} & \rTo & \pi_{t-s}(X_{s-1},X_{s}) & \rTo & \pi_{t-s-1}X_{s+1}\\
  \dTo & &&& \dTo  & &&& \dTo &&& &  \\
  \pi_{t-s+1}X_{s-2} & \rTo & \pi_{t-s+1}(X_{s-2},X_{s-1}) & \rTo & \pi_{t-s}X_{s-1} & \rTo & \pi_{t-s}(X_{s-2},X_{s-1}) & \rTo & \pi_{t-s-1}X_{s}\\
  \dTo & &&& \dTo  & &&& \dTo &&& &  \\
  \vdots &&&& \vdots  &&&& \vdots 
\end{diagram}

You can, without loss of generality (use a mapping-cylinder construction) assume all the maps $X_i\to X_{i-1}$ are injections.  
Thus, let
\[f:(D^{t-s+1},S^{t-s})\to (X_s,X_{s+1})\]
be an element of $\pi_{t-s+1}(X_{s-1},X_{s})$.  
Then $\partial_1(f)$ is the boundary of $f$, that is, $f|_{S^{t-s}}$.  
The image of $\partial_1(f)$ is in $X_{s}$, but you may be able to find a homotopy compressing the image to $X_{s+r-1}$.
Thus $d_r(f)$ is the inclusion of this map into $\pi_{t-s}(X_{s+r-1},X_{s+r})$.  
This is the strategy we will use to compute the vast majority of the differentials.  

\subsection{Convergence of the Adams Spectral Sequence}

Before we go any further in discussing how to calculate $E_\infty$, let us prove the following result
\begin{Theorem}
  Use the notation of above.  Let 
  \[F^{s,t}=Im\left([Y,X_s]_{t-s}\to [Y,X]_{t-s}\right)\]
  Then 
  \[\bigcap_n F^{s+n,t+n}=\mbox{Torsion}_{p>2}[Y,X]_{t-s}\]
  where $\mbox{Torsion}_{p>2}$ means the set of all elements annihilated by a power of an odd prime.  
  Finally, for each $(s,t)$ there is an $R$ such that for all $r\ge R$
  \[E_R^{s,t}=\frac{F^{s,t}}{F^{s+1,t+1}}\]
  We call $E_r^{s,t}$ by $E_\infty^{s,t}$.  
  We can write this compactly as
  \[E_2=Ext^{s,t}_\A(H^*(X),H^*(Y))\implies [Y,X]_{t-s}\otimes \Z_2\]
\end{Theorem}



We need a lemma to prove the rest.  
First of all.
\begin{Lemma}
  Adams Resolutions are ``comparable'', that is, given spectra $X$ and $Y$, $f:X\to Y$ and Adams Resolutions $X_i$ and a complex $Y_i$, you can find $f_i:X_i\to Y_i$ making the following diagram commutative
  \begin{diagram}
    X & \lTo & X_1 & \lTo & X_2 & \lTo & ...\\
    \dTo^f & & \dTo^{f_1} && \dTo^{f_2}\\
    Y & \lTo & Y_1 & \lTo & Y_2 & \lTo & ...
  \end{diagram}
\end{Lemma}

\begin{proof}
  Let $K_i$ be the cofiber of $X_{i+1}\to X_i$, $L_i$ the cofiber of $Y_{i+1}\to Y_i$, and recall that the suspensions of the $K_i$ and $L_i$ give free resolutions of $X$ and $Y$ in cohomology.  
By the comparison theorem for cohomology, we can find $\hat{f}^*_i$ in cohomology such that the following diagram commutes
  \begin{diagram}
    H^*X & \lTo & H^*K_0 & \lTo & H^*\Sigma K_1 & \lTo & ...\\
    \uTo^{f^*} & & \uTo^{\hat{f}^*_1} && \uTo^{\hat{f}^*_2}\\
    H^*Y & \lTo & H^*L_0 & \lTo & H^*\Sigma L_1 & \lTo & ...
  \end{diagram}
But recall that that
\[[?,K_i]\cong Hom_\A(H^*(K_i),H^*(?))\]
And thus $\hat{f}^*_i$ is induced by $\hat{f}_i:K_i\to L_i$.  
But $K_i$, $K_{i+1}$ and $X_i$ form one distinguished triangle, $L_i$, $L_{i+1}$ and $Y_i$ form another, and we have maps from the $K$'s to the $L$'s, so we automatically get a map $f_i:X_i\to Y_i$ so that everything commutes. 
\end{proof}

The proof from here on out will proceed much like in \cite[Ch~2]{HatcherSS}.

Now, let us focus our attention of $\bigcap_i F^{s+i,t+i}$.  
Recall in \ref{fig:exactCouple} that $[Y,K_i]_{l}$ is a $\F_2$ vector space, and thus has no odd prime torsion.
The staircase is exact, so the vertical maps must be isomorphisms on the odd prime torsion. 
Thus the odd prime torsion in $[X,Y]_{t-s}$ is passed all the way down from $[Y,X_s]_{t-s}$ to $[Y,X]_{t-s}$ and so $F^{s,t}$ contains it for each $(s,t)$.

For the other direction, pick some integer $k$.  
Since $[X,X]$ is an abelian group, we can take the identity and multiply it by $2^k$, which we will let denote a map.
Let $Q$ be the cofiber of this map, so that we have the long exact sequence
\[...\to [Y,X]_i\xrightarrow{(*2^k)} [Y,X]_i \to [Y,Q]_i\to ...\]
and note by exactness that the image of the map $[Y,Q]_i\to [Y,X]_{i-1}$ is all $2$-torsion, as is the kernel of $[Y,X]_i\to [Y,Q]_i$, and thus $[Y,Q]_i$ is all 2-torsion.
If $\alpha\in [Y,X]_i$ is either odd prime torsion or non-torsion, then, by our connective and finite-type hypothesis implies $[Y,X]_i$ is finitely generated, there is a $k$ such that $\alpha$ is not divisible by $2^k$.  
Thus $\alpha$ is not in the image of $(*2^k)$, and so has nonzero image in $[Y,Q]_i$.  
By the comparison theorem, if $\alpha$ has a preimage in $[Y,X_j]_i$ for all $j$, then the image of $\alpha$ in $[Y,Q]_i$ will have a similar property for the Adams resolution of $Q$.  
Thus it is sufficient to prove that if $[Y,X]_i$ is all 2-torsion then the Adams resolution eventually becomes has $[Y,X]_i=0$.  

To show this, assume $X=Z_0$ is all 2-torsion.
Note that all the $[Y,X]_i$ are finite.  
We inductively build an Adams Complex
\begin{diagram}
  Z_0 & \lTo & Z_1 & \lTo & Z_2 & \lTo & Z_3 & \lTo &...\\ 
  \dTo & & \dTo && \dTo && \dTo\\
  L_0 && L_1 && L_2 && L_3
\end{diagram}
let $n_i$ be the smallest number with $[Y,Z_i]_{n_i}\ne 0$, and let $L_i$ be a wedge sum of $H\F_2$ on a basis for $H^{n_i}(Z_i)$.  
Let the map from $X_i\to L_i$ be the obvious one and let $Z_{i+1}$ be the cofiber.  
Notice that in $H^{n_i}$ the map $Z_i\to L_i$ is an isomorphism, so also in $H_{n_i}$, so in $[Y,?]_k$ for $k< n_i$ it is an isomorphism and we have
\[[Y,L_i]_{n_i}=[Y,Z_i]_{n_i}\otimes \F_2\]
This is a surjection, so by the cofiber sequence we have for $k<n_i$ the group $[Y,Z_{i+1}]_k=0$ and $[Y,Z_{i+1}]_{n_i}$ is smaller than $[Y,Z_{i}]_{n_i}$.  
Since these groups are finite, we must eventually get $[Y,Z_{i+1}]_{n_i}=0$.  
This means for each $i$, there is an $n$ such that $[Y,Z_{i+1}]_i=0$ for $i\ge n$.  
Applying the comparison theorem over the identity map $X\to X$, 
we find if any element in $[Y,X]_{t-s}$ has preimage in $[Y,X_s]_{t-s}$ for all $s$ (recall to make this sequence we take suspensions, which is equivalent to a grating shift), that element would have nonzero image in $[Y,Z_s]_{t-s}$ for all $s$, which we just said is impossible.  
Thus the intersection of the $F^{s+n,t+n}$ must be only odd prime torsion.  


We can finally prove the convergence result.
Recall that $A_r^{s,t}$ is all the elements of $[Y,X_s]_{t-s}$ with vertical preimages in $[Y,X_{s+r}]_{t-s}$.  
By that which has been proved thus far, for sufficiently large $r$ this contains no 2-torsion.  
Also, the map $A_r^{s,t}\to A_r^{s-1,t-1}$ is an isomorphism on the non torsion and odd prime torsion, so this map is injective.
Thus, recalling the definition of the differential $d_r$, since the map $E_r^{s,t}\to A_r^{s,t}$ is 0 by exactness of the staircase, for large $r$ the differentials originating at $E_r^{s,t}$ are zero.
Also for large enough $r$ there are no differentials into $E_r^{s,t}$, since such differentials would come from $E_r^{s-r,t-r-1}$, which is nothing for $r>s$.  
Thus for all $r$ greater than some $R$, we have that projection $E_r^{s,t}\cong E_{r+1}^{s,t}$.
Notice that in fact $E_\infty^{s,t}$, by exactness of the staircase, is isomorphic to the cokernel of the previous vertical maps, which for large $r$ is exactly the inclusion $F^{s+1,t+1}\to F^{s,t}$, which is the theorem.  


\subsection{Some Remarks}

There are a few ways to generalize this process or just make it a bit nicer.
First of all, we can use homology instead of cohomology.  
The difference here is that we end up using smash products instead of wedge products of $H\F_2$, but in the end we get a spectral sequence
\[E_2 = Ext_{A_*}(H_*(Y),H_*(X))\implies [Y,X]\otimes \Z_2\]
where the $\implies$ symbol means that there is an $E_\infty$ page and it is isomorphic successive quotients of the right hand of the $\implies$ arrow, and
\[A_*=Hom(A,\F_2)=(H\F_2)_*(H\F_2)\]
We can also replace $H\F_2$ with any generalized cohomology theory $E$, like cohomology mod odd primes, 
you're favorite flavor of K-Theory (for instance $E=BU,BO,KO$) or cobordism ($E=MU,MSU,BP)$).  
The zoo of spectra and cohomology theories are discussed in great detail in Ravenel's famous ``Green Book'' \cite{RavenelGreen}.
In this case, under certain hypothesis about $E,X$ and $Y$, we have
\[E_2 = Ext_{E_*E}(E_*(Y),E_*(X))\implies [Y,X]^{E}\]
where $[Y,X]^{E}$ is, roughly, maps $f$ whose equivalence is detected by the induced map $E_*(f)$.  
More details on this can be found in, for instance \cite{RavenelGreen}, \cite[Ch~IV]{H00RingSpectra}.  
When $E=BP$, the spectral sequence is has many many fewer nonzero differentials, and earns the name ``Adams-Novikov Spectral Sequence''.  


\subsection{Hopf Invariant One Maps in $Ext_\A(\F_2,\F_2)$}

Let $f\in \pi_{4n-1}(S^{2n})$ have odd Hopf Invariant.  
Since the group $\pi_{4n}(S^{2n+1})$ is stable and the Fruedenthal map is a surjection on $\pi_{4n-1}(S^{2n})\to \pi_{4n}(S^{2n+1})$, there is an $\hat{f}=\Sigma f\in\pi_{2n-1}^s(S)$ coming from $f$.  
Notice that $\hat{f}\ne 0$, since the Steenrod Squares commute with suspensions, so in the cohomology ring $S^{2n+1}\cup_{\Sigma f} D^{4n+1}$ the Steenrod Square $S^{2n}$ is not zero.  
Notice also that in $S^{2n+1}\cup_{f} D^{4n+1}$ for some $g\in \pi_{4n}(S^{2n+1})$, we cannot ask what the cup product square on the $2n+1$ cohomology class is.
However, we can still apply $Sq^{2n}$ and ask if it is zero or not.  
If $g$ is the suspension of a map with a Hopf invariant, this will detect the parity of the Hopf Invariant.
Since Hopf Invariant is a homomorphism and nullhomotpic maps have Hopf Invariant 0, we know that $\Sigma f$ has even order.  
This means that $\hat{f}$ is detected by an element in the Adams Spectral Sequence.  


We can do better however.  We are able to say exactly what elements of $Ext_\A(\F_2,\F_2)$ correspond to possible maps of Hopf Invariant One.  
We need the following lemma
\begin{Lemma}
  Suppose that $\hat{f}:S^{2n-1}\to S$ comes from an element of odd Hopf Invariant.  
  Suppose there is a spectrum $X$ such that the following diagram commutes
  \begin{diagram}
    S^{2n-1} & \rTo^{f_1} & X\\
             & \rdTo_{\hat{f}} & \dTo^{f_2}\\
             &            &   S
  \end{diagram}
  Then either $f_1^*$ or $f^*_2$ is nonzero.
\end{Lemma}

\begin{proof}
  Suppose both maps are zero in cohomology.  
  We have the following diagram of spectra
  \begin{diagram}
      &            & S^{2n-1}\\
      &            & \dTo^{f_1}\\
    S & \lTo^{f_2} & X \\
    \dTo &  & \dTo^j\\
    S\cup_{\hat{f}} D^{2n} & \lTo^i & X\cup_{f_1} D^{2n}\\
    \dTo & & \dTo\\
    S^{2n} & \lEq & S^{2n}
  \end{diagram}
where the two columns are cofibrations.  
Applying cohomology, we have that the two columns are exact.  
In fact, since $f_1^*=0$, the right column is short-exact.  
  \begin{diagram}
    \F_2 & \rTo^{0} & H^*X\\
    \uTo &  & \uOnto_{j^*}\\
    \F_2\oplus \Sigma^{2n}\F_2 & \rTo^{i^*} & H^*(X\cup_{f_1} D^{2n})\\
    \uTo & & \uInto\\
    \Sigma^{2n}\F_2 & \lEq & \Sigma^{2n}\F_2\\
  \end{diagram}
  Let $\alpha$ be the generator of $H^0( S\cup_{\hat{f}} D^{2n})$ and $\beta$ be the generator of $H^{2n}( S\cup_{\hat{f}} D^{2n})$.  
  Since we know $\beta$ has a preimage in $H^*(S^{2n})$ and since the bottom right vertical map is an injection (by exactness), $i^*\beta$ generate the kernel of $j^*$ and is nonzero.  
  But $\beta=Sq^{2n}\alpha$, so $Sq^{2n}i^*\alpha=i^*\beta$, so $i^*\alpha\ne 0$ and thus is not in the kernel of $j^*$.  
  This means $j^*i^*\alpha\ne 0$.  But by commutativity of the square, it should be, so we get a contradiction.  
\end{proof}

\begin{Cor}
  If there is a map of odd Hopf Invariant in $\pi_{4n-1}(S^{2n})$, there is an element of $Ext^{1,2n}_\A(\F_2,\F_2)$ which survives to the $E_\infty$ page.  
\end{Cor}
\begin{proof}
  By Lemma \ref{sec:filtlemma}.
\end{proof}


Since the kernel of $\A\to \F_2$ is generated over $\F_2$ by the indecomposable elements, we can conclude that $Ext^{1,t}(\F_2,\F_2)$ is nonzero when and only when $t$ is a power of 2.  
We will refer to the generator of $Ext^{1,2^i}(\F_2,\F_2)$ as $h_i$.  
The element $h_0$ detects twice the identity map in $\pi_0(S)$, $h_1$ detects the Hopf Fibration and $h_2$ and $h_3$ detect similar maps for the Quaternions and Octonions.  



\subsection{Calculating $Ext_\A(H^*(X),\Z/2)$}

In the special case of stable homotopy groups, there is a relatively straightforward way to calculate $Ext$.  
Simply use a ``minimal'' resolution of $\A$, that is, a resolution constructed inductively with the least number of generators in each degree in each dimension and a preferred generator set.  
Let me describe the algorithm.
We want a resolution
\[H^*(X)\from F_1\from F_2\from F_3\from ...\]
The generators of $H^*(X)$ as an $\A$ module are the generators of $F_1$, mapping in the obvious way into $H^*(X)$.  
Construct $F_i$ inductively as follows.  Let $F_i^j$ be the degree $j$ elements in $F_i$, and $\A^i$ be elements of $\A$ of degree $i$.  
The algorithm will work by adding generators where needed, so let $G_i^j$ be the generators added in $F_i^j$.  
\[{F_i^j}'=\langle sg|g\in G^k_i, k<j, s\in \A^{j-k} \rangle\]
\[I=d({F_i^j}')\subseteq F_{i-1}^{j+1}\]
\[K=ker(d:{F_{i-1}^{j+1}}'\to {f_{i-2}^{j+2}}')\]
\[F_i^j= {F_i^j}'\oplus \langle G_i^j\rangle\]
Where $G_i^j$ is formed by adding generators to ${F_i^j}'$ until $d|_I:I\to K$ is an isomorphism.
Notice that we can use ${F_{i-1}^{j+1}}'$ instead of $F_{i-1}^{j+1}$ since the kernel of $d$ is the same on both modules.  
A computer can easily be made to do these computations, and it is easy to see that, by construction, these are free-$\A$ modules and the differentials are exact.  

Let $\sum_i s_ig_i$ be a homogeneous boundary, with the $g_i\in G_*^*$ and $s_i\in \A$.  
Then all the $s_i$ are homogeneous, and cannot $Sq^0$ by the construction.  
But any $\phi\in Hom_\A(F_*,\Z/2)$ will have 
\[\phi(\sum s_ig_i)=s_i(\sum \phi(g_i))=0\]
since $s_i\cdot 1=0$ fir $s_i$ is homogeneous and not $Sq^0$, so $d^*\phi=\phi\circ d=0$.  
Thus all differentials are zero, so
\[Ext_\A(H^*(X),\Z/2)=Hom_\A(F_*,\Z/2)\]
This is isomorphic to the $\F_2$ vector space on the same basis as $F_*$.  
I have implemented this algorithm.  


