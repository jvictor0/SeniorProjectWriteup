\section{Introduction and Prerequisites}

Before we can get started on this journey, we will need some preliminary results.  
We will not present every proof and every detail, but rather list the necessary results and make some remarks on their consequences. 
We will define homotopy groups, state some theorems about their consequences, define stable groups and introduce some important technical tools: the Steenrod Operations.

\subsection{Homotopy Groups}

\label{sec:HomotopyIntro}

\begin{Def}[Homotopy Groups]
  Let $S^n$ be the $n$-sphere and $X$ be a topological space with base-point.  
  Define the set 
  \[\pi_n(X)=[S^n,X]\]
  be the set of homotopy classes of base-point preserving maps.  
  Notice that $\pi_1(X)$ is the familiar fundamental group.  
\end{Def}

We need to define the group operation on these groups, which in fact we can do for the suspension of any space $Y$
Let $\Sigma$ be the reduced suspension functor, and 
\[f,g:\Sigma Y\to X\]
be pointed maps, so we can form
\[f\vee g :S^n\vee S^n\to X\]
Let $C$ be the cone functor.  
Since $\Sigma Y\cong CY/Y$, we can write $\Sigma Y\vee \Sigma Y\cong \Sigma Y/Y$, where we identify $Y$ as the ``equator'' of $\Sigma Y$.  
Letting $p: \Sigma Y\to \Sigma Y\vee \Sigma Y$ be the projection, we define
\[f+g=(f\vee g)\circ p : \Sigma Y\to X\]
It isn't hard to see this sum is well-defined on homotopy classes of pointed maps, homotopy-associative, has null-homotopic maps as identity and 
has inverses up to homotopy, given by ``swapping the cones'' in the suspension.
This means that $[\Sigma Y,X]$ is a group.  
Better yet, if $\Sigma^2Y=S^2\wedge Y$ is an iterated suspension, this is an abelian group, where the homotopy $f+g\sim g+f$ is given by ``rotating'' the $S^2$ smash-factor.  
It is an easy exercise to make these constructions precise, and they can be found in \cite{HatcherAT}.  Since $S^n=\Sigma^n(S^0)$, we have that $\pi_n(X)$ is a group and if $n\ge 2$ then it is an abelian group.  

The calculation of these homotopy groups is notoriously difficult.  
There is no analogy of Mayer-Vietoris or Siefert-Van Kampen, making it difficult to build homotopy groups of spaces from the homotopy groups of smaller spaces.
Even for the simplest and most ubiquitous spaces, spheres, these computations are a long (but, in my opinion, quite beautiful) journey.  
We can make a few quick computations, however.  
First of all, if $k<n$, then a map $S^k\to S^n$ is not essentially surjective, this such a map factors through $S^n-\{*\}=\R^n$, which is contractible.  
Thus $\pi_k(S^n)=0$.  
We also have that, since $S^n$ is simply-connected for $n>1$, we have that any map from $S^n\to S^1$ factors through the universal cover, $\R$, which again is contractible.  
This $\pi_n(S^1)=0$.  
You might be inclined to hope the rest of the $\pi_k(S^n)$ will fall this easily, but in fact this is the last of the easy computations.  

\subsection{The Hurewicz Map}

Just like $H_1$ is the abelianization of the fundamental group, we can relate the first nontrivial homology group to the first nontrivial homotopy group.  
We make the following terminology:
\begin{Def}
We say a space $X$ is $n$-connected if $\pi_i(X)=0$ for $i\le n$.
\end{Def}


There is a map 
\[h_n:\pi_n(X)\to H_n(X)\]
called the Hurewicz Map, given as follows.  Let $[f]\in \pi_n(X)$, and let $\alpha$ generate $H_n(S^n)$.  
Then $h([f])=f_*(\alpha)\in H_n(X)$.  
\begin{Theorem}[Hurewicz Theorem]
  Let $X$ be $n-1$ connected for $n>1$.  
  Then $h_n$ is an isomorphism and $h_{n+1}$ is a surjection.  
\end{Theorem}

This is proved in \cite[Thm~4.32]{HatcherAT}.

\subsection{The Whitehead Theorem}
The homotopy groups, and induced maps between them, completely determine the homotopy type of a CW-complexes.  

\begin{Theorem}[Whitehead Theorem]
  Let $f:X\to Y$ be a map of base-pointed CW-complexes, such that $f_*:\pi_n(X)\to \pi_n(Y)$ is an isomorphism.  Then $f$ is a homotopy equivalence.  
\end{Theorem}

This is proved in \cite[Thm~4.5]{HatcherAT}.  
The proof comes from the so-called Compression Lemma, which we state for it's usefulness.  It is proved in \cite[Thm~4.6]{HatcherAT}.

\begin{Lemma}[Compression Lemma]
  Let $f:(X,A)\to (Y,B)$ be a map of CW-pairs.  Suppose that for each $n$ such that $X-A$ has an $n$-cell, $\pi_n(Y/B)=0$.  Then $f$ is homotopic, relative to $A$, to a map $X\to B$.  
\end{Lemma}

\begin{Def}
  In the notation of the Lemma, we say that $f$ is ``compressed'' to a map into $B$.  If $\pi_n(Y/B)$ is nonzero, we call the nonzero elements obstructions to compression.  
\end{Def}

\subsection{The Freudenthal Suspension Theorem}

\label{sec:Freudenthal}

The homotopy groups of spheres are difficult to compute, but there is little a miracle which makes it possible to compute certain homotopy groups in a stable range.  
Let $\Omega$ be the loop functor, and recall the adjoint relationship
\[[\Sigma X, Y]=[X, \Omega Y]\]
The miracle is the Fruedenthal Suspension Theorem, which is stated as follows:
\begin{Theorem}[Freudenthal Suspension Theorem]
  Let $X$ be an $n$-connected space.  
  Then the natural map $X\to \Omega\Sigma X$ induces a map 
  \[\pi_k(X)\to \pi_k(\Omega\Sigma X)\cong\pi_{k+1}(\Sigma X)\]
  is an isomorphism for $k\le 2n$ and an epimorphism for $k=n2+1$.  
\end{Theorem}
\begin{Cor}
  If $n>k+1$, $\pi_{k+n}(S^n)$ is independent of $n$.
\end{Cor}

There are a few proofs of this.  A primitive homotopy based proof is given in \cite[Cor~4.24]{HatcherAT}.  
A proof using the Serre Spectral Sequence is given in \cite[Ch~12]{MosherTengora}.  
There is also a Morse Theory based proof given in \cite[Cor~22.3]{MilnorMorse}.  


Because of this theorem, we can define the so called Stable Homotopy Groups.  


\begin{Def}
  The  $k^{\mbox{th}}$ stable homotopy group of a space $X$ is given
  \[\pi_k^s(X)=\varinjlim_n \pi_{k+n}(\Sigma^n X) \]
\end{Def}

We will find that these so called Stable Homotopy Groups are somewhat easier to compute.  
The computation of these groups will be the goal of the rest of the paper.  


\subsection{Eilenberg-Maclane Spaces and the Cohomology Operations}

\begin{Def}[Eilenberg-Maclane Space]
  Let $G$ be an abelian group.  We say that a space $K(G,n)$ is an Eilenberg-Maclane Space if
  \[\pi_k(K(G,n))=\left\{\begin{array}{cc} G & n=k \\ 0 & n\ne k\end{array}\right.\]
\end{Def}
There is a construction of Eilenberg-Maclane spaces which has no $i$ cells for $i<n$, but unfortunately most are infinite-dimensional and complicated.  However, these spaces have enormous theoretic importance, for the reason we will see in a second  However, given any two, a map between them can be found inducing isomorphisms in homotopy groups, meaning $K(G,n)$ is unique up to $homotopy$. 

Now, consider 
\[H^n(K(G,n);G)\]
By the universal coefficient theorem, this is the same as
\[Hom(H_n(K(G,n)),G)\cong Hom(\pi_n(K(G,n)),G)\cong Hom(G,G)\]
using the Hurewicz Theorem.  
This means there is a cohomology class in $i\in H^n(K(G,n);G)$ corresponding to the identity in $Hom(G,G)$, which we will call the ``fundamental class''.  
Define
\[\Phi:[?,K(G,n)]\to H^n(?)\]
be the natural transformation defined by
\[\Phi([f]) = f^*(i)\]

\begin{Theorem}
  $\Phi$ is an isomorphism
\end{Theorem}

\begin{proof}
  First we notice that this works for $S^n$, since 
  \[H^n(S^n;G)\cong Hom(\Z,G)\cong G\]
  Let $X_i$ be the $i$-skeleton of $X$.  
  Then by CW-approximation, any $f:X_n\to K(G,n)$ is homotopic to a map of pairs
  \[f: (X_n,X_{n-1}) \to (K(G,n),*)\]
  since $K(G,n)$ can be given a cell-structure with no $n-1$ cells.  
  This means $f$ factors through a wedge of spheres 
  \[\bigvee S^n\]
  and so in homotopy, $f$ is essentially wedge sum of elements of $\pi_n(K(G,n))=G$, $\Phi$ surjects onto $C^*(X;G)$ at the cochain level, so also at the cohomology level.
  Thus $\Phi$ is surjective.  
  Since, at the cochain level, maps yielding the same cochain are homotopic, and since $\Phi$ is surely well-defined on cohomology, two maps yielding the same cohomology class are homotopic.  
  Thus $\Phi$ is also injective.  
  
  Letting $i:X_{n+1}\to X$ and $j:X_n\to X{n+1}$ be the inclusions, we have a diagram
  \begin{diagram}
    [X,K(G,n)] & \rTo^{i^*} & [X_{n+1},K(G,n)] & \rTo^{j^*} & [X_n,K(G,n)]\\
     \dTo^\Phi &            &   \dTo^{\Phi}    &            & \dTo_\Phi\\
      H^n(X;G) & \rTo^{i^*} & H^n(X_{n+1};G)   & \rTo^{j^*} & H^n(X_n;G)
  \end{diagram}
  Now, if a map $f:X_n\to K(G,n)$ can be extended to $X_{n+1}$, it can be extended all the way up to $X$, since $\pi_i(K(G,n))=0$ for $i>n$, 
  so the top $i^*$ is an isomorphism.  
  The top $j^*$ is an injection, since obstructions to extending a homotopy lies in $\pi_{n+1}(K(G,n))$, so homotopic maps from $X_n$ are homotopic in $X_{n+1}$. 
  Likewise, the lower $i^*$ is an isomorphism, again by cellular homology and $j^*$ is an injection.  
  Thus, because the diagram commutes ($\Phi$ is natural), $\Phi$ is an isomorphism for $X$.  
\end{proof}

This is a strange and remarkable theorem.  
Since we cannot usually visualize $K(G,n)$, this does not help with computation, but it makes the formal properties of $H^*$ obvious.  

We now introduce cohomology operations.  
\begin{Def}
  Let $\OO(n,\pi,m,G)$ be the set of natural transformations
  \[H^n(?;\pi)\to H^m(?;G)\]
  We call these cohomology operations
\end{Def}

\begin{Cor}
  \label{sec:SteenrodAreCohom}
  \[\OO(n,\pi,m,G)=H^m(K(\pi,n);G)\]
\end{Cor}

\begin{proof}
  Seeing $H^m(K(\pi,n);G)$ as maps $[K(\pi,n),K(G,m)]$, post-composition obviously gives natural transformations.  Given such a natural transformation, we can apply it to $i\in H^n(K(\pi,n);\pi)=[K(\pi,n),K(\pi,n)]$, yielding an element in $H^m(K(\pi,n);G)$.  Composition of these maps yields identity, since $i$ is the identity in $[K(\pi,n),K(\pi,n)]$.  
\end{proof}

This fact will become very important later.  


\subsection{The Steenrod Operations}

\label{sec:SteenrodCohom}

It turns out that we can explicitly describe the cohomology operations in $\F_2$ cohomology in terms of operations called Steenrod Squares.  
Loosely speaking, the Steenrod Squares measure the failure of the cup product square to be commutative.  
The $\F_2$-algebra they generate is known as the Steenrod Algebra, $\A$, and it will show up thoughout the calculations of the stable homotopy groups.  

We will first give an axiomatic description of the Steenrod Squares.
The let $H^n$ denote mod-2 cohomology.  

\begin{Theorem}[Steenrod Squares]
  For each $i\ge 0$ there is a natural map
  \[Sq^i:H^n(?)\to H^{n+i}(?)\]
  Such that
  \begin{enumerate}
    \item $Sq^0$ is the identity map
    \item $Sq^1$ is the Bockstien Map
    \item $Sq^n$ is the cup-product square
    \item $Sq^i$ is 0 for all $i>n$.  
    \item The Cartan formula holds:
      \[Sq^k(a\smile b)=\sum_{i+j=k}Sq^i(a)\smile Sq^j(b)\]
    \item The Adem relationship holds if $i<2j$
      \[Sq^iSq^j=\sum_{k=0}^{i/2}\binomial{j-k-1}{i-2k}Sq^{i+j-k}Sq^k\]
    \item The Steenrod Squares commute with the natural isomorphism $H^n(X)\to H^{n+1}(\Sigma X)$, that is, they are ``stable''
  \end{enumerate}
\end{Theorem}


The construction of these operations specifically is given in \cite[Ch~2]{MosherTengora}.  
We give a more general approach in \ref{sec:SteenrodConstruction}

We want to consider some of the structure of the Steenrod Algebra $\A$.  
First notice that by the Adem relation it is spanned by $Sq^{i_1}Sq^{i_n}...Sq^{i_k}$ where $i_j\ge i_{j+1}$.  
It is shown in \cite[Ch~6]{MosherTengora} that these elements are actually a basis, known as the Serre-Cartan Basis.  
The Steenrod Algebra is also, importantly, a Hopf Algebra, with comultiplication given
\[Sq^i\mapsto \sum_{i=j+k} Sq^j\otimes Sq^k\]
One can check that this, with the obvious unit and counit (sending $Sq^0$ to 1, other squares to 0 and 1 to $Sq^0$) is a Hopf Algebra.  

An easy application of the definition is to show the nonexistence of elements of Hopf Invariant 1 (see \ref{sec:HopfInvariant}).
For a space to have Hopf Invariant 1, it must have Hopf Invariant 1 (mod 2).  
Let $X=S^{2n}\cup_f e^{4n}$ be a space with Hopf Invariant, and let $\eta$ and $\mu$ be the two generators of cohomology.  
Then if the Hopf Invariant of $f$ is 1 (mod 2), $Sq^{2n}(\eta)=\mu$.
Now, suppose that we can decompose the operation (as a natural transformation, or equivalently, as an element of the Steenrod Algebra) $Sq^{2n}$ as a sum of products of Steenrod Squares $Sq^i$ with $i<2n$.  
This means that $Sq^{2n}(\eta)=0$, because for all $0<i<2n$ we have $H^{i+2n}(X)=0$, since there are no cells in these dimensions.  
But it is an easy exercise in using the Adem relation to show $Sq^i$ is decomposable if $i$ is not a power of 2.  
Thus we can conclude
\begin{Cor}
  There is no element of $\pi_{4n-1}(S^{2n})$ of Hopf Invariant 1 except possibly if $n$ is a power of $2$.  
\end{Cor}
\begin{Cor}
  $\R^n$ is not a division algebra except possibly if $n$ is a power of $2$.  
\end{Cor}


