\section{Spectral Sequence Diagram}
\label{sec:sseq}

\[ \begin{sseq}[xlabelstep=8,ylabelstep=4,entrysize=.3cm]{60}{30}
    \ssmoveto{0}{1}
\ssdropbull
\ssname{"A_1"}
\ssmoveto{1}{1}
\ssdropbull
\ssname{"A_2"}
\ssmoveto{3}{1}
\ssdropbull
\ssname{"A_4"}
\ssmoveto{7}{1}
\ssdropbull
\ssname{"A_8"}
\ssmoveto{15}{1}
\ssdropbull
\ssname{"A_{16}"}
\ssmoveto{31}{1}
\ssdropbull
\ssname{"A_{32}"}
\ssmoveto{63}{1}
\ssdropbull
\ssname{"A_{64}"}
\ssmoveto{0}{2}
\ssdropbull
\ssname{"B_2"}
\ssmoveto{2}{2}
\ssdropbull
\ssname{"B_4"}
\ssmoveto{3}{2}
\ssdropbull
\ssname{"B_5"}
\ssmoveto{6}{2}
\ssdropbull
\ssname{"B_8"}
\ssmoveto{7}{2}
\ssdropbull
\ssname{"B_9"}
\ssmoveto{8}{2}
\ssdropbull
\ssname{"B_{10}"}
\ssmoveto{14}{2}
\ssdropbull
\ssname{"B_{16}"}
\ssmoveto{15}{2}
\ssdropbull
\ssname{"B_{17}"}
\ssmoveto{16}{2}
\ssdropbull
\ssname{"B_{18}"}
\ssmoveto{18}{2}
\ssdropbull
\ssname{"B_{20}"}
\ssmoveto{30}{2}
\ssdropbull
\ssname{"B_{32}"}
\ssmoveto{31}{2}
\ssdropbull
\ssname{"B_{33}"}
\ssmoveto{32}{2}
\ssdropbull
\ssname{"B_{34}"}
\ssmoveto{34}{2}
\ssdropbull
\ssname{"B_{36}"}
\ssmoveto{38}{2}
\ssdropbull
\ssname{"B_{40}"}
\ssmoveto{62}{2}
\ssdropbull
\ssname{"B_{64}"}
\ssmoveto{63}{2}
\ssdropbull
\ssname{"B_{65}"}
\ssmoveto{64}{2}
\ssdropbull
\ssname{"B_{66}"}
\ssmoveto{66}{2}
\ssdropbull
\ssname{"B_{68}"}
\ssmoveto{70}{2}
\ssdropbull
\ssname{"B_{72}"}
\ssmoveto{78}{2}
\ssdropbull
\ssname{"B_{80}"}
\ssmoveto{0}{3}
\ssdropbull
\ssname{"C_3"}
\ssmoveto{3}{3}
\ssdropbull
\ssname{"C_6"}
\ssmoveto{7}{3}
\ssdropbull
\ssname{"C_{10}"}
\ssmoveto{8}{3}
\ssdropbull
\ssname{"C_{11}"}
\ssmoveto{9}{3}
\ssdropbull
\ssname{"C_{12}"}
\ssmoveto{14}{3}
\ssdropbull
\ssname{"C_{17}"}
\ssmoveto{15}{3}
\ssdropbull
\ssname{"C_{18}"}
\ssmoveto{17}{3}
\ssdropbull
\ssname{"C_{20}"}
\ssmoveto{18}{3}
\ssdropbull
\ssname{"C_{21}"}
\ssmoveto{19}{3}
\ssdropbull
\ssname{"C_{22}"}
\ssmoveto{21}{3}
\ssdropbull
\ssname{"C_{24}"}
\ssmoveto{30}{3}
\ssdropbull
\ssname{"C_{33}"}
\ssmoveto{31}{3}
\ssdropbull
\ssname{"C_{34}"}
\ssdropbull
\ssname{"C_{34}'"}
\ssmoveto{33}{3}
\ssdropbull
\ssname{"C_{36}"}
\ssmoveto{34}{3}
\ssdropbull
\ssname{"C_{37}"}
\ssmoveto{37}{3}
\ssdropbull
\ssname{"C_{40}"}
\ssmoveto{38}{3}
\ssdropbull
\ssname{"C_{41}"}
\ssmoveto{39}{3}
\ssdropbull
\ssname{"C_{42}"}
\ssmoveto{41}{3}
\ssdropbull
\ssname{"C_{44}"}
\ssmoveto{45}{3}
\ssdropbull
\ssname{"C_{48}"}
\ssmoveto{62}{3}
\ssdropbull
\ssname{"C_{65}"}
\ssmoveto{63}{3}
\ssdropbull
\ssname{"C_{66}"}
\ssdropbull
\ssname{"C_{66}'"}
\ssmoveto{65}{3}
\ssdropbull
\ssname{"C_{68}"}
\ssdropbull
\ssname{"C_{68}'"}
\ssmoveto{66}{3}
\ssdropbull
\ssname{"C_{69}"}
\ssmoveto{69}{3}
\ssdropbull
\ssname{"C_{72}"}
\ssmoveto{70}{3}
\ssdropbull
\ssname{"C_{73}"}
\ssmoveto{71}{3}
\ssdropbull
\ssname{"C_{74}"}
\ssmoveto{77}{3}
\ssdropbull
\ssname{"C_{80}"}
\ssmoveto{78}{3}
\ssdropbull
\ssname{"C_{81}"}
\ssmoveto{79}{3}
\ssdropbull
\ssname{"C_{82}"}
\ssmoveto{81}{3}
\ssdropbull
\ssname{"C_{84}"}
\ssmoveto{0}{4}
\ssdropbull
\ssname{"D_4"}
\ssmoveto{7}{4}
\ssdropbull
\ssname{"D_{11}"}
\ssmoveto{9}{4}
\ssdropbull
\ssname{"D_{13}"}
\ssmoveto{14}{4}
\ssdropbull
\ssname{"D_{18}"}
\ssmoveto{15}{4}
\ssdropbull
\ssname{"D_{19}"}
\ssmoveto{17}{4}
\ssdropbull
\ssname{"D_{21}"}
\ssmoveto{18}{4}
\ssdropbull
\ssname{"D_{22}"}
\ssdropbull
\ssname{"D_{22}'"}
\ssmoveto{20}{4}
\ssdropbull
\ssname{"D_{24}"}
\ssmoveto{22}{4}
\ssdropbull
\ssname{"D_{26}"}
\ssmoveto{23}{4}
\ssdropbull
\ssname{"D_{27}"}
\ssmoveto{30}{4}
\ssdropbull
\ssname{"D_{34}"}
\ssmoveto{31}{4}
\ssdropbull
\ssname{"D_{35}"}
\ssmoveto{32}{4}
\ssdropbull
\ssname{"D_{36}"}
\ssmoveto{33}{4}
\ssdropbull
\ssname{"D_{37}"}
\ssmoveto{34}{4}
\ssdropbull
\ssname{"D_{38}"}
\ssmoveto{38}{4}
\ssdropbull
\ssname{"D_{42}"}
\ssdropbull
\ssname{"D_{42}'"}
\ssmoveto{39}{4}
\ssdropbull
\ssname{"D_{43}"}
\ssmoveto{40}{4}
\ssdropbull
\ssname{"D_{44}"}
\ssdropbull
\ssname{"D_{44}'"}
\ssmoveto{41}{4}
\ssdropbull
\ssname{"D_{45}"}
\ssmoveto{44}{4}
\ssdropbull
\ssname{"D_{48}"}
\ssmoveto{45}{4}
\ssdropbull
\ssname{"D_{49}"}
\ssmoveto{48}{4}
\ssdropbull
\ssname{"D_{52}"}
\ssmoveto{50}{4}
\ssdropbull
\ssname{"D_{54}"}
\ssmoveto{61}{4}
\ssdropbull
\ssname{"D_{65}"}
\ssmoveto{62}{4}
\ssdropbull
\ssname{"D_{66}"}
\ssmoveto{63}{4}
\ssdropbull
\ssname{"D_{67}"}
\ssmoveto{64}{4}
\ssdropbull
\ssname{"D_{68}"}
\ssmoveto{65}{4}
\ssdropbull
\ssname{"D_{69}"}
\ssmoveto{66}{4}
\ssdropbull
\ssname{"D_{70}"}
\ssmoveto{68}{4}
\ssdropbull
\ssname{"D_{72}"}
\ssmoveto{69}{4}
\ssdropbull
\ssname{"D_{73}"}
\ssmoveto{70}{4}
\ssdropbull
\ssname{"D_{74}"}
\ssdropbull
\ssname{"D_{74}'"}
\ssmoveto{71}{4}
\ssdropbull
\ssname{"D_{75}"}
\ssmoveto{72}{4}
\ssdropbull
\ssname{"D_{76}"}
\ssmoveto{77}{4}
\ssdropbull
\ssname{"D_{81}"}
\ssmoveto{78}{4}
\ssdropbull
\ssname{"D_{82}"}
\ssmoveto{80}{4}
\ssdropbull
\ssname{"D_{84}"}
\ssdropbull
\ssname{"D_{84}'"}
\ssmoveto{81}{4}
\ssdropbull
\ssname{"D_{85}"}
\ssmoveto{82}{4}
\ssdropbull
\ssname{"D_{86}"}
\ssmoveto{0}{5}
\ssdropbull
\ssname{"E_5"}
\ssmoveto{9}{5}
\ssdropbull
\ssname{"E_{14}"}
\ssmoveto{11}{5}
\ssdropbull
\ssname{"E_{16}"}
\ssmoveto{14}{5}
\ssdropbull
\ssname{"E_{19}"}
\ssmoveto{15}{5}
\ssdropbull
\ssname{"E_{20}"}
\ssdropbull
\ssname{"E_{20}'"}
\ssmoveto{17}{5}
\ssdropbull
\ssname{"E_{22}"}
\ssmoveto{18}{5}
\ssdropbull
\ssname{"E_{23}"}
\ssmoveto{20}{5}
\ssdropbull
\ssname{"E_{25}"}
\ssmoveto{21}{5}
\ssdropbull
\ssname{"E_{26}"}
\ssmoveto{23}{5}
\ssdropbull
\ssname{"E_{28}"}
\ssmoveto{24}{5}
\ssdropbull
\ssname{"E_{29}"}
\ssmoveto{30}{5}
\ssdropbull
\ssname{"E_{35}"}
\ssmoveto{31}{5}
\ssdropbull
\ssname{"E_{36}"}
\ssdropbull
\ssname{"E_{36}'"}
\ssmoveto{33}{5}
\ssdropbull
\ssname{"E_{38}"}
\ssmoveto{35}{5}
\ssdropbull
\ssname{"E_{40}"}
\ssmoveto{37}{5}
\ssdropbull
\ssname{"E_{42}"}
\ssmoveto{38}{5}
\ssdropbull
\ssname{"E_{43}"}
\ssmoveto{39}{5}
\ssdropbull
\ssname{"E_{44}"}
\ssmoveto{40}{5}
\ssdropbull
\ssname{"E_{45}"}
\ssdropbull
\ssname{"E_{45}'"}
\ssmoveto{41}{5}
\ssdropbull
\ssname{"E_{46}"}
\ssmoveto{44}{5}
\ssdropbull
\ssname{"E_{49}"}
\ssmoveto{45}{5}
\ssdropbull
\ssname{"E_{50}"}
\ssdropbull
\ssname{"E_{50}'"}
\ssmoveto{47}{5}
\ssdropbull
\ssname{"E_{52}"}
\ssmoveto{48}{5}
\ssdropbull
\ssname{"E_{53}"}
\ssmoveto{49}{5}
\ssdropbull
\ssname{"E_{54}"}
\ssmoveto{51}{5}
\ssdropbull
\ssname{"E_{56}"}
\ssmoveto{52}{5}
\ssdropbull
\ssname{"E_{57}"}
\ssmoveto{53}{5}
\ssdropbull
\ssname{"E_{58}"}
\ssmoveto{62}{5}
\ssdropbull
\ssname{"E_{67}"}
\ssdropbull
\ssname{"E_{67}'"}
\ssdropbull
\ssname{"E_{67}''"}
\ssmoveto{63}{5}
\ssdropbull
\ssname{"E_{68}"}
\ssmoveto{64}{5}
\ssdropbull
\ssname{"E_{69}"}
\ssmoveto{65}{5}
\ssdropbull
\ssname{"E_{70}"}
\ssmoveto{67}{5}
\ssdropbull
\ssname{"E_{72}"}
\ssdropbull
\ssname{"E_{72}'"}
\ssmoveto{68}{5}
\ssdropbull
\ssname{"E_{73}"}
\ssmoveto{69}{5}
\ssdropbull
\ssname{"E_{74}"}
\ssmoveto{70}{5}
\ssdropbull
\ssname{"E_{75}"}
\ssdropbull
\ssname{"E_{75}'"}
\ssmoveto{71}{5}
\ssdropbull
\ssname{"E_{76}"}
\ssmoveto{72}{5}
\ssdropbull
\ssname{"E_{77}"}
\ssmoveto{75}{5}
\ssdropbull
\ssname{"E_{80}"}
\ssmoveto{76}{5}
\ssdropbull
\ssname{"E_{81}"}
\ssmoveto{77}{5}
\ssdropbull
\ssname{"E_{82}"}
\ssmoveto{78}{5}
\ssdropbull
\ssname{"E_{83}"}
\ssmoveto{79}{5}
\ssdropbull
\ssname{"E_{84}"}
\ssmoveto{80}{5}
\ssdropbull
\ssname{"E_{85}"}
\ssdropbull
\ssname{"E_{85}'"}
\ssmoveto{81}{5}
\ssdropbull
\ssname{"E_{86}"}
\ssdropbull
\ssname{"E_{86}'"}
\ssmoveto{0}{6}
\ssdropbull
\ssname{"F_6"}
\ssmoveto{10}{6}
\ssdropbull
\ssname{"F_{16}"}
\ssmoveto{11}{6}
\ssdropbull
\ssname{"F_{17}"}
\ssmoveto{14}{6}
\ssdropbull
\ssname{"F_{20}"}
\ssmoveto{15}{6}
\ssdropbull
\ssname{"F_{21}"}
\ssmoveto{16}{6}
\ssdropbull
\ssname{"F_{22}"}
\ssmoveto{17}{6}
\ssdropbull
\ssname{"F_{23}"}
\ssmoveto{20}{6}
\ssdropbull
\ssname{"F_{26}"}
\ssmoveto{23}{6}
\ssdropbull
\ssname{"F_{29}"}
\ssmoveto{26}{6}
\ssdropbull
\ssname{"F_{32}"}
\ssmoveto{30}{6}
\ssdropbull
\ssname{"F_{36}"}
\ssmoveto{31}{6}
\ssdropbull
\ssname{"F_{37}"}
\ssmoveto{32}{6}
\ssdropbull
\ssname{"F_{38}"}
\ssmoveto{34}{6}
\ssdropbull
\ssname{"F_{40}"}
\ssmoveto{36}{6}
\ssdropbull
\ssname{"F_{42}"}
\ssmoveto{37}{6}
\ssdropbull
\ssname{"F_{43}"}
\ssmoveto{38}{6}
\ssdropbull
\ssname{"F_{44}"}
\ssdropbull
\ssname{"F_{44}'"}
\ssmoveto{40}{6}
\ssdropbull
\ssname{"F_{46}"}
\ssdropbull
\ssname{"F_{46}'"}
\ssmoveto{42}{6}
\ssdropbull
\ssname{"F_{48}"}
\ssmoveto{44}{6}
\ssdropbull
\ssname{"F_{50}"}
\ssmoveto{45}{6}
\ssdropbull
\ssname{"F_{51}"}
\ssmoveto{46}{6}
\ssdropbull
\ssname{"F_{52}"}
\ssmoveto{47}{6}
\ssdropbull
\ssname{"F_{53}"}
\ssmoveto{48}{6}
\ssdropbull
\ssname{"F_{54}"}
\ssmoveto{49}{6}
\ssdropbull
\ssname{"F_{55}"}
\ssmoveto{50}{6}
\ssdropbull
\ssname{"F_{56}"}
\ssmoveto{51}{6}
\ssdropbull
\ssname{"F_{57}"}
\ssmoveto{52}{6}
\ssdropbull
\ssname{"F_{58}"}
\ssmoveto{54}{6}
\ssdropbull
\ssname{"F_{60}"}
\ssmoveto{58}{6}
\ssdropbull
\ssname{"F_{64}"}
\ssmoveto{61}{6}
\ssdropbull
\ssname{"F_{67}"}
\ssdropbull
\ssname{"F_{67}'"}
\ssmoveto{62}{6}
\ssdropbull
\ssname{"F_{68}"}
\ssdropbull
\ssname{"F_{68}'"}
\ssmoveto{63}{6}
\ssdropbull
\ssname{"F_{69}"}
\ssdropbull
\ssname{"F_{69}'"}
\ssmoveto{64}{6}
\ssdropbull
\ssname{"F_{70}"}
\ssmoveto{65}{6}
\ssdropbull
\ssname{"F_{71}"}
\ssmoveto{66}{6}
\ssdropbull
\ssname{"F_{72}"}
\ssmoveto{67}{6}
\ssdropbull
\ssname{"F_{73}"}
\ssdropbull
\ssname{"F_{73}'"}
\ssmoveto{68}{6}
\ssdropbull
\ssname{"F_{74}"}
\ssmoveto{69}{6}
\ssdropbull
\ssname{"F_{75}"}
\ssdropbull
\ssname{"F_{75}'"}
\ssmoveto{70}{6}
\ssdropbull
\ssname{"F_{76}"}
\ssmoveto{71}{6}
\ssdropbull
\ssname{"F_{77}"}
\ssdropbull
\ssname{"F_{77}'"}
\ssmoveto{72}{6}
\ssdropbull
\ssname{"F_{78}"}
\ssmoveto{74}{6}
\ssdropbull
\ssname{"F_{80}"}
\ssdropbull
\ssname{"F_{80}'"}
\ssmoveto{75}{6}
\ssdropbull
\ssname{"F_{81}"}
\ssmoveto{76}{6}
\ssdropbull
\ssname{"F_{82}"}
\ssmoveto{77}{6}
\ssdropbull
\ssname{"F_{83}"}
\ssdropbull
\ssname{"F_{83}'"}
\ssmoveto{78}{6}
\ssdropbull
\ssname{"F_{84}"}
\ssdropbull
\ssname{"F_{84}'"}
\ssdropbull
\ssname{"F_{84}''"}
\ssmoveto{79}{6}
\ssdropbull
\ssname{"F_{85}"}
\ssmoveto{80}{6}
\ssdropbull
\ssname{"F_{86}"}
\ssdropbull
\ssname{"F_{86}'"}
\ssmoveto{81}{6}
\ssdropbull
\ssname{"F_{87}"}
\ssmoveto{0}{7}
\ssdropbull
\ssname{"G_7"}
\ssmoveto{11}{7}
\ssdropbull
\ssname{"G_{18}"}
\ssmoveto{15}{7}
\ssdropbull
\ssname{"G_{22}"}
\ssmoveto{16}{7}
\ssdropbull
\ssname{"G_{23}"}
\ssmoveto{17}{7}
\ssdropbull
\ssname{"G_{24}"}
\ssmoveto{23}{7}
\ssdropbull
\ssname{"G_{30}"}
\ssmoveto{26}{7}
\ssdropbull
\ssname{"G_{33}"}
\ssmoveto{29}{7}
\ssdropbull
\ssname{"G_{36}"}
\ssmoveto{30}{7}
\ssdropbull
\ssname{"G_{37}"}
\ssmoveto{31}{7}
\ssdropbull
\ssname{"G_{38}"}
\ssmoveto{32}{7}
\ssdropbull
\ssname{"G_{39}"}
\ssmoveto{33}{7}
\ssdropbull
\ssname{"G_{40}"}
\ssmoveto{35}{7}
\ssdropbull
\ssname{"G_{42}"}
\ssmoveto{37}{7}
\ssdropbull
\ssname{"G_{44}"}
\ssdropbull
\ssname{"G_{44}'"}
\ssmoveto{38}{7}
\ssdropbull
\ssname{"G_{45}"}
\ssmoveto{39}{7}
\ssdropbull
\ssname{"G_{46}"}
\ssmoveto{41}{7}
\ssdropbull
\ssname{"G_{48}"}
\ssmoveto{42}{7}
\ssdropbull
\ssname{"G_{49}"}
\ssmoveto{45}{7}
\ssdropbull
\ssname{"G_{52}"}
\ssmoveto{46}{7}
\ssdropbull
\ssname{"G_{53}"}
\ssmoveto{47}{7}
\ssdropbull
\ssname{"G_{54}"}
\ssmoveto{48}{7}
\ssdropbull
\ssname{"G_{55}"}
\ssdropbull
\ssname{"G_{55}'"}
\ssmoveto{51}{7}
\ssdropbull
\ssname{"G_{58}"}
\ssmoveto{53}{7}
\ssdropbull
\ssname{"G_{60}"}
\ssmoveto{55}{7}
\ssdropbull
\ssname{"G_{62}"}
\ssmoveto{57}{7}
\ssdropbull
\ssname{"G_{64}"}
\ssmoveto{58}{7}
\ssdropbull
\ssname{"G_{65}"}
\ssmoveto{60}{7}
\ssdropbull
\ssname{"G_{67}"}
\ssmoveto{61}{7}
\ssdropbull
\ssname{"G_{68}"}
\ssdropbull
\ssname{"G_{68}'"}
\ssmoveto{62}{7}
\ssdropbull
\ssname{"G_{69}"}
\ssmoveto{63}{7}
\ssdropbull
\ssname{"G_{70}"}
\ssdropbull
\ssname{"G_{70}'"}
\ssdropbull
\ssname{"G_{70}''"}
\ssmoveto{64}{7}
\ssdropbull
\ssname{"G_{71}"}
\ssdropbull
\ssname{"G_{71}'"}
\ssmoveto{65}{7}
\ssdropbull
\ssname{"G_{72}"}
\ssdropbull
\ssname{"G_{72}'"}
\ssmoveto{66}{7}
\ssdropbull
\ssname{"G_{73}"}
\ssdropbull
\ssname{"G_{73}'"}
\ssmoveto{67}{7}
\ssdropbull
\ssname{"G_{74}"}
\ssmoveto{68}{7}
\ssdropbull
\ssname{"G_{75}"}
\ssdropbull
\ssname{"G_{75}'"}
\ssmoveto{69}{7}
\ssdropbull
\ssname{"G_{76}"}
\ssmoveto{70}{7}
\ssdropbull
\ssname{"G_{77}"}
\ssdropbull
\ssname{"G_{77}'"}
\ssmoveto{71}{7}
\ssdropbull
\ssname{"G_{78}"}
\ssmoveto{73}{7}
\ssdropbull
\ssname{"G_{80}"}
\ssdropbull
\ssname{"G_{80}'"}
\ssdropbull
\ssname{"G_{80}''"}
\ssmoveto{74}{7}
\ssdropbull
\ssname{"G_{81}"}
\ssmoveto{75}{7}
\ssdropbull
\ssname{"G_{82}"}
\ssmoveto{76}{7}
\ssdropbull
\ssname{"G_{83}"}
\ssdropbull
\ssname{"G_{83}'"}
\ssmoveto{77}{7}
\ssdropbull
\ssname{"G_{84}"}
\ssdropbull
\ssname{"G_{84}'"}
\ssdropbull
\ssname{"G_{84}''"}
\ssdropbull
\ssname{"G_{84}'''"}
\ssmoveto{78}{7}
\ssdropbull
\ssname{"G_{85}"}
\ssdropbull
\ssname{"G_{85}'"}
\ssmoveto{79}{7}
\ssdropbull
\ssname{"G_{86}"}
\ssdropbull
\ssname{"G_{86}'"}
\ssmoveto{80}{7}
\ssdropbull
\ssname{"G_{87}"}
\ssmoveto{0}{8}
\ssdropbull
\ssname{"H_8"}
\ssmoveto{15}{8}
\ssdropbull
\ssname{"H_{23}"}
\ssmoveto{17}{8}
\ssdropbull
\ssname{"H_{25}"}
\ssmoveto{22}{8}
\ssdropbull
\ssname{"H_{30}"}
\ssmoveto{23}{8}
\ssdropbull
\ssname{"H_{31}"}
\ssmoveto{25}{8}
\ssdropbull
\ssname{"H_{33}"}
\ssmoveto{26}{8}
\ssdropbull
\ssname{"H_{34}"}
\ssmoveto{28}{8}
\ssdropbull
\ssname{"H_{36}"}
\ssmoveto{29}{8}
\ssdropbull
\ssname{"H_{37}"}
\ssmoveto{30}{8}
\ssdropbull
\ssname{"H_{38}"}
\ssmoveto{31}{8}
\ssdropbull
\ssname{"H_{39}"}
\ssdropbull
\ssname{"H_{39}'"}
\ssmoveto{32}{8}
\ssdropbull
\ssname{"H_{40}"}
\ssmoveto{34}{8}
\ssdropbull
\ssname{"H_{42}"}
\ssmoveto{35}{8}
\ssdropbull
\ssname{"H_{43}"}
\ssmoveto{37}{8}
\ssdropbull
\ssname{"H_{45}"}
\ssdropbull
\ssname{"H_{45}'"}
\ssmoveto{38}{8}
\ssdropbull
\ssname{"H_{46}"}
\ssmoveto{40}{8}
\ssdropbull
\ssname{"H_{48}"}
\ssmoveto{42}{8}
\ssdropbull
\ssname{"H_{50}"}
\ssmoveto{46}{8}
\ssdropbull
\ssname{"H_{54}"}
\ssmoveto{47}{8}
\ssdropbull
\ssname{"H_{55}"}
\ssdropbull
\ssname{"H_{55}'"}
\ssmoveto{48}{8}
\ssdropbull
\ssname{"H_{56}"}
\ssmoveto{51}{8}
\ssdropbull
\ssname{"H_{59}"}
\ssmoveto{52}{8}
\ssdropbull
\ssname{"H_{60}"}
\ssmoveto{54}{8}
\ssdropbull
\ssname{"H_{62}"}
\ssmoveto{57}{8}
\ssdropbull
\ssname{"H_{65}"}
\ssdropbull
\ssname{"H_{65}'"}
\ssmoveto{58}{8}
\ssdropbull
\ssname{"H_{66}"}
\ssmoveto{60}{8}
\ssdropbull
\ssname{"H_{68}"}
\ssmoveto{61}{8}
\ssdropbull
\ssname{"H_{69}"}
\ssmoveto{62}{8}
\ssdropbull
\ssname{"H_{70}"}
\ssdropbull
\ssname{"H_{70}'"}
\ssdropbull
\ssname{"H_{70}''"}
\ssmoveto{63}{8}
\ssdropbull
\ssname{"H_{71}"}
\ssdropbull
\ssname{"H_{71}'"}
\ssmoveto{64}{8}
\ssdropbull
\ssname{"H_{72}"}
\ssdropbull
\ssname{"H_{72}'"}
\ssdropbull
\ssname{"H_{72}''"}
\ssmoveto{65}{8}
\ssdropbull
\ssname{"H_{73}"}
\ssmoveto{66}{8}
\ssdropbull
\ssname{"H_{74}"}
\ssmoveto{67}{8}
\ssdropbull
\ssname{"H_{75}"}
\ssmoveto{68}{8}
\ssdropbull
\ssname{"H_{76}"}
\ssdropbull
\ssname{"H_{76}'"}
\ssmoveto{69}{8}
\ssdropbull
\ssname{"H_{77}"}
\ssdropbull
\ssname{"H_{77}'"}
\ssmoveto{70}{8}
\ssdropbull
\ssname{"H_{78}"}
\ssmoveto{72}{8}
\ssdropbull
\ssname{"H_{80}"}
\ssdropbull
\ssname{"H_{80}'"}
\ssmoveto{73}{8}
\ssdropbull
\ssname{"H_{81}"}
\ssmoveto{74}{8}
\ssdropbull
\ssname{"H_{82}"}
\ssdropbull
\ssname{"H_{82}'"}
\ssmoveto{75}{8}
\ssdropbull
\ssname{"H_{83}"}
\ssdropbull
\ssname{"H_{83}'"}
\ssmoveto{76}{8}
\ssdropbull
\ssname{"H_{84}"}
\ssdropbull
\ssname{"H_{84}'"}
\ssmoveto{77}{8}
\ssdropbull
\ssname{"H_{85}"}
\ssdropbull
\ssname{"H_{85}'"}
\ssdropbull
\ssname{"H_{85}''"}
\ssmoveto{78}{8}
\ssdropbull
\ssname{"H_{86}"}
\ssdropbull
\ssname{"H_{86}'"}
\ssdropbull
\ssname{"H_{86}''"}
\ssmoveto{79}{8}
\ssdropbull
\ssname{"H_{87}"}
\ssdropbull
\ssname{"H_{87}'"}
\ssmoveto{0}{9}
\ssdropbull
\ssname{"I_9"}
\ssmoveto{17}{9}
\ssdropbull
\ssname{"I_{26}"}
\ssmoveto{19}{9}
\ssdropbull
\ssname{"I_{28}"}
\ssmoveto{22}{9}
\ssdropbull
\ssname{"I_{31}"}
\ssmoveto{23}{9}
\ssdropbull
\ssname{"I_{32}"}
\ssdropbull
\ssname{"I_{32}'"}
\ssmoveto{25}{9}
\ssdropbull
\ssname{"I_{34}"}
\ssmoveto{26}{9}
\ssdropbull
\ssname{"I_{35}"}
\ssmoveto{28}{9}
\ssdropbull
\ssname{"I_{37}"}
\ssmoveto{29}{9}
\ssdropbull
\ssname{"I_{38}"}
\ssmoveto{30}{9}
\ssdropbull
\ssname{"I_{39}"}
\ssmoveto{31}{9}
\ssdropbull
\ssname{"I_{40}"}
\ssdropbull
\ssname{"I_{40}'"}
\ssmoveto{32}{9}
\ssdropbull
\ssname{"I_{41}"}
\ssmoveto{34}{9}
\ssdropbull
\ssname{"I_{43}"}
\ssmoveto{35}{9}
\ssdropbull
\ssname{"I_{44}"}
\ssmoveto{37}{9}
\ssdropbull
\ssname{"I_{46}"}
\ssmoveto{38}{9}
\ssdropbull
\ssname{"I_{47}"}
\ssmoveto{39}{9}
\ssdropbull
\ssname{"I_{48}"}
\ssmoveto{42}{9}
\ssdropbull
\ssname{"I_{51}"}
\ssmoveto{45}{9}
\ssdropbull
\ssname{"I_{54}"}
\ssmoveto{48}{9}
\ssdropbull
\ssname{"I_{57}"}
\ssdropbull
\ssname{"I_{57}'"}
\ssmoveto{51}{9}
\ssdropbull
\ssname{"I_{60}"}
\ssmoveto{53}{9}
\ssdropbull
\ssname{"I_{62}"}
\ssmoveto{54}{9}
\ssdropbull
\ssname{"I_{63}"}
\ssmoveto{56}{9}
\ssdropbull
\ssname{"I_{65}"}
\ssmoveto{57}{9}
\ssdropbull
\ssname{"I_{66}"}
\ssdropbull
\ssname{"I_{66}'"}
\ssmoveto{60}{9}
\ssdropbull
\ssname{"I_{69}"}
\ssdropbull
\ssname{"I_{69}'"}
\ssmoveto{61}{9}
\ssdropbull
\ssname{"I_{70}"}
\ssmoveto{62}{9}
\ssdropbull
\ssname{"I_{71}"}
\ssmoveto{63}{9}
\ssdropbull
\ssname{"I_{72}"}
\ssdropbull
\ssname{"I_{72}'"}
\ssdropbull
\ssname{"I_{72}''"}
\ssmoveto{64}{9}
\ssdropbull
\ssname{"I_{73}"}
\ssmoveto{65}{9}
\ssdropbull
\ssname{"I_{74}"}
\ssdropbull
\ssname{"I_{74}'"}
\ssmoveto{67}{9}
\ssdropbull
\ssname{"I_{76}"}
\ssdropbull
\ssname{"I_{76}'"}
\ssmoveto{68}{9}
\ssdropbull
\ssname{"I_{77}"}
\ssmoveto{69}{9}
\ssdropbull
\ssname{"I_{78}"}
\ssdropbull
\ssname{"I_{78}'"}
\ssmoveto{70}{9}
\ssdropbull
\ssname{"I_{79}"}
\ssmoveto{71}{9}
\ssdropbull
\ssname{"I_{80}"}
\ssmoveto{72}{9}
\ssdropbull
\ssname{"I_{81}"}
\ssmoveto{73}{9}
\ssdropbull
\ssname{"I_{82}"}
\ssmoveto{74}{9}
\ssdropbull
\ssname{"I_{83}"}
\ssmoveto{75}{9}
\ssdropbull
\ssname{"I_{84}"}
\ssdropbull
\ssname{"I_{84}'"}
\ssmoveto{76}{9}
\ssdropbull
\ssname{"I_{85}"}
\ssdropbull
\ssname{"I_{85}'"}
\ssmoveto{77}{9}
\ssdropbull
\ssname{"I_{86}"}
\ssdropbull
\ssname{"I_{86}'"}
\ssmoveto{78}{9}
\ssdropbull
\ssname{"I_{87}"}
\ssdropbull
\ssname{"I_{87}'"}
\ssmoveto{0}{10}
\ssdropbull
\ssname{"J_{10}"}
\ssmoveto{18}{10}
\ssdropbull
\ssname{"J_{28}"}
\ssmoveto{19}{10}
\ssdropbull
\ssname{"J_{29}"}
\ssmoveto{22}{10}
\ssdropbull
\ssname{"J_{32}"}
\ssmoveto{23}{10}
\ssdropbull
\ssname{"J_{33}"}
\ssmoveto{24}{10}
\ssdropbull
\ssname{"J_{34}"}
\ssmoveto{25}{10}
\ssdropbull
\ssname{"J_{35}"}
\ssmoveto{28}{10}
\ssdropbull
\ssname{"J_{38}"}
\ssmoveto{30}{10}
\ssdropbull
\ssname{"J_{40}"}
\ssmoveto{31}{10}
\ssdropbull
\ssname{"J_{41}"}
\ssdropbull
\ssname{"J_{41}'"}
\ssmoveto{34}{10}
\ssdropbull
\ssname{"J_{44}"}
\ssmoveto{37}{10}
\ssdropbull
\ssname{"J_{47}"}
\ssmoveto{40}{10}
\ssdropbull
\ssname{"J_{50}"}
\ssmoveto{41}{10}
\ssdropbull
\ssname{"J_{51}"}
\ssmoveto{44}{10}
\ssdropbull
\ssname{"J_{54}"}
\ssmoveto{47}{10}
\ssdropbull
\ssname{"J_{57}"}
\ssmoveto{50}{10}
\ssdropbull
\ssname{"J_{60}"}
\ssmoveto{53}{10}
\ssdropbull
\ssname{"J_{63}"}
\ssmoveto{54}{10}
\ssdropbull
\ssname{"J_{64}"}
\ssdropbull
\ssname{"J_{64}'"}
\ssmoveto{56}{10}
\ssdropbull
\ssname{"J_{66}"}
\ssdropbull
\ssname{"J_{66}'"}
\ssmoveto{57}{10}
\ssdropbull
\ssname{"J_{67}"}
\ssmoveto{59}{10}
\ssdropbull
\ssname{"J_{69}"}
\ssmoveto{60}{10}
\ssdropbull
\ssname{"J_{70}"}
\ssmoveto{61}{10}
\ssdropbull
\ssname{"J_{71}"}
\ssmoveto{62}{10}
\ssdropbull
\ssname{"J_{72}"}
\ssdropbull
\ssname{"J_{72}'"}
\ssdropbull
\ssname{"J_{72}''"}
\ssmoveto{63}{10}
\ssdropbull
\ssname{"J_{73}"}
\ssdropbull
\ssname{"J_{73}'"}
\ssmoveto{64}{10}
\ssdropbull
\ssname{"J_{74}"}
\ssdropbull
\ssname{"J_{74}'"}
\ssmoveto{65}{10}
\ssdropbull
\ssname{"J_{75}"}
\ssmoveto{66}{10}
\ssdropbull
\ssname{"J_{76}"}
\ssdropbull
\ssname{"J_{76}'"}
\ssmoveto{67}{10}
\ssdropbull
\ssname{"J_{77}"}
\ssmoveto{68}{10}
\ssdropbull
\ssname{"J_{78}"}
\ssdropbull
\ssname{"J_{78}'"}
\ssmoveto{69}{10}
\ssdropbull
\ssname{"J_{79}"}
\ssmoveto{70}{10}
\ssdropbull
\ssname{"J_{80}"}
\ssmoveto{71}{10}
\ssdropbull
\ssname{"J_{81}"}
\ssmoveto{72}{10}
\ssdropbull
\ssname{"J_{82}"}
\ssmoveto{74}{10}
\ssdropbull
\ssname{"J_{84}"}
\ssmoveto{75}{10}
\ssdropbull
\ssname{"J_{85}"}
\ssmoveto{76}{10}
\ssdropbull
\ssname{"J_{86}"}
\ssmoveto{77}{10}
\ssdropbull
\ssname{"J_{87}"}
\ssmoveto{0}{11}
\ssdropbull
\ssname{"K_{11}"}
\ssmoveto{19}{11}
\ssdropbull
\ssname{"K_{30}"}
\ssmoveto{23}{11}
\ssdropbull
\ssname{"K_{34}"}
\ssmoveto{24}{11}
\ssdropbull
\ssname{"K_{35}"}
\ssmoveto{25}{11}
\ssdropbull
\ssname{"K_{36}"}
\ssmoveto{30}{11}
\ssdropbull
\ssname{"K_{41}"}
\ssmoveto{31}{11}
\ssdropbull
\ssname{"K_{42}"}
\ssmoveto{34}{11}
\ssdropbull
\ssname{"K_{45}"}
\ssmoveto{37}{11}
\ssdropbull
\ssname{"K_{48}"}
\ssmoveto{40}{11}
\ssdropbull
\ssname{"K_{51}"}
\ssmoveto{41}{11}
\ssdropbull
\ssname{"K_{52}"}
\ssmoveto{43}{11}
\ssdropbull
\ssname{"K_{54}"}
\ssmoveto{46}{11}
\ssdropbull
\ssname{"K_{57}"}
\ssmoveto{49}{11}
\ssdropbull
\ssname{"K_{60}"}
\ssmoveto{52}{11}
\ssdropbull
\ssname{"K_{63}"}
\ssmoveto{53}{11}
\ssdropbull
\ssname{"K_{64}"}
\ssmoveto{54}{11}
\ssdropbull
\ssname{"K_{65}"}
\ssdropbull
\ssname{"K_{65}'"}
\ssmoveto{55}{11}
\ssdropbull
\ssname{"K_{66}"}
\ssmoveto{56}{11}
\ssdropbull
\ssname{"K_{67}"}
\ssmoveto{57}{11}
\ssdropbull
\ssname{"K_{68}"}
\ssmoveto{59}{11}
\ssdropbull
\ssname{"K_{70}"}
\ssmoveto{60}{11}
\ssdropbull
\ssname{"K_{71}"}
\ssdropbull
\ssname{"K_{71}'"}
\ssmoveto{61}{11}
\ssdropbull
\ssname{"K_{72}"}
\ssdropbull
\ssname{"K_{72}'"}
\ssmoveto{62}{11}
\ssdropbull
\ssname{"K_{73}"}
\ssdropbull
\ssname{"K_{73}'"}
\ssmoveto{63}{11}
\ssdropbull
\ssname{"K_{74}"}
\ssdropbull
\ssname{"K_{74}'"}
\ssdropbull
\ssname{"K_{74}''"}
\ssmoveto{65}{11}
\ssdropbull
\ssname{"K_{76}"}
\ssdropbull
\ssname{"K_{76}'"}
\ssmoveto{66}{11}
\ssdropbull
\ssname{"K_{77}"}
\ssdropbull
\ssname{"K_{77}'"}
\ssmoveto{67}{11}
\ssdropbull
\ssname{"K_{78}"}
\ssdropbull
\ssname{"K_{78}'"}
\ssmoveto{68}{11}
\ssdropbull
\ssname{"K_{79}"}
\ssdropbull
\ssname{"K_{79}'"}
\ssmoveto{69}{11}
\ssdropbull
\ssname{"K_{80}"}
\ssdropbull
\ssname{"K_{80}'"}
\ssmoveto{71}{11}
\ssdropbull
\ssname{"K_{82}"}
\ssmoveto{72}{11}
\ssdropbull
\ssname{"K_{83}"}
\ssmoveto{74}{11}
\ssdropbull
\ssname{"K_{85}"}
\ssmoveto{75}{11}
\ssdropbull
\ssname{"K_{86}"}
\ssmoveto{0}{12}
\ssdropbull
\ssname{"L_{12}"}
\ssmoveto{23}{12}
\ssdropbull
\ssname{"L_{35}"}
\ssmoveto{25}{12}
\ssdropbull
\ssname{"L_{37}"}
\ssmoveto{30}{12}
\ssdropbull
\ssname{"L_{42}"}
\ssmoveto{31}{12}
\ssdropbull
\ssname{"L_{43}"}
\ssmoveto{33}{12}
\ssdropbull
\ssname{"L_{45}"}
\ssmoveto{34}{12}
\ssdropbull
\ssname{"L_{46}"}
\ssmoveto{36}{12}
\ssdropbull
\ssname{"L_{48}"}
\ssmoveto{37}{12}
\ssdropbull
\ssname{"L_{49}"}
\ssmoveto{39}{12}
\ssdropbull
\ssname{"L_{51}"}
\ssmoveto{40}{12}
\ssdropbull
\ssname{"L_{52}"}
\ssmoveto{42}{12}
\ssdropbull
\ssname{"L_{54}"}
\ssmoveto{43}{12}
\ssdropbull
\ssname{"L_{55}"}
\ssmoveto{45}{12}
\ssdropbull
\ssname{"L_{57}"}
\ssmoveto{48}{12}
\ssdropbull
\ssname{"L_{60}"}
\ssmoveto{51}{12}
\ssdropbull
\ssname{"L_{63}"}
\ssmoveto{53}{12}
\ssdropbull
\ssname{"L_{65}"}
\ssmoveto{54}{12}
\ssdropbull
\ssname{"L_{66}"}
\ssdropbull
\ssname{"L_{66}'"}
\ssmoveto{55}{12}
\ssdropbull
\ssname{"L_{67}"}
\ssmoveto{56}{12}
\ssdropbull
\ssname{"L_{68}"}
\ssmoveto{57}{12}
\ssdropbull
\ssname{"L_{69}"}
\ssmoveto{59}{12}
\ssdropbull
\ssname{"L_{71}"}
\ssmoveto{60}{12}
\ssdropbull
\ssname{"L_{72}"}
\ssdropbull
\ssname{"L_{72}'"}
\ssmoveto{61}{12}
\ssdropbull
\ssname{"L_{73}"}
\ssmoveto{62}{12}
\ssdropbull
\ssname{"L_{74}"}
\ssdropbull
\ssname{"L_{74}'"}
\ssmoveto{63}{12}
\ssdropbull
\ssname{"L_{75}"}
\ssmoveto{65}{12}
\ssdropbull
\ssname{"L_{77}"}
\ssdropbull
\ssname{"L_{77}'"}
\ssmoveto{66}{12}
\ssdropbull
\ssname{"L_{78}"}
\ssdropbull
\ssname{"L_{78}'"}
\ssmoveto{67}{12}
\ssdropbull
\ssname{"L_{79}"}
\ssmoveto{68}{12}
\ssdropbull
\ssname{"L_{80}"}
\ssdropbull
\ssname{"L_{80}'"}
\ssmoveto{70}{12}
\ssdropbull
\ssname{"L_{82}"}
\ssmoveto{71}{12}
\ssdropbull
\ssname{"L_{83}"}
\ssdropbull
\ssname{"L_{83}'"}
\ssmoveto{72}{12}
\ssdropbull
\ssname{"L_{84}"}
\ssmoveto{74}{12}
\ssdropbull
\ssname{"L_{86}"}
\ssmoveto{0}{13}
\ssdropbull
\ssname{"M_{13}"}
\ssmoveto{25}{13}
\ssdropbull
\ssname{"M_{38}"}
\ssmoveto{27}{13}
\ssdropbull
\ssname{"M_{40}"}
\ssmoveto{30}{13}
\ssdropbull
\ssname{"M_{43}"}
\ssmoveto{31}{13}
\ssdropbull
\ssname{"M_{44}"}
\ssdropbull
\ssname{"M_{44}'"}
\ssmoveto{33}{13}
\ssdropbull
\ssname{"M_{46}"}
\ssmoveto{34}{13}
\ssdropbull
\ssname{"M_{47}"}
\ssmoveto{36}{13}
\ssdropbull
\ssname{"M_{49}"}
\ssmoveto{37}{13}
\ssdropbull
\ssname{"M_{50}"}
\ssmoveto{39}{13}
\ssdropbull
\ssname{"M_{52}"}
\ssmoveto{40}{13}
\ssdropbull
\ssname{"M_{53}"}
\ssmoveto{42}{13}
\ssdropbull
\ssname{"M_{55}"}
\ssmoveto{43}{13}
\ssdropbull
\ssname{"M_{56}"}
\ssmoveto{47}{13}
\ssdropbull
\ssname{"M_{60}"}
\ssdropbull
\ssname{"M_{60}'"}
\ssmoveto{50}{13}
\ssdropbull
\ssname{"M_{63}"}
\ssmoveto{53}{13}
\ssdropbull
\ssname{"M_{66}"}
\ssdropbull
\ssname{"M_{66}'"}
\ssmoveto{54}{13}
\ssdropbull
\ssname{"M_{67}"}
\ssmoveto{56}{13}
\ssdropbull
\ssname{"M_{69}"}
\ssdropbull
\ssname{"M_{69}'"}
\ssmoveto{59}{13}
\ssdropbull
\ssname{"M_{72}"}
\ssmoveto{60}{13}
\ssdropbull
\ssname{"M_{73}"}
\ssmoveto{61}{13}
\ssdropbull
\ssname{"M_{74}"}
\ssmoveto{62}{13}
\ssdropbull
\ssname{"M_{75}"}
\ssdropbull
\ssname{"M_{75}'"}
\ssmoveto{63}{13}
\ssdropbull
\ssname{"M_{76}"}
\ssmoveto{65}{13}
\ssdropbull
\ssname{"M_{78}"}
\ssdropbull
\ssname{"M_{78}'"}
\ssmoveto{68}{13}
\ssdropbull
\ssname{"M_{81}"}
\ssdropbull
\ssname{"M_{81}'"}
\ssmoveto{69}{13}
\ssdropbull
\ssname{"M_{82}"}
\ssmoveto{71}{13}
\ssdropbull
\ssname{"M_{84}"}
\ssdropbull
\ssname{"M_{84}'"}
\ssdropbull
\ssname{"M_{84}''"}
\ssmoveto{72}{13}
\ssdropbull
\ssname{"M_{85}"}
\ssmoveto{74}{13}
\ssdropbull
\ssname{"M_{87}"}
\ssmoveto{0}{14}
\ssdropbull
\ssname{"N_{14}"}
\ssmoveto{26}{14}
\ssdropbull
\ssname{"N_{40}"}
\ssmoveto{27}{14}
\ssdropbull
\ssname{"N_{41}"}
\ssmoveto{30}{14}
\ssdropbull
\ssname{"N_{44}"}
\ssmoveto{31}{14}
\ssdropbull
\ssname{"N_{45}"}
\ssmoveto{32}{14}
\ssdropbull
\ssname{"N_{46}"}
\ssmoveto{33}{14}
\ssdropbull
\ssname{"N_{47}"}
\ssmoveto{36}{14}
\ssdropbull
\ssname{"N_{50}"}
\ssmoveto{39}{14}
\ssdropbull
\ssname{"N_{53}"}
\ssmoveto{42}{14}
\ssdropbull
\ssname{"N_{56}"}
\ssmoveto{46}{14}
\ssdropbull
\ssname{"N_{60}"}
\ssmoveto{47}{14}
\ssdropbull
\ssname{"N_{61}"}
\ssmoveto{48}{14}
\ssdropbull
\ssname{"N_{62}"}
\ssmoveto{49}{14}
\ssdropbull
\ssname{"N_{63}"}
\ssmoveto{52}{14}
\ssdropbull
\ssname{"N_{66}"}
\ssmoveto{53}{14}
\ssdropbull
\ssname{"N_{67}"}
\ssmoveto{54}{14}
\ssdropbull
\ssname{"N_{68}"}
\ssmoveto{55}{14}
\ssdropbull
\ssname{"N_{69}"}
\ssmoveto{58}{14}
\ssdropbull
\ssname{"N_{72}"}
\ssmoveto{60}{14}
\ssdropbull
\ssname{"N_{74}"}
\ssmoveto{61}{14}
\ssdropbull
\ssname{"N_{75}"}
\ssmoveto{62}{14}
\ssdropbull
\ssname{"N_{76}"}
\ssmoveto{63}{14}
\ssdropbull
\ssname{"N_{77}"}
\ssmoveto{64}{14}
\ssdropbull
\ssname{"N_{78}"}
\ssdropbull
\ssname{"N_{78}'"}
\ssmoveto{65}{14}
\ssdropbull
\ssname{"N_{79}"}
\ssmoveto{67}{14}
\ssdropbull
\ssname{"N_{81}"}
\ssdropbull
\ssname{"N_{81}'"}
\ssmoveto{68}{14}
\ssdropbull
\ssname{"N_{82}"}
\ssmoveto{70}{14}
\ssdropbull
\ssname{"N_{84}"}
\ssdropbull
\ssname{"N_{84}'"}
\ssdropbull
\ssname{"N_{84}''"}
\ssmoveto{71}{14}
\ssdropbull
\ssname{"N_{85}"}
\ssdropbull
\ssname{"N_{85}'"}
\ssmoveto{73}{14}
\ssdropbull
\ssname{"N_{87}"}
\ssmoveto{0}{15}
\ssdropbull
\ssname{"O_{15}"}
\ssmoveto{27}{15}
\ssdropbull
\ssname{"O_{42}"}
\ssmoveto{31}{15}
\ssdropbull
\ssname{"O_{46}"}
\ssmoveto{32}{15}
\ssdropbull
\ssname{"O_{47}"}
\ssmoveto{33}{15}
\ssdropbull
\ssname{"O_{48}"}
\ssmoveto{39}{15}
\ssdropbull
\ssname{"O_{54}"}
\ssmoveto{42}{15}
\ssdropbull
\ssname{"O_{57}"}
\ssmoveto{45}{15}
\ssdropbull
\ssname{"O_{60}"}
\ssmoveto{46}{15}
\ssdropbull
\ssname{"O_{61}"}
\ssmoveto{47}{15}
\ssdropbull
\ssname{"O_{62}"}
\ssmoveto{48}{15}
\ssdropbull
\ssname{"O_{63}"}
\ssmoveto{49}{15}
\ssdropbull
\ssname{"O_{64}"}
\ssmoveto{51}{15}
\ssdropbull
\ssname{"O_{66}"}
\ssmoveto{53}{15}
\ssdropbull
\ssname{"O_{68}"}
\ssmoveto{54}{15}
\ssdropbull
\ssname{"O_{69}"}
\ssdropbull
\ssname{"O_{69}'"}
\ssmoveto{57}{15}
\ssdropbull
\ssname{"O_{72}"}
\ssmoveto{60}{15}
\ssdropbull
\ssname{"O_{75}"}
\ssmoveto{62}{15}
\ssdropbull
\ssname{"O_{77}"}
\ssdropbull
\ssname{"O_{77}'"}
\ssmoveto{63}{15}
\ssdropbull
\ssname{"O_{78}"}
\ssdropbull
\ssname{"O_{78}'"}
\ssmoveto{64}{15}
\ssdropbull
\ssname{"O_{79}"}
\ssmoveto{65}{15}
\ssdropbull
\ssname{"O_{80}"}
\ssmoveto{66}{15}
\ssdropbull
\ssname{"O_{81}"}
\ssmoveto{67}{15}
\ssdropbull
\ssname{"O_{82}"}
\ssmoveto{68}{15}
\ssdropbull
\ssname{"O_{83}"}
\ssmoveto{69}{15}
\ssdropbull
\ssname{"O_{84}"}
\ssmoveto{70}{15}
\ssdropbull
\ssname{"O_{85}"}
\ssmoveto{71}{15}
\ssdropbull
\ssname{"O_{86}"}
\ssdropbull
\ssname{"O_{86}'"}
\ssmoveto{72}{15}
\ssdropbull
\ssname{"O_{87}"}
\ssmoveto{0}{16}
\ssdropbull
\ssname{"P_{16}"}
\ssmoveto{31}{16}
\ssdropbull
\ssname{"P_{47}"}
\ssmoveto{33}{16}
\ssdropbull
\ssname{"P_{49}"}
\ssmoveto{38}{16}
\ssdropbull
\ssname{"P_{54}"}
\ssmoveto{39}{16}
\ssdropbull
\ssname{"P_{55}"}
\ssmoveto{41}{16}
\ssdropbull
\ssname{"P_{57}"}
\ssmoveto{42}{16}
\ssdropbull
\ssname{"P_{58}"}
\ssmoveto{44}{16}
\ssdropbull
\ssname{"P_{60}"}
\ssmoveto{45}{16}
\ssdropbull
\ssname{"P_{61}"}
\ssmoveto{46}{16}
\ssdropbull
\ssname{"P_{62}"}
\ssmoveto{47}{16}
\ssdropbull
\ssname{"P_{63}"}
\ssdropbull
\ssname{"P_{63}'"}
\ssmoveto{48}{16}
\ssdropbull
\ssname{"P_{64}"}
\ssmoveto{50}{16}
\ssdropbull
\ssname{"P_{66}"}
\ssmoveto{51}{16}
\ssdropbull
\ssname{"P_{67}"}
\ssmoveto{53}{16}
\ssdropbull
\ssname{"P_{69}"}
\ssdropbull
\ssname{"P_{69}'"}
\ssmoveto{54}{16}
\ssdropbull
\ssname{"P_{70}"}
\ssmoveto{56}{16}
\ssdropbull
\ssname{"P_{72}"}
\ssmoveto{59}{16}
\ssdropbull
\ssname{"P_{75}"}
\ssmoveto{62}{16}
\ssdropbull
\ssname{"P_{78}"}
\ssdropbull
\ssname{"P_{78}'"}
\ssmoveto{63}{16}
\ssdropbull
\ssname{"P_{79}"}
\ssdropbull
\ssname{"P_{79}'"}
\ssmoveto{64}{16}
\ssdropbull
\ssname{"P_{80}"}
\ssmoveto{65}{16}
\ssdropbull
\ssname{"P_{81}"}
\ssmoveto{67}{16}
\ssdropbull
\ssname{"P_{83}"}
\ssmoveto{68}{16}
\ssdropbull
\ssname{"P_{84}"}
\ssmoveto{70}{16}
\ssdropbull
\ssname{"P_{86}"}
\ssmoveto{71}{16}
\ssdropbull
\ssname{"P_{87}"}
\ssmoveto{0}{17}
\ssdropbull
\ssname{"Q_{17}"}
\ssmoveto{33}{17}
\ssdropbull
\ssname{"Q_{50}"}
\ssmoveto{35}{17}
\ssdropbull
\ssname{"Q_{52}"}
\ssmoveto{38}{17}
\ssdropbull
\ssname{"Q_{55}"}
\ssmoveto{39}{17}
\ssdropbull
\ssname{"Q_{56}"}
\ssdropbull
\ssname{"Q_{56}'"}
\ssmoveto{41}{17}
\ssdropbull
\ssname{"Q_{58}"}
\ssmoveto{42}{17}
\ssdropbull
\ssname{"Q_{59}"}
\ssmoveto{44}{17}
\ssdropbull
\ssname{"Q_{61}"}
\ssmoveto{45}{17}
\ssdropbull
\ssname{"Q_{62}"}
\ssmoveto{46}{17}
\ssdropbull
\ssname{"Q_{63}"}
\ssmoveto{47}{17}
\ssdropbull
\ssname{"Q_{64}"}
\ssdropbull
\ssname{"Q_{64}'"}
\ssmoveto{48}{17}
\ssdropbull
\ssname{"Q_{65}"}
\ssmoveto{50}{17}
\ssdropbull
\ssname{"Q_{67}"}
\ssmoveto{51}{17}
\ssdropbull
\ssname{"Q_{68}"}
\ssmoveto{53}{17}
\ssdropbull
\ssname{"Q_{70}"}
\ssmoveto{54}{17}
\ssdropbull
\ssname{"Q_{71}"}
\ssmoveto{55}{17}
\ssdropbull
\ssname{"Q_{72}"}
\ssmoveto{58}{17}
\ssdropbull
\ssname{"Q_{75}"}
\ssmoveto{61}{17}
\ssdropbull
\ssname{"Q_{78}"}
\ssmoveto{62}{17}
\ssdropbull
\ssname{"Q_{79}"}
\ssmoveto{63}{17}
\ssdropbull
\ssname{"Q_{80}"}
\ssmoveto{64}{17}
\ssdropbull
\ssname{"Q_{81}"}
\ssdropbull
\ssname{"Q_{81}'"}
\ssmoveto{67}{17}
\ssdropbull
\ssname{"Q_{84}"}
\ssmoveto{70}{17}
\ssdropbull
\ssname{"Q_{87}"}
\ssdropbull
\ssname{"Q_{87}'"}
\ssmoveto{0}{18}
\ssdropbull
\ssname{"R_{18}"}
\ssmoveto{34}{18}
\ssdropbull
\ssname{"R_{52}"}
\ssmoveto{35}{18}
\ssdropbull
\ssname{"R_{53}"}
\ssmoveto{38}{18}
\ssdropbull
\ssname{"R_{56}"}
\ssmoveto{39}{18}
\ssdropbull
\ssname{"R_{57}"}
\ssmoveto{40}{18}
\ssdropbull
\ssname{"R_{58}"}
\ssmoveto{41}{18}
\ssdropbull
\ssname{"R_{59}"}
\ssmoveto{44}{18}
\ssdropbull
\ssname{"R_{62}"}
\ssmoveto{46}{18}
\ssdropbull
\ssname{"R_{64}"}
\ssmoveto{47}{18}
\ssdropbull
\ssname{"R_{65}"}
\ssdropbull
\ssname{"R_{65}'"}
\ssmoveto{50}{18}
\ssdropbull
\ssname{"R_{68}"}
\ssmoveto{53}{18}
\ssdropbull
\ssname{"R_{71}"}
\ssmoveto{56}{18}
\ssdropbull
\ssname{"R_{74}"}
\ssmoveto{57}{18}
\ssdropbull
\ssname{"R_{75}"}
\ssmoveto{60}{18}
\ssdropbull
\ssname{"R_{78}"}
\ssmoveto{62}{18}
\ssdropbull
\ssname{"R_{80}"}
\ssmoveto{63}{18}
\ssdropbull
\ssname{"R_{81}"}
\ssdropbull
\ssname{"R_{81}'"}
\ssmoveto{66}{18}
\ssdropbull
\ssname{"R_{84}"}
\ssmoveto{69}{18}
\ssdropbull
\ssname{"R_{87}"}
\ssdropbull
\ssname{"R_{87}'"}
\ssmoveto{0}{19}
\ssdropbull
\ssname{"S_{19}"}
\ssmoveto{35}{19}
\ssdropbull
\ssname{"S_{54}"}
\ssmoveto{39}{19}
\ssdropbull
\ssname{"S_{58}"}
\ssmoveto{40}{19}
\ssdropbull
\ssname{"S_{59}"}
\ssmoveto{41}{19}
\ssdropbull
\ssname{"S_{60}"}
\ssmoveto{46}{19}
\ssdropbull
\ssname{"S_{65}"}
\ssmoveto{47}{19}
\ssdropbull
\ssname{"S_{66}"}
\ssmoveto{50}{19}
\ssdropbull
\ssname{"S_{69}"}
\ssmoveto{53}{19}
\ssdropbull
\ssname{"S_{72}"}
\ssmoveto{56}{19}
\ssdropbull
\ssname{"S_{75}"}
\ssmoveto{57}{19}
\ssdropbull
\ssname{"S_{76}"}
\ssmoveto{59}{19}
\ssdropbull
\ssname{"S_{78}"}
\ssmoveto{62}{19}
\ssdropbull
\ssname{"S_{81}"}
\ssdropbull
\ssname{"S_{81}'"}
\ssmoveto{63}{19}
\ssdropbull
\ssname{"S_{82}"}
\ssmoveto{65}{19}
\ssdropbull
\ssname{"S_{84}"}
\ssmoveto{68}{19}
\ssdropbull
\ssname{"S_{87}"}
\ssmoveto{0}{20}
\ssdropbull
\ssname{"T_{20}"}
\ssmoveto{39}{20}
\ssdropbull
\ssname{"T_{59}"}
\ssmoveto{41}{20}
\ssdropbull
\ssname{"T_{61}"}
\ssmoveto{46}{20}
\ssdropbull
\ssname{"T_{66}"}
\ssmoveto{47}{20}
\ssdropbull
\ssname{"T_{67}"}
\ssmoveto{49}{20}
\ssdropbull
\ssname{"T_{69}"}
\ssmoveto{50}{20}
\ssdropbull
\ssname{"T_{70}"}
\ssmoveto{52}{20}
\ssdropbull
\ssname{"T_{72}"}
\ssmoveto{53}{20}
\ssdropbull
\ssname{"T_{73}"}
\ssmoveto{55}{20}
\ssdropbull
\ssname{"T_{75}"}
\ssmoveto{56}{20}
\ssdropbull
\ssname{"T_{76}"}
\ssmoveto{58}{20}
\ssdropbull
\ssname{"T_{78}"}
\ssmoveto{59}{20}
\ssdropbull
\ssname{"T_{79}"}
\ssmoveto{61}{20}
\ssdropbull
\ssname{"T_{81}"}
\ssmoveto{62}{20}
\ssdropbull
\ssname{"T_{82}"}
\ssmoveto{63}{20}
\ssdropbull
\ssname{"T_{83}"}
\ssmoveto{64}{20}
\ssdropbull
\ssname{"T_{84}"}
\ssmoveto{67}{20}
\ssdropbull
\ssname{"T_{87}"}
\ssmoveto{0}{21}
\ssdropbull
\ssname{"U_{21}"}
\ssmoveto{41}{21}
\ssdropbull
\ssname{"U_{62}"}
\ssmoveto{43}{21}
\ssdropbull
\ssname{"U_{64}"}
\ssmoveto{46}{21}
\ssdropbull
\ssname{"U_{67}"}
\ssmoveto{47}{21}
\ssdropbull
\ssname{"U_{68}"}
\ssdropbull
\ssname{"U_{68}'"}
\ssmoveto{49}{21}
\ssdropbull
\ssname{"U_{70}"}
\ssmoveto{50}{21}
\ssdropbull
\ssname{"U_{71}"}
\ssmoveto{52}{21}
\ssdropbull
\ssname{"U_{73}"}
\ssmoveto{53}{21}
\ssdropbull
\ssname{"U_{74}"}
\ssmoveto{55}{21}
\ssdropbull
\ssname{"U_{76}"}
\ssmoveto{56}{21}
\ssdropbull
\ssname{"U_{77}"}
\ssmoveto{58}{21}
\ssdropbull
\ssname{"U_{79}"}
\ssmoveto{59}{21}
\ssdropbull
\ssname{"U_{80}"}
\ssmoveto{62}{21}
\ssdropbull
\ssname{"U_{83}"}
\ssmoveto{63}{21}
\ssdropbull
\ssname{"U_{84}"}
\ssdropbull
\ssname{"U_{84}'"}
\ssmoveto{66}{21}
\ssdropbull
\ssname{"U_{87}"}
\ssmoveto{0}{22}
\ssdropbull
\ssname{"V_{22}"}
\ssmoveto{42}{22}
\ssdropbull
\ssname{"V_{64}"}
\ssmoveto{43}{22}
\ssdropbull
\ssname{"V_{65}"}
\ssmoveto{46}{22}
\ssdropbull
\ssname{"V_{68}"}
\ssmoveto{47}{22}
\ssdropbull
\ssname{"V_{69}"}
\ssmoveto{48}{22}
\ssdropbull
\ssname{"V_{70}"}
\ssmoveto{49}{22}
\ssdropbull
\ssname{"V_{71}"}
\ssmoveto{52}{22}
\ssdropbull
\ssname{"V_{74}"}
\ssmoveto{55}{22}
\ssdropbull
\ssname{"V_{77}"}
\ssmoveto{58}{22}
\ssdropbull
\ssname{"V_{80}"}
\ssmoveto{62}{22}
\ssdropbull
\ssname{"V_{84}"}
\ssmoveto{63}{22}
\ssdropbull
\ssname{"V_{85}"}
\ssmoveto{64}{22}
\ssdropbull
\ssname{"V_{86}"}
\ssmoveto{65}{22}
\ssdropbull
\ssname{"V_{87}"}
\ssmoveto{0}{23}
\ssdropbull
\ssname{"W_{23}"}
\ssmoveto{43}{23}
\ssdropbull
\ssname{"W_{66}"}
\ssmoveto{47}{23}
\ssdropbull
\ssname{"W_{70}"}
\ssmoveto{48}{23}
\ssdropbull
\ssname{"W_{71}"}
\ssmoveto{49}{23}
\ssdropbull
\ssname{"W_{72}"}
\ssmoveto{55}{23}
\ssdropbull
\ssname{"W_{78}"}
\ssmoveto{58}{23}
\ssdropbull
\ssname{"W_{81}"}
\ssmoveto{61}{23}
\ssdropbull
\ssname{"W_{84}"}
\ssmoveto{62}{23}
\ssdropbull
\ssname{"W_{85}"}
\ssmoveto{63}{23}
\ssdropbull
\ssname{"W_{86}"}
\ssmoveto{64}{23}
\ssdropbull
\ssname{"W_{87}"}
\ssmoveto{0}{24}
\ssdropbull
\ssname{"X_{24}"}
\ssmoveto{47}{24}
\ssdropbull
\ssname{"X_{71}"}
\ssmoveto{49}{24}
\ssdropbull
\ssname{"X_{73}"}
\ssmoveto{54}{24}
\ssdropbull
\ssname{"X_{78}"}
\ssmoveto{55}{24}
\ssdropbull
\ssname{"X_{79}"}
\ssmoveto{57}{24}
\ssdropbull
\ssname{"X_{81}"}
\ssmoveto{58}{24}
\ssdropbull
\ssname{"X_{82}"}
\ssmoveto{60}{24}
\ssdropbull
\ssname{"X_{84}"}
\ssmoveto{61}{24}
\ssdropbull
\ssname{"X_{85}"}
\ssmoveto{62}{24}
\ssdropbull
\ssname{"X_{86}"}
\ssmoveto{63}{24}
\ssdropbull
\ssname{"X_{87}"}
\ssdropbull
\ssname{"X_{87}'"}
\ssmoveto{0}{25}
\ssdropbull
\ssname{"Y_{25}"}
\ssmoveto{49}{25}
\ssdropbull
\ssname{"Y_{74}"}
\ssmoveto{51}{25}
\ssdropbull
\ssname{"Y_{76}"}
\ssmoveto{54}{25}
\ssdropbull
\ssname{"Y_{79}"}
\ssmoveto{55}{25}
\ssdropbull
\ssname{"Y_{80}"}
\ssdropbull
\ssname{"Y_{80}'"}
\ssmoveto{57}{25}
\ssdropbull
\ssname{"Y_{82}"}
\ssmoveto{58}{25}
\ssdropbull
\ssname{"Y_{83}"}
\ssmoveto{60}{25}
\ssdropbull
\ssname{"Y_{85}"}
\ssmoveto{61}{25}
\ssdropbull
\ssname{"Y_{86}"}
\ssmoveto{62}{25}
\ssdropbull
\ssname{"Y_{87}"}
\ssmoveto{0}{26}
\ssdropbull
\ssname{"Z_{26}"}
\ssmoveto{50}{26}
\ssdropbull
\ssname{"Z_{76}"}
\ssmoveto{51}{26}
\ssdropbull
\ssname{"Z_{77}"}
\ssmoveto{54}{26}
\ssdropbull
\ssname{"Z_{80}"}
\ssmoveto{55}{26}
\ssdropbull
\ssname{"Z_{81}"}
\ssmoveto{56}{26}
\ssdropbull
\ssname{"Z_{82}"}
\ssmoveto{57}{26}
\ssdropbull
\ssname{"Z_{83}"}
\ssmoveto{60}{26}
\ssdropbull
\ssname{"Z_{86}"}
\ssmoveto{0}{27}
\ssdropbull
\ssname{"AA_{27}"}
\ssmoveto{51}{27}
\ssdropbull
\ssname{"AA_{78}"}
\ssmoveto{55}{27}
\ssdropbull
\ssname{"AA_{82}"}
\ssmoveto{56}{27}
\ssdropbull
\ssname{"AA_{83}"}
\ssmoveto{57}{27}
\ssdropbull
\ssname{"AA_{84}"}
\ssmoveto{0}{28}
\ssdropbull
\ssname{"AB_{28}"}
\ssmoveto{55}{28}
\ssdropbull
\ssname{"AB_{83}"}
\ssmoveto{57}{28}
\ssdropbull
\ssname{"AB_{85}"}
\ssmoveto{0}{29}
\ssdropbull
\ssname{"AC_{29}"}
\ssmoveto{57}{29}
\ssdropbull
\ssname{"AC_{86}"}
\ssmoveto{0}{30}
\ssdropbull
\ssname{"AD_{30}"}
\ssmoveto{0}{31}
\ssdropbull
\ssname{"AE_{31}"}
\ssmoveto{0}{32}
\ssdropbull
\ssname{"AF_{32}"}
\ssmoveto{0}{33}
\ssdropbull
\ssname{"AG_{33}"}
\ssmoveto{0}{34}
\ssdropbull
\ssname{"AH_{34}"}
\ssmoveto{0}{35}
\ssdropbull
\ssname{"AI_{35}"}
\ssmoveto{0}{36}
\ssdropbull
\ssname{"AJ_{36}"}
\ssmoveto{0}{37}
\ssdropbull
\ssname{"AK_{37}"}
\ssmoveto{0}{38}
\ssdropbull
\ssname{"AL_{38}"}
\ssmoveto{0}{39}
\ssdropbull
\ssname{"AM_{39}"}
\ssmoveto{0}{40}
\ssdropbull
\ssname{"AN_{40}"}
\ssmoveto{0}{41}
\ssdropbull
\ssname{"AO_{41}"}
\ssmoveto{0}{42}
\ssdropbull
\ssname{"AP_{42}"}
\ssmoveto{0}{43}
\ssdropbull
\ssname{"AQ_{43}"}
\ssmoveto{0}{44}
\ssdropbull
\ssname{"AR_{44}"}
\ssmoveto{0}{45}
\ssdropbull
\ssname{"AS_{45}"}
\ssmoveto{0}{46}
\ssdropbull
\ssname{"AT_{46}"}
\ssmoveto{0}{47}
\ssdropbull
\ssname{"AU_{47}"}
\ssmoveto{0}{48}
\ssdropbull
\ssname{"AV_{48}"}
\ssmoveto{0}{49}
\ssdropbull
\ssname{"AW_{49}"}
\ssmoveto{0}{50}
\ssdropbull
\ssname{"AX_{50}"}
\ssmoveto{0}{51}
\ssdropbull
\ssname{"AY_{51}"}
\ssmoveto{0}{52}
\ssdropbull
\ssname{"AZ_{52}"}
\ssmoveto{0}{53}
\ssdropbull
\ssname{"BA_{53}"}
\ssmoveto{0}{54}
\ssdropbull
\ssname{"BB_{54}"}
\ssmoveto{0}{55}
\ssdropbull
\ssname{"BC_{55}"}
\ssmoveto{0}{56}
\ssdropbull
\ssname{"BD_{56}"}
\ssmoveto{0}{57}
\ssdropbull
\ssname{"BE_{57}"}
\ssmoveto{0}{58}
\ssdropbull
\ssname{"BF_{58}"}
\ssmoveto{0}{59}
\ssdropbull
\ssname{"BG_{59}"}
\ssmoveto{0}{60}
\ssdropbull
\ssname{"BH_{60}"}
\ssmoveto{0}{61}
\ssdropbull
\ssname{"BI_{61}"}
\ssmoveto{0}{62}
\ssdropbull
\ssname{"BJ_{62}"}
\ssmoveto{0}{63}
\ssdropbull
\ssname{"BK_{63}"}
\ssmoveto{0}{64}
\ssdropbull
\ssname{"BL_{64}"}
\ssmoveto{0}{65}
\ssdropbull
\ssname{"BM_{65}"}
\ssmoveto{0}{66}
\ssdropbull
\ssname{"BN_{66}"}
\ssmoveto{0}{67}
\ssdropbull
\ssname{"BO_{67}"}
\ssmoveto{0}{68}
\ssdropbull
\ssname{"BP_{68}"}
\ssmoveto{0}{69}
\ssdropbull
\ssname{"BQ_{69}"}
\ssmoveto{0}{70}
\ssdropbull
\ssname{"BR_{70}"}
\ssmoveto{0}{71}
\ssdropbull
\ssname{"BS_{71}"}
\ssmoveto{0}{72}
\ssdropbull
\ssname{"BT_{72}"}
\ssmoveto{0}{73}
\ssdropbull
\ssname{"BU_{73}"}
\ssmoveto{0}{74}
\ssdropbull
\ssname{"BV_{74}"}
\ssmoveto{0}{75}
\ssdropbull
\ssname{"BW_{75}"}
\ssmoveto{0}{76}
\ssdropbull
\ssname{"BX_{76}"}
\ssmoveto{0}{77}
\ssdropbull
\ssname{"BY_{77}"}
\ssmoveto{0}{78}
\ssdropbull
\ssname{"BZ_{78}"}
\ssmoveto{0}{79}
\ssdropbull
\ssname{"CA_{79}"}
\ssmoveto{0}{80}
\ssdropbull
\ssname{"CB_{80}"}
\ssmoveto{0}{81}
\ssdropbull
\ssname{"CC_{81}"}
\ssmoveto{0}{82}
\ssdropbull
\ssname{"CD_{82}"}
\ssmoveto{0}{83}
\ssdropbull
\ssname{"CE_{83}"}
\ssmoveto{0}{84}
\ssdropbull
\ssname{"CF_{84}"}
\ssmoveto{0}{85}
\ssdropbull
\ssname{"CG_{85}"}
\ssmoveto{0}{86}
\ssdropbull
\ssname{"CH_{86}"}
\ssmoveto{0}{87}
\ssdropbull
\ssname{"CI_{87}"}
\ssgoto{"A_1"}
\ssgoto{"B_2"}
\ssstroke
\ssgoto{"A_4"}
\ssgoto{"B_5"}
\ssstroke
\ssgoto{"A_8"}
\ssgoto{"B_9"}
\ssstroke
\ssgoto{"A_{16}"}
\ssgoto{"B_{17}"}
\ssstroke
\ssgoto{"A_{32}"}
\ssgoto{"B_{33}"}
\ssstroke
\ssgoto{"A_{64}"}
\ssgoto{"B_{65}"}
\ssstroke
\ssgoto{"B_2"}
\ssgoto{"C_3"}
\ssstroke
\ssgoto{"B_5"}
\ssgoto{"C_6"}
\ssstroke
\ssgoto{"B_9"}
\ssgoto{"C_{10}"}
\ssstroke
\ssgoto{"B_{16}"}
\ssgoto{"C_{17}"}
\ssstroke
\ssgoto{"B_{17}"}
\ssgoto{"C_{18}"}
\ssstroke
\ssgoto{"B_{20}"}
\ssgoto{"C_{21}"}
\ssstroke
\ssgoto{"B_{32}"}
\ssgoto{"C_{33}"}
\ssstroke
\ssgoto{"B_{33}"}
\ssgoto{"C_{34}"}
\ssstroke
\ssgoto{"B_{36}"}
\ssgoto{"C_{37}"}
\ssstroke
\ssgoto{"B_{40}"}
\ssgoto{"C_{41}"}
\ssstroke
\ssgoto{"B_{64}"}
\ssgoto{"C_{65}"}
\ssstroke
\ssgoto{"B_{65}"}
\ssgoto{"C_{66}"}
\ssstroke
\ssgoto{"B_{68}"}
\ssgoto{"C_{69}"}
\ssstroke
\ssgoto{"B_{72}"}
\ssgoto{"C_{73}"}
\ssstroke
\ssgoto{"B_{80}"}
\ssgoto{"C_{81}"}
\ssstroke
\ssgoto{"C_3"}
\ssgoto{"D_4"}
\ssstroke
\ssgoto{"C_{10}"}
\ssgoto{"D_{11}"}
\ssstroke
\ssgoto{"C_{18}"}
\ssgoto{"D_{19}"}
\ssstroke
\ssgoto{"C_{21}"}
\ssgoto{"D_{22}"}
\ssstroke
\ssgoto{"C_{33}"}
\ssgoto{"D_{34}"}
\ssstroke
\ssgoto{"C_{34}"}
\ssgoto{"D_{35}"}
\ssstroke
\ssgoto{"C_{37}"}
\ssgoto{"D_{38}"}
\ssstroke
\ssgoto{"C_{41}"}
\ssgoto{"D_{42}"}
\ssstroke
\ssgoto{"C_{44}"}
\ssgoto{"D_{45}"}
\ssstroke
\ssgoto{"C_{48}"}
\ssgoto{"D_{49}"}
\ssstroke
\ssgoto{"C_{65}"}
\ssgoto{"D_{66}"}
\ssstroke
\ssgoto{"C_{66}"}
\ssgoto{"D_{67}"}
\ssstroke
\ssgoto{"C_{68}'"}
\ssgoto{"D_{69}"}
\ssstroke
\ssgoto{"C_{69}"}
\ssgoto{"D_{70}"}
\ssstroke
\ssgoto{"C_{73}"}
\ssgoto{"D_{74}"}
\ssstroke
\ssgoto{"C_{80}"}
\ssgoto{"D_{81}"}
\ssstroke
\ssgoto{"C_{81}"}
\ssgoto{"D_{82}"}
\ssstroke
\ssgoto{"C_{84}"}
\ssgoto{"D_{85}"}
\ssstroke
\ssgoto{"D_4"}
\ssgoto{"E_5"}
\ssstroke
\ssgoto{"D_{18}"}
\ssgoto{"E_{19}"}
\ssstroke
\ssgoto{"D_{19}"}
\ssgoto{"E_{20}"}
\ssstroke
\ssgoto{"D_{21}"}
\ssgoto{"E_{22}"}
\ssstroke
\ssgoto{"D_{22}'"}
\ssgoto{"E_{23}"}
\ssstroke
\ssgoto{"D_{24}"}
\ssgoto{"E_{25}"}
\ssstroke
\ssgoto{"D_{34}"}
\ssgoto{"E_{35}"}
\ssstroke
\ssgoto{"D_{35}"}
\ssgoto{"E_{36}"}
\ssstroke
\ssgoto{"D_{37}"}
\ssgoto{"E_{38}"}
\ssstroke
\ssgoto{"D_{42}"}
\ssgoto{"E_{43}"}
\ssstroke
\ssgoto{"D_{44}'"}
\ssgoto{"E_{45}"}
\ssstroke
\ssgoto{"D_{45}"}
\ssgoto{"E_{46}"}
\ssstroke
\ssgoto{"D_{48}"}
\ssgoto{"E_{49}"}
\ssstroke
\ssgoto{"D_{66}"}
\ssgoto{"E_{67}"}
\ssstroke
\ssgoto{"D_{67}"}
\ssgoto{"E_{68}"}
\ssstroke
\ssgoto{"D_{69}"}
\ssgoto{"E_{70}"}
\ssstroke
\ssgoto{"D_{72}"}
\ssgoto{"E_{73}"}
\ssstroke
\ssgoto{"D_{73}"}
\ssgoto{"E_{74}"}
\ssstroke
\ssgoto{"D_{74}"}
\ssgoto{"E_{75}"}
\ssstroke
\ssgoto{"D_{82}"}
\ssgoto{"E_{83}"}
\ssstroke
\ssgoto{"D_{84}'"}
\ssgoto{"E_{85}"}
\ssstroke
\ssgoto{"D_{85}"}
\ssgoto{"E_{86}"}
\ssstroke
\ssgoto{"E_5"}
\ssgoto{"F_6"}
\ssstroke
\ssgoto{"E_{16}"}
\ssgoto{"F_{17}"}
\ssstroke
\ssgoto{"E_{19}"}
\ssgoto{"F_{20}"}
\ssstroke
\ssgoto{"E_{20}"}
\ssgoto{"F_{21}"}
\ssstroke
\ssgoto{"E_{22}"}
\ssgoto{"F_{23}"}
\ssstroke
\ssgoto{"E_{25}"}
\ssgoto{"F_{26}"}
\ssstroke
\ssgoto{"E_{28}"}
\ssgoto{"F_{29}"}
\ssstroke
\ssgoto{"E_{36}"}
\ssgoto{"F_{37}"}
\ssstroke
\ssgoto{"E_{42}"}
\ssgoto{"F_{43}"}
\ssstroke
\ssgoto{"E_{45}"}
\ssgoto{"F_{46}"}
\ssstroke
\ssgoto{"E_{49}"}
\ssgoto{"F_{50}"}
\ssstroke
\ssgoto{"E_{50}'"}
\ssgoto{"F_{51}"}
\ssstroke
\ssgoto{"E_{52}"}
\ssgoto{"F_{53}"}
\ssstroke
\ssgoto{"E_{53}"}
\ssgoto{"F_{54}"}
\ssstroke
\ssgoto{"E_{54}"}
\ssgoto{"F_{55}"}
\ssstroke
\ssgoto{"E_{56}"}
\ssgoto{"F_{57}"}
\ssstroke
\ssgoto{"E_{67}"}
\ssgoto{"F_{68}"}
\ssstroke
\ssgoto{"E_{68}"}
\ssgoto{"F_{69}"}
\ssstroke
\ssgoto{"E_{72}"}
\ssgoto{"F_{73}'"}
\ssstroke
\ssgoto{"E_{72}'"}
\ssgoto{"F_{73}"}
\ssstroke
\ssgoto{"E_{74}"}
\ssgoto{"F_{75}"}
\ssstroke
\ssgoto{"E_{80}"}
\ssgoto{"F_{81}"}
\ssstroke
\ssgoto{"E_{82}"}
\ssgoto{"F_{83}"}
\ssstroke
\ssgoto{"E_{83}"}
\ssgoto{"F_{84}"}
\ssstroke
\ssgoto{"E_{84}"}
\ssgoto{"F_{85}"}
\ssstroke
\ssgoto{"E_{85}"}
\ssgoto{"F_{86}'"}
\ssstroke
\ssgoto{"E_{85}'"}
\ssgoto{"F_{86}"}
\ssstroke
\ssgoto{"E_{86}'"}
\ssgoto{"F_{87}"}
\ssstroke
\ssgoto{"F_6"}
\ssgoto{"G_7"}
\ssstroke
\ssgoto{"F_{17}"}
\ssgoto{"G_{18}"}
\ssstroke
\ssgoto{"F_{21}"}
\ssgoto{"G_{22}"}
\ssstroke
\ssgoto{"F_{23}"}
\ssgoto{"G_{24}"}
\ssstroke
\ssgoto{"F_{36}"}
\ssgoto{"G_{37}"}
\ssstroke
\ssgoto{"F_{37}"}
\ssgoto{"G_{38}"}
\ssstroke
\ssgoto{"F_{43}"}
\ssgoto{"G_{44}"}
\ssstroke
\ssgoto{"F_{44}'"}
\ssgoto{"G_{45}"}
\ssstroke
\ssgoto{"F_{48}"}
\ssgoto{"G_{49}"}
\ssstroke
\ssgoto{"F_{51}"}
\ssgoto{"G_{52}"}
\ssstroke
\ssgoto{"F_{54}"}
\ssgoto{"G_{55}"}
\ssstroke
\ssgoto{"F_{57}"}
\ssgoto{"G_{58}"}
\ssstroke
\ssgoto{"F_{64}"}
\ssgoto{"G_{65}"}
\ssstroke
\ssgoto{"F_{67}"}
\ssgoto{"G_{68}'"}
\ssstroke
\ssgoto{"F_{67}'"}
\ssgoto{"G_{68}"}
\ssstroke
\ssgoto{"F_{67}'"}
\ssgoto{"G_{68}'"}
\ssstroke
\ssgoto{"F_{68}"}
\ssgoto{"G_{69}"}
\ssstroke
\ssgoto{"F_{69}"}
\ssgoto{"G_{70}"}
\ssstroke
\ssgoto{"F_{70}"}
\ssgoto{"G_{71}"}
\ssstroke
\ssgoto{"F_{72}"}
\ssgoto{"G_{73}"}
\ssstroke
\ssgoto{"F_{73}'"}
\ssgoto{"G_{74}"}
\ssstroke
\ssgoto{"F_{75}"}
\ssgoto{"G_{76}"}
\ssstroke
\ssgoto{"F_{76}"}
\ssgoto{"G_{77}"}
\ssstroke
\ssgoto{"F_{77}'"}
\ssgoto{"G_{78}"}
\ssstroke
\ssgoto{"F_{80}"}
\ssgoto{"G_{81}"}
\ssstroke
\ssgoto{"F_{82}"}
\ssgoto{"G_{83}"}
\ssstroke
\ssgoto{"F_{83}"}
\ssgoto{"G_{84}"}
\ssstroke
\ssgoto{"F_{84}"}
\ssgoto{"G_{85}'"}
\ssstroke
\ssgoto{"F_{84}''"}
\ssgoto{"G_{85}"}
\ssstroke
\ssgoto{"F_{86}"}
\ssgoto{"G_{87}"}
\ssstroke
\ssgoto{"G_7"}
\ssgoto{"H_8"}
\ssstroke
\ssgoto{"G_{22}"}
\ssgoto{"H_{23}"}
\ssstroke
\ssgoto{"G_{30}"}
\ssgoto{"H_{31}"}
\ssstroke
\ssgoto{"G_{33}"}
\ssgoto{"H_{34}"}
\ssstroke
\ssgoto{"G_{36}"}
\ssgoto{"H_{37}"}
\ssstroke
\ssgoto{"G_{37}"}
\ssgoto{"H_{38}"}
\ssstroke
\ssgoto{"G_{38}"}
\ssgoto{"H_{39}"}
\ssstroke
\ssgoto{"G_{39}"}
\ssgoto{"H_{40}"}
\ssstroke
\ssgoto{"G_{42}"}
\ssgoto{"H_{43}"}
\ssstroke
\ssgoto{"G_{44}"}
\ssgoto{"H_{45}"}
\ssstroke
\ssgoto{"G_{45}"}
\ssgoto{"H_{46}"}
\ssstroke
\ssgoto{"G_{49}"}
\ssgoto{"H_{50}"}
\ssstroke
\ssgoto{"G_{55}'"}
\ssgoto{"H_{56}"}
\ssstroke
\ssgoto{"G_{64}"}
\ssgoto{"H_{65}"}
\ssstroke
\ssgoto{"G_{65}"}
\ssgoto{"H_{66}"}
\ssstroke
\ssgoto{"G_{67}"}
\ssgoto{"H_{68}"}
\ssstroke
\ssgoto{"G_{68}'"}
\ssgoto{"H_{69}"}
\ssstroke
\ssgoto{"G_{69}"}
\ssgoto{"H_{70}"}
\ssstroke
\ssgoto{"G_{70}"}
\ssgoto{"H_{71}'"}
\ssstroke
\ssgoto{"G_{70}'"}
\ssgoto{"H_{71}"}
\ssstroke
\ssgoto{"G_{70}''"}
\ssgoto{"H_{71}"}
\ssstroke
\ssgoto{"G_{71}"}
\ssgoto{"H_{72}'"}
\ssstroke
\ssgoto{"G_{71}'"}
\ssgoto{"H_{72}"}
\ssstroke
\ssgoto{"G_{71}'"}
\ssgoto{"H_{72}'"}
\ssstroke
\ssgoto{"G_{72}"}
\ssgoto{"H_{73}"}
\ssstroke
\ssgoto{"G_{73}'"}
\ssgoto{"H_{74}"}
\ssstroke
\ssgoto{"G_{75}"}
\ssgoto{"H_{76}"}
\ssstroke
\ssgoto{"G_{80}'"}
\ssgoto{"H_{81}"}
\ssstroke
\ssgoto{"G_{81}"}
\ssgoto{"H_{82}"}
\ssstroke
\ssgoto{"G_{82}"}
\ssgoto{"H_{83}"}
\ssstroke
\ssgoto{"G_{83}"}
\ssgoto{"H_{84}"}
\ssstroke
\ssgoto{"G_{84}'"}
\ssgoto{"H_{85}'"}
\ssstroke
\ssgoto{"G_{84}''"}
\ssgoto{"H_{85}'"}
\ssstroke
\ssgoto{"G_{84}'''"}
\ssgoto{"H_{85}"}
\ssstroke
\ssgoto{"G_{85}"}
\ssgoto{"H_{86}'"}
\ssstroke
\ssgoto{"G_{85}'"}
\ssgoto{"H_{86}"}
\ssstroke
\ssgoto{"G_{86}'"}
\ssgoto{"H_{87}"}
\ssstroke
\ssgoto{"H_8"}
\ssgoto{"I_9"}
\ssstroke
\ssgoto{"H_{30}"}
\ssgoto{"I_{31}"}
\ssstroke
\ssgoto{"H_{31}"}
\ssgoto{"I_{32}"}
\ssstroke
\ssgoto{"H_{33}"}
\ssgoto{"I_{34}"}
\ssstroke
\ssgoto{"H_{34}"}
\ssgoto{"I_{35}"}
\ssstroke
\ssgoto{"H_{36}"}
\ssgoto{"I_{37}"}
\ssstroke
\ssgoto{"H_{37}"}
\ssgoto{"I_{38}"}
\ssstroke
\ssgoto{"H_{38}"}
\ssgoto{"I_{39}"}
\ssstroke
\ssgoto{"H_{39}"}
\ssgoto{"I_{40}'"}
\ssstroke
\ssgoto{"H_{39}'"}
\ssgoto{"I_{40}"}
\ssstroke
\ssgoto{"H_{40}"}
\ssgoto{"I_{41}"}
\ssstroke
\ssgoto{"H_{42}"}
\ssgoto{"I_{43}"}
\ssstroke
\ssgoto{"H_{43}"}
\ssgoto{"I_{44}"}
\ssstroke
\ssgoto{"H_{45}"}
\ssgoto{"I_{46}"}
\ssstroke
\ssgoto{"H_{45}'"}
\ssgoto{"I_{46}"}
\ssstroke
\ssgoto{"H_{46}"}
\ssgoto{"I_{47}"}
\ssstroke
\ssgoto{"H_{56}"}
\ssgoto{"I_{57}"}
\ssstroke
\ssgoto{"H_{62}"}
\ssgoto{"I_{63}"}
\ssstroke
\ssgoto{"H_{65}"}
\ssgoto{"I_{66}'"}
\ssstroke
\ssgoto{"H_{65}'"}
\ssgoto{"I_{66}"}
\ssstroke
\ssgoto{"H_{65}'"}
\ssgoto{"I_{66}'"}
\ssstroke
\ssgoto{"H_{68}"}
\ssgoto{"I_{69}"}
\ssstroke
\ssgoto{"H_{70}"}
\ssgoto{"I_{71}"}
\ssstroke
\ssgoto{"H_{70}'"}
\ssgoto{"I_{71}"}
\ssstroke
\ssgoto{"H_{70}''"}
\ssgoto{"I_{71}"}
\ssstroke
\ssgoto{"H_{71}"}
\ssgoto{"I_{72}'"}
\ssstroke
\ssgoto{"H_{71}'"}
\ssgoto{"I_{72}"}
\ssstroke
\ssgoto{"H_{72}''"}
\ssgoto{"I_{73}"}
\ssstroke
\ssgoto{"H_{73}"}
\ssgoto{"I_{74}"}
\ssstroke
\ssgoto{"H_{76}'"}
\ssgoto{"I_{77}"}
\ssstroke
\ssgoto{"H_{77}"}
\ssgoto{"I_{78}"}
\ssstroke
\ssgoto{"H_{77}'"}
\ssgoto{"I_{78}"}
\ssstroke
\ssgoto{"H_{80}"}
\ssgoto{"I_{81}"}
\ssstroke
\ssgoto{"H_{81}"}
\ssgoto{"I_{82}"}
\ssstroke
\ssgoto{"H_{82}'"}
\ssgoto{"I_{83}"}
\ssstroke
\ssgoto{"H_{83}"}
\ssgoto{"I_{84}"}
\ssstroke
\ssgoto{"H_{83}"}
\ssgoto{"I_{84}'"}
\ssstroke
\ssgoto{"H_{84}"}
\ssgoto{"I_{85}"}
\ssstroke
\ssgoto{"H_{85}"}
\ssgoto{"I_{86}'"}
\ssstroke
\ssgoto{"H_{85}'"}
\ssgoto{"I_{86}"}
\ssstroke
\ssgoto{"H_{86}"}
\ssgoto{"I_{87}"}
\ssstroke
\ssgoto{"I_9"}
\ssgoto{"J_{10}"}
\ssstroke
\ssgoto{"I_{28}"}
\ssgoto{"J_{29}"}
\ssstroke
\ssgoto{"I_{31}"}
\ssgoto{"J_{32}"}
\ssstroke
\ssgoto{"I_{32}"}
\ssgoto{"J_{33}"}
\ssstroke
\ssgoto{"I_{34}"}
\ssgoto{"J_{35}"}
\ssstroke
\ssgoto{"I_{37}"}
\ssgoto{"J_{38}"}
\ssstroke
\ssgoto{"I_{39}"}
\ssgoto{"J_{40}"}
\ssstroke
\ssgoto{"I_{40}"}
\ssgoto{"J_{41}'"}
\ssstroke
\ssgoto{"I_{40}'"}
\ssgoto{"J_{41}"}
\ssstroke
\ssgoto{"I_{43}"}
\ssgoto{"J_{44}"}
\ssstroke
\ssgoto{"I_{46}"}
\ssgoto{"J_{47}"}
\ssstroke
\ssgoto{"I_{63}"}
\ssgoto{"J_{64}"}
\ssstroke
\ssgoto{"I_{66}"}
\ssgoto{"J_{67}"}
\ssstroke
\ssgoto{"I_{69}'"}
\ssgoto{"J_{70}"}
\ssstroke
\ssgoto{"I_{70}"}
\ssgoto{"J_{71}"}
\ssstroke
\ssgoto{"I_{72}"}
\ssgoto{"J_{73}"}
\ssstroke
\ssgoto{"I_{73}"}
\ssgoto{"J_{74}"}
\ssstroke
\ssgoto{"I_{76}"}
\ssgoto{"J_{77}"}
\ssstroke
\ssgoto{"I_{77}"}
\ssgoto{"J_{78}"}
\ssstroke
\ssgoto{"I_{80}"}
\ssgoto{"J_{81}"}
\ssstroke
\ssgoto{"I_{83}"}
\ssgoto{"J_{84}"}
\ssstroke
\ssgoto{"I_{84}"}
\ssgoto{"J_{85}"}
\ssstroke
\ssgoto{"I_{84}'"}
\ssgoto{"J_{85}"}
\ssstroke
\ssgoto{"I_{85}'"}
\ssgoto{"J_{86}"}
\ssstroke
\ssgoto{"I_{86}"}
\ssgoto{"J_{87}"}
\ssstroke
\ssgoto{"J_{10}"}
\ssgoto{"K_{11}"}
\ssstroke
\ssgoto{"J_{29}"}
\ssgoto{"K_{30}"}
\ssstroke
\ssgoto{"J_{33}"}
\ssgoto{"K_{34}"}
\ssstroke
\ssgoto{"J_{35}"}
\ssgoto{"K_{36}"}
\ssstroke
\ssgoto{"J_{40}"}
\ssgoto{"K_{41}"}
\ssstroke
\ssgoto{"J_{41}"}
\ssgoto{"K_{42}"}
\ssstroke
\ssgoto{"J_{51}"}
\ssgoto{"K_{52}"}
\ssstroke
\ssgoto{"J_{63}"}
\ssgoto{"K_{64}"}
\ssstroke
\ssgoto{"J_{64}'"}
\ssgoto{"K_{65}"}
\ssstroke
\ssgoto{"J_{66}"}
\ssgoto{"K_{67}"}
\ssstroke
\ssgoto{"J_{69}"}
\ssgoto{"K_{70}"}
\ssstroke
\ssgoto{"J_{70}"}
\ssgoto{"K_{71}"}
\ssstroke
\ssgoto{"J_{71}"}
\ssgoto{"K_{72}"}
\ssstroke
\ssgoto{"J_{72}'"}
\ssgoto{"K_{73}'"}
\ssstroke
\ssgoto{"J_{72}''"}
\ssgoto{"K_{73}"}
\ssstroke
\ssgoto{"J_{72}''"}
\ssgoto{"K_{73}'"}
\ssstroke
\ssgoto{"J_{73}"}
\ssgoto{"K_{74}'"}
\ssstroke
\ssgoto{"J_{73}'"}
\ssgoto{"K_{74}"}
\ssstroke
\ssgoto{"J_{75}"}
\ssgoto{"K_{76}"}
\ssstroke
\ssgoto{"J_{76}"}
\ssgoto{"K_{77}'"}
\ssstroke
\ssgoto{"J_{76}'"}
\ssgoto{"K_{77}"}
\ssstroke
\ssgoto{"J_{76}'"}
\ssgoto{"K_{77}'"}
\ssstroke
\ssgoto{"J_{77}"}
\ssgoto{"K_{78}"}
\ssstroke
\ssgoto{"J_{78}"}
\ssgoto{"K_{79}"}
\ssstroke
\ssgoto{"J_{79}"}
\ssgoto{"K_{80}"}
\ssstroke
\ssgoto{"J_{82}"}
\ssgoto{"K_{83}"}
\ssstroke
\ssgoto{"J_{84}"}
\ssgoto{"K_{85}"}
\ssstroke
\ssgoto{"J_{85}"}
\ssgoto{"K_{86}"}
\ssstroke
\ssgoto{"K_{11}"}
\ssgoto{"L_{12}"}
\ssstroke
\ssgoto{"K_{34}"}
\ssgoto{"L_{35}"}
\ssstroke
\ssgoto{"K_{42}"}
\ssgoto{"L_{43}"}
\ssstroke
\ssgoto{"K_{45}"}
\ssgoto{"L_{46}"}
\ssstroke
\ssgoto{"K_{48}"}
\ssgoto{"L_{49}"}
\ssstroke
\ssgoto{"K_{51}"}
\ssgoto{"L_{52}"}
\ssstroke
\ssgoto{"K_{54}"}
\ssgoto{"L_{55}"}
\ssstroke
\ssgoto{"K_{64}"}
\ssgoto{"L_{65}"}
\ssstroke
\ssgoto{"K_{65}"}
\ssgoto{"L_{66}"}
\ssstroke
\ssgoto{"K_{67}"}
\ssgoto{"L_{68}"}
\ssstroke
\ssgoto{"K_{70}"}
\ssgoto{"L_{71}"}
\ssstroke
\ssgoto{"K_{71}"}
\ssgoto{"L_{72}"}
\ssstroke
\ssgoto{"K_{72}"}
\ssgoto{"L_{73}"}
\ssstroke
\ssgoto{"K_{73}"}
\ssgoto{"L_{74}'"}
\ssstroke
\ssgoto{"K_{73}'"}
\ssgoto{"L_{74}"}
\ssstroke
\ssgoto{"K_{73}'"}
\ssgoto{"L_{74}'"}
\ssstroke
\ssgoto{"K_{74}'"}
\ssgoto{"L_{75}"}
\ssstroke
\ssgoto{"K_{76}"}
\ssgoto{"L_{77}"}
\ssstroke
\ssgoto{"K_{77}'"}
\ssgoto{"L_{78}"}
\ssstroke
\ssgoto{"K_{77}'"}
\ssgoto{"L_{78}'"}
\ssstroke
\ssgoto{"K_{78}"}
\ssgoto{"L_{79}"}
\ssstroke
\ssgoto{"K_{78}'"}
\ssgoto{"L_{79}"}
\ssstroke
\ssgoto{"K_{79}"}
\ssgoto{"L_{80}'"}
\ssstroke
\ssgoto{"K_{79}'"}
\ssgoto{"L_{80}"}
\ssstroke
\ssgoto{"K_{79}'"}
\ssgoto{"L_{80}'"}
\ssstroke
\ssgoto{"K_{82}"}
\ssgoto{"L_{83}"}
\ssstroke
\ssgoto{"K_{83}"}
\ssgoto{"L_{84}"}
\ssstroke
\ssgoto{"K_{85}"}
\ssgoto{"L_{86}"}
\ssstroke
\ssgoto{"L_{12}"}
\ssgoto{"M_{13}"}
\ssstroke
\ssgoto{"L_{42}"}
\ssgoto{"M_{43}"}
\ssstroke
\ssgoto{"L_{43}"}
\ssgoto{"M_{44}"}
\ssstroke
\ssgoto{"L_{45}"}
\ssgoto{"M_{46}"}
\ssstroke
\ssgoto{"L_{46}"}
\ssgoto{"M_{47}"}
\ssstroke
\ssgoto{"L_{48}"}
\ssgoto{"M_{49}"}
\ssstroke
\ssgoto{"L_{49}"}
\ssgoto{"M_{50}"}
\ssstroke
\ssgoto{"L_{51}"}
\ssgoto{"M_{52}"}
\ssstroke
\ssgoto{"L_{52}"}
\ssgoto{"M_{53}"}
\ssstroke
\ssgoto{"L_{54}"}
\ssgoto{"M_{55}"}
\ssstroke
\ssgoto{"L_{55}"}
\ssgoto{"M_{56}"}
\ssstroke
\ssgoto{"L_{65}"}
\ssgoto{"M_{66}"}
\ssstroke
\ssgoto{"L_{66}"}
\ssgoto{"M_{67}"}
\ssstroke
\ssgoto{"L_{68}"}
\ssgoto{"M_{69}"}
\ssstroke
\ssgoto{"L_{72}"}
\ssgoto{"M_{73}"}
\ssstroke
\ssgoto{"L_{73}"}
\ssgoto{"M_{74}"}
\ssstroke
\ssgoto{"L_{74}"}
\ssgoto{"M_{75}"}
\ssstroke
\ssgoto{"L_{75}"}
\ssgoto{"M_{76}"}
\ssstroke
\ssgoto{"L_{80}"}
\ssgoto{"M_{81}"}
\ssstroke
\ssgoto{"L_{80}'"}
\ssgoto{"M_{81}"}
\ssstroke
\ssgoto{"M_{13}"}
\ssgoto{"N_{14}"}
\ssstroke
\ssgoto{"M_{40}"}
\ssgoto{"N_{41}"}
\ssstroke
\ssgoto{"M_{43}"}
\ssgoto{"N_{44}"}
\ssstroke
\ssgoto{"M_{44}"}
\ssgoto{"N_{45}"}
\ssstroke
\ssgoto{"M_{46}"}
\ssgoto{"N_{47}"}
\ssstroke
\ssgoto{"M_{49}"}
\ssgoto{"N_{50}"}
\ssstroke
\ssgoto{"M_{52}"}
\ssgoto{"N_{53}"}
\ssstroke
\ssgoto{"M_{55}"}
\ssgoto{"N_{56}"}
\ssstroke
\ssgoto{"M_{60}"}
\ssgoto{"N_{61}"}
\ssstroke
\ssgoto{"M_{66}"}
\ssgoto{"N_{67}"}
\ssstroke
\ssgoto{"M_{67}"}
\ssgoto{"N_{68}"}
\ssstroke
\ssgoto{"M_{73}"}
\ssgoto{"N_{74}"}
\ssstroke
\ssgoto{"M_{75}"}
\ssgoto{"N_{76}"}
\ssstroke
\ssgoto{"M_{76}"}
\ssgoto{"N_{77}"}
\ssstroke
\ssgoto{"M_{78}"}
\ssgoto{"N_{79}"}
\ssstroke
\ssgoto{"M_{81}'"}
\ssgoto{"N_{82}"}
\ssstroke
\ssgoto{"M_{84}"}
\ssgoto{"N_{85}'"}
\ssstroke
\ssgoto{"M_{84}'"}
\ssgoto{"N_{85}"}
\ssstroke
\ssgoto{"N_{14}"}
\ssgoto{"O_{15}"}
\ssstroke
\ssgoto{"N_{41}"}
\ssgoto{"O_{42}"}
\ssstroke
\ssgoto{"N_{45}"}
\ssgoto{"O_{46}"}
\ssstroke
\ssgoto{"N_{47}"}
\ssgoto{"O_{48}"}
\ssstroke
\ssgoto{"N_{60}"}
\ssgoto{"O_{61}"}
\ssstroke
\ssgoto{"N_{61}"}
\ssgoto{"O_{62}"}
\ssstroke
\ssgoto{"N_{63}"}
\ssgoto{"O_{64}"}
\ssstroke
\ssgoto{"N_{67}"}
\ssgoto{"O_{68}"}
\ssstroke
\ssgoto{"N_{68}"}
\ssgoto{"O_{69}"}
\ssstroke
\ssgoto{"N_{76}"}
\ssgoto{"O_{77}"}
\ssstroke
\ssgoto{"N_{77}"}
\ssgoto{"O_{78}"}
\ssstroke
\ssgoto{"N_{78}'"}
\ssgoto{"O_{79}"}
\ssstroke
\ssgoto{"N_{79}"}
\ssgoto{"O_{80}"}
\ssstroke
\ssgoto{"N_{81}'"}
\ssgoto{"O_{82}"}
\ssstroke
\ssgoto{"N_{82}"}
\ssgoto{"O_{83}"}
\ssstroke
\ssgoto{"N_{84}"}
\ssgoto{"O_{85}"}
\ssstroke
\ssgoto{"N_{84}'"}
\ssgoto{"O_{85}"}
\ssstroke
\ssgoto{"N_{85}"}
\ssgoto{"O_{86}'"}
\ssstroke
\ssgoto{"N_{85}'"}
\ssgoto{"O_{86}"}
\ssstroke
\ssgoto{"O_{15}"}
\ssgoto{"P_{16}"}
\ssstroke
\ssgoto{"O_{46}"}
\ssgoto{"P_{47}"}
\ssstroke
\ssgoto{"O_{54}"}
\ssgoto{"P_{55}"}
\ssstroke
\ssgoto{"O_{57}"}
\ssgoto{"P_{58}"}
\ssstroke
\ssgoto{"O_{60}"}
\ssgoto{"P_{61}"}
\ssstroke
\ssgoto{"O_{61}"}
\ssgoto{"P_{62}"}
\ssstroke
\ssgoto{"O_{62}"}
\ssgoto{"P_{63}"}
\ssstroke
\ssgoto{"O_{63}"}
\ssgoto{"P_{64}"}
\ssstroke
\ssgoto{"O_{66}"}
\ssgoto{"P_{67}"}
\ssstroke
\ssgoto{"O_{68}"}
\ssgoto{"P_{69}"}
\ssstroke
\ssgoto{"O_{69}"}
\ssgoto{"P_{70}"}
\ssstroke
\ssgoto{"O_{69}'"}
\ssgoto{"P_{70}"}
\ssstroke
\ssgoto{"O_{77}"}
\ssgoto{"P_{78}"}
\ssstroke
\ssgoto{"O_{78}"}
\ssgoto{"P_{79}"}
\ssstroke
\ssgoto{"O_{79}"}
\ssgoto{"P_{80}"}
\ssstroke
\ssgoto{"O_{82}"}
\ssgoto{"P_{83}"}
\ssstroke
\ssgoto{"O_{85}"}
\ssgoto{"P_{86}"}
\ssstroke
\ssgoto{"P_{16}"}
\ssgoto{"Q_{17}"}
\ssstroke
\ssgoto{"P_{54}"}
\ssgoto{"Q_{55}"}
\ssstroke
\ssgoto{"P_{55}"}
\ssgoto{"Q_{56}"}
\ssstroke
\ssgoto{"P_{57}"}
\ssgoto{"Q_{58}"}
\ssstroke
\ssgoto{"P_{58}"}
\ssgoto{"Q_{59}"}
\ssstroke
\ssgoto{"P_{60}"}
\ssgoto{"Q_{61}"}
\ssstroke
\ssgoto{"P_{61}"}
\ssgoto{"Q_{62}"}
\ssstroke
\ssgoto{"P_{62}"}
\ssgoto{"Q_{63}"}
\ssstroke
\ssgoto{"P_{63}"}
\ssgoto{"Q_{64}'"}
\ssstroke
\ssgoto{"P_{63}'"}
\ssgoto{"Q_{64}"}
\ssstroke
\ssgoto{"P_{64}"}
\ssgoto{"Q_{65}"}
\ssstroke
\ssgoto{"P_{66}"}
\ssgoto{"Q_{67}"}
\ssstroke
\ssgoto{"P_{67}"}
\ssgoto{"Q_{68}"}
\ssstroke
\ssgoto{"P_{69}"}
\ssgoto{"Q_{70}"}
\ssstroke
\ssgoto{"P_{70}"}
\ssgoto{"Q_{71}"}
\ssstroke
\ssgoto{"P_{78}"}
\ssgoto{"Q_{79}"}
\ssstroke
\ssgoto{"P_{79}"}
\ssgoto{"Q_{80}"}
\ssstroke
\ssgoto{"P_{80}"}
\ssgoto{"Q_{81}"}
\ssstroke
\ssgoto{"Q_{17}"}
\ssgoto{"R_{18}"}
\ssstroke
\ssgoto{"Q_{52}"}
\ssgoto{"R_{53}"}
\ssstroke
\ssgoto{"Q_{55}"}
\ssgoto{"R_{56}"}
\ssstroke
\ssgoto{"Q_{56}"}
\ssgoto{"R_{57}"}
\ssstroke
\ssgoto{"Q_{58}"}
\ssgoto{"R_{59}"}
\ssstroke
\ssgoto{"Q_{61}"}
\ssgoto{"R_{62}"}
\ssstroke
\ssgoto{"Q_{63}"}
\ssgoto{"R_{64}"}
\ssstroke
\ssgoto{"Q_{64}"}
\ssgoto{"R_{65}'"}
\ssstroke
\ssgoto{"Q_{64}'"}
\ssgoto{"R_{65}"}
\ssstroke
\ssgoto{"Q_{67}"}
\ssgoto{"R_{68}"}
\ssstroke
\ssgoto{"Q_{70}"}
\ssgoto{"R_{71}"}
\ssstroke
\ssgoto{"Q_{79}"}
\ssgoto{"R_{80}"}
\ssstroke
\ssgoto{"Q_{80}"}
\ssgoto{"R_{81}"}
\ssstroke
\ssgoto{"R_{18}"}
\ssgoto{"S_{19}"}
\ssstroke
\ssgoto{"R_{53}"}
\ssgoto{"S_{54}"}
\ssstroke
\ssgoto{"R_{57}"}
\ssgoto{"S_{58}"}
\ssstroke
\ssgoto{"R_{59}"}
\ssgoto{"S_{60}"}
\ssstroke
\ssgoto{"R_{64}"}
\ssgoto{"S_{65}"}
\ssstroke
\ssgoto{"R_{65}"}
\ssgoto{"S_{66}"}
\ssstroke
\ssgoto{"R_{75}"}
\ssgoto{"S_{76}"}
\ssstroke
\ssgoto{"R_{80}"}
\ssgoto{"S_{81}"}
\ssstroke
\ssgoto{"R_{81}"}
\ssgoto{"S_{82}"}
\ssstroke
\ssgoto{"S_{19}"}
\ssgoto{"T_{20}"}
\ssstroke
\ssgoto{"S_{58}"}
\ssgoto{"T_{59}"}
\ssstroke
\ssgoto{"S_{66}"}
\ssgoto{"T_{67}"}
\ssstroke
\ssgoto{"S_{69}"}
\ssgoto{"T_{70}"}
\ssstroke
\ssgoto{"S_{72}"}
\ssgoto{"T_{73}"}
\ssstroke
\ssgoto{"S_{75}"}
\ssgoto{"T_{76}"}
\ssstroke
\ssgoto{"S_{78}"}
\ssgoto{"T_{79}"}
\ssstroke
\ssgoto{"S_{81}"}
\ssgoto{"T_{82}"}
\ssstroke
\ssgoto{"S_{81}'"}
\ssgoto{"T_{82}"}
\ssstroke
\ssgoto{"S_{82}"}
\ssgoto{"T_{83}"}
\ssstroke
\ssgoto{"T_{20}"}
\ssgoto{"U_{21}"}
\ssstroke
\ssgoto{"T_{66}"}
\ssgoto{"U_{67}"}
\ssstroke
\ssgoto{"T_{67}"}
\ssgoto{"U_{68}"}
\ssstroke
\ssgoto{"T_{69}"}
\ssgoto{"U_{70}"}
\ssstroke
\ssgoto{"T_{70}"}
\ssgoto{"U_{71}"}
\ssstroke
\ssgoto{"T_{72}"}
\ssgoto{"U_{73}"}
\ssstroke
\ssgoto{"T_{73}"}
\ssgoto{"U_{74}"}
\ssstroke
\ssgoto{"T_{75}"}
\ssgoto{"U_{76}"}
\ssstroke
\ssgoto{"T_{76}"}
\ssgoto{"U_{77}"}
\ssstroke
\ssgoto{"T_{78}"}
\ssgoto{"U_{79}"}
\ssstroke
\ssgoto{"T_{79}"}
\ssgoto{"U_{80}"}
\ssstroke
\ssgoto{"T_{82}"}
\ssgoto{"U_{83}"}
\ssstroke
\ssgoto{"T_{83}"}
\ssgoto{"U_{84}"}
\ssstroke
\ssgoto{"U_{21}"}
\ssgoto{"V_{22}"}
\ssstroke
\ssgoto{"U_{64}"}
\ssgoto{"V_{65}"}
\ssstroke
\ssgoto{"U_{67}"}
\ssgoto{"V_{68}"}
\ssstroke
\ssgoto{"U_{68}"}
\ssgoto{"V_{69}"}
\ssstroke
\ssgoto{"U_{70}"}
\ssgoto{"V_{71}"}
\ssstroke
\ssgoto{"U_{73}"}
\ssgoto{"V_{74}"}
\ssstroke
\ssgoto{"U_{76}"}
\ssgoto{"V_{77}"}
\ssstroke
\ssgoto{"U_{79}"}
\ssgoto{"V_{80}"}
\ssstroke
\ssgoto{"U_{84}"}
\ssgoto{"V_{85}"}
\ssstroke
\ssgoto{"V_{22}"}
\ssgoto{"W_{23}"}
\ssstroke
\ssgoto{"V_{65}"}
\ssgoto{"W_{66}"}
\ssstroke
\ssgoto{"V_{69}"}
\ssgoto{"W_{70}"}
\ssstroke
\ssgoto{"V_{71}"}
\ssgoto{"W_{72}"}
\ssstroke
\ssgoto{"V_{84}"}
\ssgoto{"W_{85}"}
\ssstroke
\ssgoto{"V_{85}"}
\ssgoto{"W_{86}"}
\ssstroke
\ssgoto{"W_{23}"}
\ssgoto{"X_{24}"}
\ssstroke
\ssgoto{"W_{70}"}
\ssgoto{"X_{71}"}
\ssstroke
\ssgoto{"W_{78}"}
\ssgoto{"X_{79}"}
\ssstroke
\ssgoto{"W_{81}"}
\ssgoto{"X_{82}"}
\ssstroke
\ssgoto{"W_{84}"}
\ssgoto{"X_{85}"}
\ssstroke
\ssgoto{"W_{85}"}
\ssgoto{"X_{86}"}
\ssstroke
\ssgoto{"W_{86}"}
\ssgoto{"X_{87}"}
\ssstroke
\ssgoto{"X_{24}"}
\ssgoto{"Y_{25}"}
\ssstroke
\ssgoto{"X_{78}"}
\ssgoto{"Y_{79}"}
\ssstroke
\ssgoto{"X_{79}"}
\ssgoto{"Y_{80}"}
\ssstroke
\ssgoto{"X_{81}"}
\ssgoto{"Y_{82}"}
\ssstroke
\ssgoto{"X_{82}"}
\ssgoto{"Y_{83}"}
\ssstroke
\ssgoto{"X_{84}"}
\ssgoto{"Y_{85}"}
\ssstroke
\ssgoto{"X_{85}"}
\ssgoto{"Y_{86}"}
\ssstroke
\ssgoto{"X_{86}"}
\ssgoto{"Y_{87}"}
\ssstroke
\ssgoto{"Y_{25}"}
\ssgoto{"Z_{26}"}
\ssstroke
\ssgoto{"Y_{76}"}
\ssgoto{"Z_{77}"}
\ssstroke
\ssgoto{"Y_{79}"}
\ssgoto{"Z_{80}"}
\ssstroke
\ssgoto{"Y_{80}"}
\ssgoto{"Z_{81}"}
\ssstroke
\ssgoto{"Y_{82}"}
\ssgoto{"Z_{83}"}
\ssstroke
\ssgoto{"Y_{85}"}
\ssgoto{"Z_{86}"}
\ssstroke
\ssgoto{"Z_{26}"}
\ssgoto{"AA_{27}"}
\ssstroke
\ssgoto{"Z_{77}"}
\ssgoto{"AA_{78}"}
\ssstroke
\ssgoto{"Z_{81}"}
\ssgoto{"AA_{82}"}
\ssstroke
\ssgoto{"Z_{83}"}
\ssgoto{"AA_{84}"}
\ssstroke
\ssgoto{"AA_{27}"}
\ssgoto{"AB_{28}"}
\ssstroke
\ssgoto{"AA_{82}"}
\ssgoto{"AB_{83}"}
\ssstroke
\ssgoto{"AB_{28}"}
\ssgoto{"AC_{29}"}
\ssstroke
\ssgoto{"AC_{29}"}
\ssgoto{"AD_{30}"}
\ssstroke
\ssgoto{"AD_{30}"}
\ssgoto{"AE_{31}"}
\ssstroke
\ssgoto{"AE_{31}"}
\ssgoto{"AF_{32}"}
\ssstroke
\ssgoto{"AF_{32}"}
\ssgoto{"AG_{33}"}
\ssstroke
\ssgoto{"AG_{33}"}
\ssgoto{"AH_{34}"}
\ssstroke
\ssgoto{"AH_{34}"}
\ssgoto{"AI_{35}"}
\ssstroke
\ssgoto{"AI_{35}"}
\ssgoto{"AJ_{36}"}
\ssstroke
\ssgoto{"AJ_{36}"}
\ssgoto{"AK_{37}"}
\ssstroke
\ssgoto{"AK_{37}"}
\ssgoto{"AL_{38}"}
\ssstroke
\ssgoto{"AL_{38}"}
\ssgoto{"AM_{39}"}
\ssstroke
\ssgoto{"AM_{39}"}
\ssgoto{"AN_{40}"}
\ssstroke
\ssgoto{"AN_{40}"}
\ssgoto{"AO_{41}"}
\ssstroke
\ssgoto{"AO_{41}"}
\ssgoto{"AP_{42}"}
\ssstroke
\ssgoto{"AP_{42}"}
\ssgoto{"AQ_{43}"}
\ssstroke
\ssgoto{"AQ_{43}"}
\ssgoto{"AR_{44}"}
\ssstroke
\ssgoto{"AR_{44}"}
\ssgoto{"AS_{45}"}
\ssstroke
\ssgoto{"AS_{45}"}
\ssgoto{"AT_{46}"}
\ssstroke
\ssgoto{"AT_{46}"}
\ssgoto{"AU_{47}"}
\ssstroke
\ssgoto{"AU_{47}"}
\ssgoto{"AV_{48}"}
\ssstroke
\ssgoto{"AV_{48}"}
\ssgoto{"AW_{49}"}
\ssstroke
\ssgoto{"AW_{49}"}
\ssgoto{"AX_{50}"}
\ssstroke
\ssgoto{"AX_{50}"}
\ssgoto{"AY_{51}"}
\ssstroke
\ssgoto{"AY_{51}"}
\ssgoto{"AZ_{52}"}
\ssstroke
\ssgoto{"AZ_{52}"}
\ssgoto{"BA_{53}"}
\ssstroke
\ssgoto{"BA_{53}"}
\ssgoto{"BB_{54}"}
\ssstroke
\ssgoto{"BB_{54}"}
\ssgoto{"BC_{55}"}
\ssstroke
\ssgoto{"BC_{55}"}
\ssgoto{"BD_{56}"}
\ssstroke
\ssgoto{"BD_{56}"}
\ssgoto{"BE_{57}"}
\ssstroke
\ssgoto{"BE_{57}"}
\ssgoto{"BF_{58}"}
\ssstroke
\ssgoto{"BF_{58}"}
\ssgoto{"BG_{59}"}
\ssstroke
\ssgoto{"BG_{59}"}
\ssgoto{"BH_{60}"}
\ssstroke
\ssgoto{"BH_{60}"}
\ssgoto{"BI_{61}"}
\ssstroke
\ssgoto{"BI_{61}"}
\ssgoto{"BJ_{62}"}
\ssstroke
\ssgoto{"BJ_{62}"}
\ssgoto{"BK_{63}"}
\ssstroke
\ssgoto{"BK_{63}"}
\ssgoto{"BL_{64}"}
\ssstroke
\ssgoto{"BL_{64}"}
\ssgoto{"BM_{65}"}
\ssstroke
\ssgoto{"BM_{65}"}
\ssgoto{"BN_{66}"}
\ssstroke
\ssgoto{"BN_{66}"}
\ssgoto{"BO_{67}"}
\ssstroke
\ssgoto{"BO_{67}"}
\ssgoto{"BP_{68}"}
\ssstroke
\ssgoto{"BP_{68}"}
\ssgoto{"BQ_{69}"}
\ssstroke
\ssgoto{"BQ_{69}"}
\ssgoto{"BR_{70}"}
\ssstroke
\ssgoto{"BR_{70}"}
\ssgoto{"BS_{71}"}
\ssstroke
\ssgoto{"BS_{71}"}
\ssgoto{"BT_{72}"}
\ssstroke
\ssgoto{"BT_{72}"}
\ssgoto{"BU_{73}"}
\ssstroke
\ssgoto{"BU_{73}"}
\ssgoto{"BV_{74}"}
\ssstroke
\ssgoto{"BV_{74}"}
\ssgoto{"BW_{75}"}
\ssstroke
\ssgoto{"BW_{75}"}
\ssgoto{"BX_{76}"}
\ssstroke
\ssgoto{"BX_{76}"}
\ssgoto{"BY_{77}"}
\ssstroke
\ssgoto{"BY_{77}"}
\ssgoto{"BZ_{78}"}
\ssstroke
\ssgoto{"BZ_{78}"}
\ssgoto{"CA_{79}"}
\ssstroke
\ssgoto{"CA_{79}"}
\ssgoto{"CB_{80}"}
\ssstroke
\ssgoto{"CB_{80}"}
\ssgoto{"CC_{81}"}
\ssstroke
\ssgoto{"CC_{81}"}
\ssgoto{"CD_{82}"}
\ssstroke
\ssgoto{"CD_{82}"}
\ssgoto{"CE_{83}"}
\ssstroke
\ssgoto{"CE_{83}"}
\ssgoto{"CF_{84}"}
\ssstroke
\ssgoto{"CF_{84}"}
\ssgoto{"CG_{85}"}
\ssstroke
\ssgoto{"CG_{85}"}
\ssgoto{"CH_{86}"}
\ssstroke
\ssgoto{"CH_{86}"}
\ssgoto{"CI_{87}"}
\ssstroke
\ssgoto{"A_2"}
\ssgoto{"B_4"}
\ssstroke
\ssgoto{"A_8"}
\ssgoto{"B_{10}"}
\ssstroke
\ssgoto{"A_{16}"}
\ssgoto{"B_{18}"}
\ssstroke
\ssgoto{"A_{32}"}
\ssgoto{"B_{34}"}
\ssstroke
\ssgoto{"A_{64}"}
\ssgoto{"B_{66}"}
\ssstroke
\ssgoto{"B_4"}
\ssgoto{"C_6"}
\ssstroke
\ssgoto{"B_{10}"}
\ssgoto{"C_{12}"}
\ssstroke
\ssgoto{"B_{18}"}
\ssgoto{"C_{20}"}
\ssstroke
\ssgoto{"B_{32}"}
\ssgoto{"C_{34}'"}
\ssstroke
\ssgoto{"B_{34}"}
\ssgoto{"C_{36}"}
\ssstroke
\ssgoto{"B_{40}"}
\ssgoto{"C_{42}"}
\ssstroke
\ssgoto{"B_{64}"}
\ssgoto{"C_{66}'"}
\ssstroke
\ssgoto{"B_{66}"}
\ssgoto{"C_{68}"}
\ssstroke
\ssgoto{"B_{72}"}
\ssgoto{"C_{74}"}
\ssstroke
\ssgoto{"B_{80}"}
\ssgoto{"C_{82}"}
\ssstroke
\ssgoto{"C_{11}"}
\ssgoto{"D_{13}"}
\ssstroke
\ssgoto{"C_{20}"}
\ssgoto{"D_{22}"}
\ssstroke
\ssgoto{"C_{36}"}
\ssgoto{"D_{38}"}
\ssstroke
\ssgoto{"C_{42}"}
\ssgoto{"D_{44}"}
\ssstroke
\ssgoto{"C_{66}'"}
\ssgoto{"D_{68}"}
\ssstroke
\ssgoto{"C_{68}"}
\ssgoto{"D_{70}"}
\ssstroke
\ssgoto{"C_{74}"}
\ssgoto{"D_{76}"}
\ssstroke
\ssgoto{"C_{82}"}
\ssgoto{"D_{84}"}
\ssstroke
\ssgoto{"D_{18}"}
\ssgoto{"E_{20}'"}
\ssstroke
\ssgoto{"D_{21}"}
\ssgoto{"E_{23}"}
\ssstroke
\ssgoto{"D_{24}"}
\ssgoto{"E_{26}"}
\ssstroke
\ssgoto{"D_{27}"}
\ssgoto{"E_{29}"}
\ssstroke
\ssgoto{"D_{36}"}
\ssgoto{"E_{38}"}
\ssstroke
\ssgoto{"D_{42}'"}
\ssgoto{"E_{44}"}
\ssstroke
\ssgoto{"D_{43}"}
\ssgoto{"E_{45}'"}
\ssstroke
\ssgoto{"D_{44}'"}
\ssgoto{"E_{46}"}
\ssstroke
\ssgoto{"D_{48}"}
\ssgoto{"E_{50}"}
\ssstroke
\ssgoto{"D_{65}"}
\ssgoto{"E_{67}'"}
\ssstroke
\ssgoto{"D_{68}"}
\ssgoto{"E_{70}"}
\ssstroke
\ssgoto{"D_{73}"}
\ssgoto{"E_{75}'"}
\ssstroke
\ssgoto{"D_{74}'"}
\ssgoto{"E_{76}"}
\ssstroke
\ssgoto{"D_{75}"}
\ssgoto{"E_{77}"}
\ssstroke
\ssgoto{"D_{84}"}
\ssgoto{"E_{86}"}
\ssstroke
\ssgoto{"E_{14}"}
\ssgoto{"F_{16}"}
\ssstroke
\ssgoto{"E_{20}'"}
\ssgoto{"F_{22}"}
\ssstroke
\ssgoto{"E_{42}"}
\ssgoto{"F_{44}"}
\ssstroke
\ssgoto{"E_{44}"}
\ssgoto{"F_{46}"}
\ssstroke
\ssgoto{"E_{50}'"}
\ssgoto{"F_{52}"}
\ssstroke
\ssgoto{"E_{53}"}
\ssgoto{"F_{55}"}
\ssstroke
\ssgoto{"E_{56}"}
\ssgoto{"F_{58}"}
\ssstroke
\ssgoto{"E_{67}''"}
\ssgoto{"F_{69}'"}
\ssstroke
\ssgoto{"E_{72}"}
\ssgoto{"F_{74}"}
\ssstroke
\ssgoto{"E_{75}'"}
\ssgoto{"F_{77}"}
\ssstroke
\ssgoto{"E_{81}"}
\ssgoto{"F_{83}'"}
\ssstroke
\ssgoto{"E_{82}"}
\ssgoto{"F_{84}'"}
\ssstroke
\ssgoto{"E_{84}"}
\ssgoto{"F_{86}'"}
\ssstroke
\ssgoto{"E_{85}'"}
\ssgoto{"F_{87}"}
\ssstroke
\ssgoto{"F_{16}"}
\ssgoto{"G_{18}"}
\ssstroke
\ssgoto{"F_{22}"}
\ssgoto{"G_{24}"}
\ssstroke
\ssgoto{"F_{38}"}
\ssgoto{"G_{40}"}
\ssstroke
\ssgoto{"F_{42}"}
\ssgoto{"G_{44}'"}
\ssstroke
\ssgoto{"F_{44}'"}
\ssgoto{"G_{46}"}
\ssstroke
\ssgoto{"F_{46}'"}
\ssgoto{"G_{48}"}
\ssstroke
\ssgoto{"F_{52}"}
\ssgoto{"G_{54}"}
\ssstroke
\ssgoto{"F_{60}"}
\ssgoto{"G_{62}"}
\ssstroke
\ssgoto{"F_{74}"}
\ssgoto{"G_{76}"}
\ssstroke
\ssgoto{"F_{75}'"}
\ssgoto{"G_{77}'"}
\ssstroke
\ssgoto{"F_{78}"}
\ssgoto{"G_{80}"}
\ssstroke
\ssgoto{"F_{82}"}
\ssgoto{"G_{84}'"}
\ssstroke
\ssgoto{"F_{82}"}
\ssgoto{"G_{84}''"}
\ssstroke
\ssgoto{"F_{84}'"}
\ssgoto{"G_{86}"}
\ssstroke
\ssgoto{"G_{23}"}
\ssgoto{"H_{25}"}
\ssstroke
\ssgoto{"G_{48}"}
\ssgoto{"H_{50}"}
\ssstroke
\ssgoto{"G_{53}"}
\ssgoto{"H_{55}"}
\ssstroke
\ssgoto{"G_{64}"}
\ssgoto{"H_{66}"}
\ssstroke
\ssgoto{"G_{67}"}
\ssgoto{"H_{69}"}
\ssstroke
\ssgoto{"G_{70}'"}
\ssgoto{"H_{72}'"}
\ssstroke
\ssgoto{"G_{70}''"}
\ssgoto{"H_{72}"}
\ssstroke
\ssgoto{"G_{70}''"}
\ssgoto{"H_{72}'"}
\ssstroke
\ssgoto{"G_{73}'"}
\ssgoto{"H_{75}"}
\ssstroke
\ssgoto{"G_{80}"}
\ssgoto{"H_{82}"}
\ssstroke
\ssgoto{"G_{82}"}
\ssgoto{"H_{84}'"}
\ssstroke
\ssgoto{"G_{84}'"}
\ssgoto{"H_{86}'"}
\ssstroke
\ssgoto{"G_{84}''"}
\ssgoto{"H_{86}''"}
\ssstroke
\ssgoto{"G_{84}'''"}
\ssgoto{"H_{86}''"}
\ssstroke
\ssgoto{"H_{30}"}
\ssgoto{"I_{32}'"}
\ssstroke
\ssgoto{"H_{33}"}
\ssgoto{"I_{35}"}
\ssstroke
\ssgoto{"H_{36}"}
\ssgoto{"I_{38}"}
\ssstroke
\ssgoto{"H_{39}'"}
\ssgoto{"I_{41}"}
\ssstroke
\ssgoto{"H_{42}"}
\ssgoto{"I_{44}"}
\ssstroke
\ssgoto{"H_{45}'"}
\ssgoto{"I_{47}"}
\ssstroke
\ssgoto{"H_{55}"}
\ssgoto{"I_{57}"}
\ssstroke
\ssgoto{"H_{55}'"}
\ssgoto{"I_{57}'"}
\ssstroke
\ssgoto{"H_{70}'"}
\ssgoto{"I_{72}''"}
\ssstroke
\ssgoto{"H_{72}''"}
\ssgoto{"I_{74}"}
\ssstroke
\ssgoto{"H_{76}'"}
\ssgoto{"I_{78}'"}
\ssstroke
\ssgoto{"H_{77}"}
\ssgoto{"I_{79}"}
\ssstroke
\ssgoto{"H_{80}'"}
\ssgoto{"I_{82}"}
\ssstroke
\ssgoto{"H_{82}'"}
\ssgoto{"I_{84}"}
\ssstroke
\ssgoto{"H_{82}'"}
\ssgoto{"I_{84}'"}
\ssstroke
\ssgoto{"H_{83}'"}
\ssgoto{"I_{85}"}
\ssstroke
\ssgoto{"H_{84}'"}
\ssgoto{"I_{86}'"}
\ssstroke
\ssgoto{"I_{26}"}
\ssgoto{"J_{28}"}
\ssstroke
\ssgoto{"I_{32}'"}
\ssgoto{"J_{34}"}
\ssstroke
\ssgoto{"I_{48}"}
\ssgoto{"J_{50}"}
\ssstroke
\ssgoto{"I_{62}"}
\ssgoto{"J_{64}"}
\ssstroke
\ssgoto{"I_{65}"}
\ssgoto{"J_{67}"}
\ssstroke
\ssgoto{"I_{70}"}
\ssgoto{"J_{72}"}
\ssstroke
\ssgoto{"I_{72}''"}
\ssgoto{"J_{74}"}
\ssstroke
\ssgoto{"I_{76}'"}
\ssgoto{"J_{78}'"}
\ssstroke
\ssgoto{"I_{78}'"}
\ssgoto{"J_{80}"}
\ssstroke
\ssgoto{"I_{84}"}
\ssgoto{"J_{86}"}
\ssstroke
\ssgoto{"I_{84}'"}
\ssgoto{"J_{86}"}
\ssstroke
\ssgoto{"J_{28}"}
\ssgoto{"K_{30}"}
\ssstroke
\ssgoto{"J_{34}"}
\ssgoto{"K_{36}"}
\ssstroke
\ssgoto{"J_{50}"}
\ssgoto{"K_{52}"}
\ssstroke
\ssgoto{"J_{63}"}
\ssgoto{"K_{65}'"}
\ssstroke
\ssgoto{"J_{66}"}
\ssgoto{"K_{68}"}
\ssstroke
\ssgoto{"J_{69}"}
\ssgoto{"K_{71}'"}
\ssstroke
\ssgoto{"J_{72}"}
\ssgoto{"K_{74}"}
\ssstroke
\ssgoto{"J_{72}"}
\ssgoto{"K_{74}''"}
\ssstroke
\ssgoto{"J_{72}'"}
\ssgoto{"K_{74}"}
\ssstroke
\ssgoto{"J_{72}'"}
\ssgoto{"K_{74}''"}
\ssstroke
\ssgoto{"J_{72}''"}
\ssgoto{"K_{74}''"}
\ssstroke
\ssgoto{"J_{74}'"}
\ssgoto{"K_{76}'"}
\ssstroke
\ssgoto{"J_{75}"}
\ssgoto{"K_{77}"}
\ssstroke
\ssgoto{"K_{35}"}
\ssgoto{"L_{37}"}
\ssstroke
\ssgoto{"K_{65}'"}
\ssgoto{"L_{67}"}
\ssstroke
\ssgoto{"K_{76}'"}
\ssgoto{"L_{78}"}
\ssstroke
\ssgoto{"K_{76}'"}
\ssgoto{"L_{78}'"}
\ssstroke
\ssgoto{"K_{82}"}
\ssgoto{"L_{84}"}
\ssstroke
\ssgoto{"L_{42}"}
\ssgoto{"M_{44}'"}
\ssstroke
\ssgoto{"L_{45}"}
\ssgoto{"M_{47}"}
\ssstroke
\ssgoto{"L_{48}"}
\ssgoto{"M_{50}"}
\ssstroke
\ssgoto{"L_{51}"}
\ssgoto{"M_{53}"}
\ssstroke
\ssgoto{"L_{54}"}
\ssgoto{"M_{56}"}
\ssstroke
\ssgoto{"L_{67}"}
\ssgoto{"M_{69}"}
\ssstroke
\ssgoto{"L_{83}'"}
\ssgoto{"M_{85}"}
\ssstroke
\ssgoto{"M_{38}"}
\ssgoto{"N_{40}"}
\ssstroke
\ssgoto{"M_{44}'"}
\ssgoto{"N_{46}"}
\ssstroke
\ssgoto{"M_{60}'"}
\ssgoto{"N_{62}"}
\ssstroke
\ssgoto{"M_{82}"}
\ssgoto{"N_{84}"}
\ssstroke
\ssgoto{"M_{82}"}
\ssgoto{"N_{84}'"}
\ssstroke
\ssgoto{"N_{40}"}
\ssgoto{"O_{42}"}
\ssstroke
\ssgoto{"N_{46}"}
\ssgoto{"O_{48}"}
\ssstroke
\ssgoto{"N_{62}"}
\ssgoto{"O_{64}"}
\ssstroke
\ssgoto{"N_{78}'"}
\ssgoto{"O_{80}"}
\ssstroke
\ssgoto{"N_{81}'"}
\ssgoto{"O_{83}"}
\ssstroke
\ssgoto{"N_{84}"}
\ssgoto{"O_{86}"}
\ssstroke
\ssgoto{"N_{84}'"}
\ssgoto{"O_{86}"}
\ssstroke
\ssgoto{"N_{84}'"}
\ssgoto{"O_{86}'"}
\ssstroke
\ssgoto{"O_{47}"}
\ssgoto{"P_{49}"}
\ssstroke
\ssgoto{"O_{77}'"}
\ssgoto{"P_{79}'"}
\ssstroke
\ssgoto{"P_{54}"}
\ssgoto{"Q_{56}'"}
\ssstroke
\ssgoto{"P_{57}"}
\ssgoto{"Q_{59}"}
\ssstroke
\ssgoto{"P_{60}"}
\ssgoto{"Q_{62}"}
\ssstroke
\ssgoto{"P_{63}'"}
\ssgoto{"Q_{65}"}
\ssstroke
\ssgoto{"P_{66}"}
\ssgoto{"Q_{68}"}
\ssstroke
\ssgoto{"P_{69}'"}
\ssgoto{"Q_{71}"}
\ssstroke
\ssgoto{"P_{79}'"}
\ssgoto{"Q_{81}"}
\ssstroke
\ssgoto{"Q_{50}"}
\ssgoto{"R_{52}"}
\ssstroke
\ssgoto{"Q_{56}'"}
\ssgoto{"R_{58}"}
\ssstroke
\ssgoto{"Q_{72}"}
\ssgoto{"R_{74}"}
\ssstroke
\ssgoto{"R_{52}"}
\ssgoto{"S_{54}"}
\ssstroke
\ssgoto{"R_{58}"}
\ssgoto{"S_{60}"}
\ssstroke
\ssgoto{"R_{74}"}
\ssgoto{"S_{76}"}
\ssstroke
\ssgoto{"S_{59}"}
\ssgoto{"T_{61}"}
\ssstroke
\ssgoto{"T_{66}"}
\ssgoto{"U_{68}'"}
\ssstroke
\ssgoto{"T_{69}"}
\ssgoto{"U_{71}"}
\ssstroke
\ssgoto{"T_{72}"}
\ssgoto{"U_{74}"}
\ssstroke
\ssgoto{"T_{75}"}
\ssgoto{"U_{77}"}
\ssstroke
\ssgoto{"T_{78}"}
\ssgoto{"U_{80}"}
\ssstroke
\ssgoto{"T_{81}"}
\ssgoto{"U_{83}"}
\ssstroke
\ssgoto{"U_{62}"}
\ssgoto{"V_{64}"}
\ssstroke
\ssgoto{"U_{68}'"}
\ssgoto{"V_{70}"}
\ssstroke
\ssgoto{"U_{84}'"}
\ssgoto{"V_{86}"}
\ssstroke
\ssgoto{"V_{64}"}
\ssgoto{"W_{66}"}
\ssstroke
\ssgoto{"V_{70}"}
\ssgoto{"W_{72}"}
\ssstroke
\ssgoto{"W_{71}"}
\ssgoto{"X_{73}"}
\ssstroke
\ssgoto{"X_{78}"}
\ssgoto{"Y_{80}'"}
\ssstroke
\ssgoto{"X_{81}"}
\ssgoto{"Y_{83}"}
\ssstroke
\ssgoto{"X_{84}"}
\ssgoto{"Y_{86}"}
\ssstroke
\ssgoto{"Y_{74}"}
\ssgoto{"Z_{76}"}
\ssstroke
\ssgoto{"Y_{80}'"}
\ssgoto{"Z_{82}"}
\ssstroke
\ssgoto{"Z_{76}"}
\ssgoto{"AA_{78}"}
\ssstroke
\ssgoto{"Z_{82}"}
\ssgoto{"AA_{84}"}
\ssstroke
\ssgoto{"AA_{83}"}
\ssgoto{"AB_{85}"}
\ssstroke

  \end{sseq}
  \]


We have here the a diagram of the first 60 stems of the $E_2$ page of the Adams Spectral Sequence.
The horizontal axis represents $t-s$ and the vertical axis represents $s$, so a dot in slot $(s,t-s)$ represents a $\F_2$ summand of $E_s^{s,t-s}$.  
The vertical lines represent multiplication (see Theorem~\ref{sec:product}) by $h_0$, which converges to multiplication by 2 in $\pi_*^s$ (see Section \ref{sec:hopfe2} and \ref{sec:HopfInvariant} for a definition of $h_i$); that is, $x$ and $h_0x$ are connected by a vertical line.  
The diagonal lines represent multiplication by $h_1$, the element of the Adams Spectral Sequence detecting the Hopf Fibration.  
A differential, $d_r$, will go 1 square left and $r$ squares up.  
Just from this picture and the Leibniz Rule, we can reason that all differentials before $t-s=13$ are zero, as mentioned in Section \label{sec:pairing}.

One is forced to notice an imperfect pattern when staring at this picture.  
The patterns in the dots seem to have a mod-8 periodicity  towards the top of the diagram.
This is very deep and consequence of the so-called $J$-homomorphism and Bott Periodicity.  
Indeed, the three dots in the ``triangles'' at the top of the diagram ending when $t-s\equiv 3$ (mod 8) are exactly the image of $J$.
The long vertical ``towers'' when $t-s\equiv 7$ (mod 8) are also the image of the $J$ in those dimensions, and these converge to large cyclic summands of $\pi_*^s$, in part explaining the apparent jump in the size of $\pi_i^s$ when $i\equiv 7$ (mod 8).  
The reader is referred to \cite{RavenelGreen} and \cite{imJ}.   


