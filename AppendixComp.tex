\section{Computational Methods}

We implemented computer algorithms in the programming language Haskell to do the computations described in this paper.
The computations include computing free resolutions, $\Ext_\A$ modules, Yoneda Products and Steenrod Operations on $\Ext_\A$.  
Many of the algorithms in this section are similar, if not identical, to the ones in \cite{brunext}, although the author discovered them independently.  


\subsection{Representing Modules}

All modules in this program are free.
For a ring $R$ and (possibly infinite) totally ordered set $B$, let $R(B)$ be the free $R$-module with basis $B$.
Any vector in this module can be written uniquely in the form
\[\sum_{i=1}^n r_ib_i\]
where the $b_i$ are unique and the $r_i\ne 0$ for all $i$.
We represent this vector as a map data structure built on a balanced binary search tree, with keys $b_i$ and values $r_i$.
It is apparent, then, that we can efficiently do scalar multiplication, addition and coefficient look-up.
While operations are never destructive in Haskell, much memory can be shared between vectors involved in the same computations; if $x+y=z$, then $z$ may share internal pointers with $x$ and $y$.  
Tricks are used to prevent unnecessary copying of vector data-structures, for instance, $1\cdot x$ returns $x$ in $O(1)$ time and memory.
We found this to be a substantial bonus, especially in the case of interest $R=\F_2$.

Suppose that $R=k(B)$ is itself a ring, which is true if there is a specified map of sets from $B\times B \to R$.
Consider the free $R$-module $R(C)$.  
This is represented by a binary search tree whose nodes are labeled with binary search trees.
However, it is often convenient to consider $R(C)$ as a $k$-vector space.
We use the isomorphism
\[R(C)\cong R\otimes_R R(C) \cong k(C\times B)\]
We denote the tuples in the basis $C\times B$ by $c\otimes b$, and switch between these two representations when necessary.  

\subsection{Representing Subspaces}
\label{sec:ss}
\begin{Def}
  Let  $v_1,...,v_n\in R(B)$, all nonzero, and write
  \[v_{i,j} = \sum_{j=1}^{n_i} r_{i,j}b_{i,j}\]
  ordered so that for all $i$, if $j<j'$ then $b_{i,j}<b_{i,j'}$ and with all $r_{i,j}\ne 0$.   
  We say the set is lower triangular if, $i<i'$ implies
  \[b_{i,1}<b_{i',1}\]
  It is clear that the $v_i$ are linearly independent.  
\end{Def}

We represent $n$-dimensional subspaces of $R(B)$ by a set of $n$ lower triangular vectors, 
where these vectors are stored in a set data structure built on a balanced binary search tree.  

Let $S=\langle v_1,..,v_n\rangle\subset k(B)$ be a subspace represented in this way, with $k$ a field.  
Any vector $v$ can be written as
\[v=u+\sum a_i v_i\]
with $u\notin S$.  
Recovering the $r_i$ is as easy as solving an over-constrained lower-triangular system of linear equations.  
If the $v_i$ and $v$ are dense, this can be expensive, but if they are sparse (as is often the case), this is relatively cheap.
If the $v_i$ have on average $l$ nonzero coefficients then the run time of this operation is $O(nl)$. 
\begin{Def}
  We say we have reduced $v$ by $S$ if we have written it in the form above
\end{Def}

This has many applications.  
One application is to see if $x=y$ in $k(B)/S$, which specializes to seeing if $x\in S$ when $y=0$.
Another is creating the subspace $S + \langle x\rangle$: simply let $x'$ be $x$ reduced by $S$ and, if $x'\ne 0$, insert it into to the set.
Notice that $x'$ has minimal nonzero entry distinct from any such in $S$, so this algorithm allows us to build $S$ from the zero subspace by adding one vector at a time.

Finally, let $X\oplus Y\subset k(B)$ with $X$ and $Y$ subspaces (in the mathematical sense) and consider a map
\[f: X\to Y\]
Let $x_1,...,x_n$ be vectors in $X$ with $f(x_1),...,f(x_n)$ linearly independent in $Y$, and let $y\in \langle f(x_1),...,f(x_n)\rangle$.
To compute a preimage of $y$ in $X$, reduce 
\[y=\sum a_i f(x_i)\]
and conclude that
\[f^{-1}(y)=\sum a_i x_i\]
is a preimage.
Better yet, this process is linear in $y$, which is algebraically convenient.  


\subsection{Calculating $\Ext_A(X,k)$}

There is a relatively straightforward way to calculate $\Ext_A(X,k)$ if $A$ is a graded $k$-algebra with the property that $A^0=k$ (we say in this case that $A$ is connective).  
To do this, we use a ``minimal'' resolution of $A$, that is, a resolution constructed inductively with the least number of generators in each degree in each dimension and a preferred generator set.  
Let me describe the algorithm.
We want a resolution
\[X\from F_1\from F_2\from F_3\from ...\]
The generators of $X$ as an $A$ module are the generators of $F_1$, mapping in the obvious way into $A$.  
Construct $F_i$ inductively as follows.  Let $F_i^j$ be the degree $j$ elements in $F_i$, and $A^i$ be elements of $A$ of degree $i$.  
The algorithm will work by adding generators where needed, so let $G_i^j$ be the generators added in $F_i^j$.  Define
\[(F_i^j)'=\langle sg|g\in G^k_i, k<j, s\in A^{j-k} \rangle\]
\[I=d((F_i^j)')\subseteq F_{i-1}^{j}\]
\[K=\ker(d:(F_{i-1}^{j})'\to (F_{i-2}^{j})')\]
\[F_i^j= (F_i^j)'\oplus \langle G_i^j\rangle\]
The set $G_i^j$ is formed by adding generators to ${F_i^j}'$ until $d|_I:I\to K$ is an isomorphism.  
Notice that we can use $(F_{i-1}^{j})'$ instead of $F_{i-1}^{j}$ since the kernel of $d$ is the same on both modules.  
It is easy to see that, by construction, these are free-$A$ modules and the differentials are exact.  

To make this explicit, when computing $F^j_i$, we compute $K$ and $I$, stored as $k$-subspaces as described in Section \ref{sec:ss}.
Then we simply reduce the basis vectors in $u\in K$ against $I$ and, if we find a $u\notin I$, we add an element to $G_i^j$ whose differential is the reduction of $u$ and add $u$ to $I$.  
What was $I$ during the computation of $F_i^j$ is reused as $K$ during the computation of $F_{i+1}^j$, and $K$ is freed.  

\begin{Lemma}
  \[\Ext^s_A(X,k)\cong k(G_s^*)\]
  That is, $\Ext^*_A(X,k)$ is $k$-free on the $A$-generators of a minimal resolution $F_*$.  
\end{Lemma}
\begin{proof}
  Let $\sum_i s_ig_i$ be a homogeneous boundary, with the $g_i\in G_*^*$ and $s_i\in A$.  
  Then all the $s_i$ are homogeneous, and cannot be $1$ by the construction.  
  But any $\phi\in \Hom_A(F_*,k)$ will have 
  \[\phi\left(\sum s_ig_i\right)=s_i\left(\sum \phi(g_i)\right)=0\]
  since $s_i\cdot 1=0$ for $s_i\in A^j$, for $j>0$, for grading reasons.
  Thus  $d^*\phi=\phi\circ d=0$.  
  Thus all differentials are zero, so
  \[\Ext_A(X,k)=\Hom_A(F_*,k)\]
  This is isomorphic to the $k$-vector space on the same basis as $F_*$.  
\end{proof}

\subsection{Yoneda Products}

We want to compute the module structure
\[\Ext_A(M,k)\otimes \Ext_A(k,k)\to \Ext_A(M,k)\]
given by Yoneda products (see \ref{sec:ExtStructure}).
It is clear that, to do this, we need only compute the isomorphism  between $\Ext_A(M,k)$ and the homotopy classes of chain maps between $A$-free resolutions of of $M$ and $k$.
Let $F_*\to M$ and $K_*\to k$ be minimal $A$-resolutions, computed as above (in particular, with specified $A$-basis), and $g$ a $A$-basis vector of $F_s$. 
Let $g^*$ denote the cocycle which sends $g\mapsto 1$ and all other basis vectors of $F_*$ to zero.
Let $g^*_i:F_{s+t}\to K_i$ be the a chain map associated to $g^*$ and the object of our computation.
Then $g^*_0$ sends $g$ to $1\in F_0=A$ and the rest of the  $A$-basis of $F_s$ to 0.  
We can compute $g^*_i$ inductively.  
We have the law that for each basis vector $u\in F_{s+i}$ for $i> 0$,
\[dg^*_iu=g^*_{i-1}du\]
So to compute $g^*_iu$, merely pick a preimage of the differential on $g^*_{i-1}du$ by reducing it against an easily computable $k$-basis for the image of the differential in the proper grading.  

Once the cocycles are extended to chain maps, it is easy to compute $g^*\times h^*$ for $h^*$ dual to an $A$-basis vector of $K_{s'}$.  
The formula is
\[g^*\times h^* = \sum_l h^*(g^*(l)) l^*\]
where $l$ runs over all $A$-basis vectors in the appropriate grading of $F_{s+s'}$.  

\subsection{Steenrod Operations}

While the Steenrod Operations give us much information about the differentials, it turns out they are massively difficult to compute.
The formula
\[\Delta_n\partial + \partial\Delta_n = \rho\Delta_{n-1}+\Delta_{n-1}\]
from Appendix \ref{sec:SteenrodConstruction} gives (using the same trick to invert differentials as above) an algorithm for computing Steenrod Operations, but it requires tensoring already huge resolutions together, which is computationally infeasible.  
Instead, there is a trick, communicated to me by Bob Bruner, who claimed it was due to Christian Nassau.

Recall that the classes in $\Ext_A^s(k,k)$ are in bijection with equivalences classes of extensions of length $s$ from $k$ to $k$.
Let $F_*$ be an $A$-free resolution of $k$ and let
\[\mathcal{X}_*=0\to k \to X_{s-1}\to ...\to X_0\to k\to 0\]
 be such an extension.
If there is a chain map $F_*\to \mathcal{X}_*$ such that in dimension $s$ the map $F_s\to k$ is a cocycle $u$ and in dimension -1 is identity $k\to k$, then this extension represents $u$.
Tensoring this extension with itself, we get a length $2s$ extension, and using the free resolution comparison lemma we can find a chain map 
\[\Delta_0:F_*\to \mathcal{X}_*\otimes \mathcal{X}_*\]
over the isomorphism $k\to k\otimes k$, and this cocycle will represent $u^2$.
Of course, the switching map $\rho$ is realizable in $\mathcal{X}_*\otimes \mathcal{X}_*$, so we can inductively compute
\[\Delta_n : F_*\to \mathcal{X}_*\otimes \mathcal{X}_*\]
by the formula
\[\Delta_n\partial + \partial\Delta_n = \rho\Delta_{n-1}+\Delta_{n-1}\]
And use this to compute Steenrod Operations.

The reason this helps is that the size of $\mathcal{X}_*$ can be, in principle, much smaller than $F_*$, since it need not be $A$-free.
However, finding small extensions algorithmically is quite difficult.  
The obvious extension, given a nonzero cocycle $u:F_s\to k$, is to set $X_{s-1}=F_{s-1}/d_s(\ker u)$ and $X_i=F_i$ for $0 \le i < s-1$.  
Clearly this is just as big as $F_*$ itself, and of little use in and of itself.
It does however provide a starting point.

Suppose we already have an extension $\mathcal{X}_*$ representing $u$, but we feel it is too big.
If we can find a subcomplex $L_*\subset \mathcal{X}_*$ with $\mathcal{X}_*/L_*$ exact and the $k$ at either end preserved, we will have succeeded in finding a smaller extension.
Notice that property of $\mathcal{X}_*/L_*$ being exact is equivalent to, for each $i$
\[d_i(L_i) = \im(d_i)\cap L_{i-1}\]
which is itself equivalent to
\[\frac{L_{i-1}\cap \im(d_i)}{L_{i-1}\cap d_i(L_i)} = 0 \subset \frac{X_{i-1}}{d_i(L_i)}\]
\begin{Def}
  Finding $L_*\subset \mathcal{X}_*$ satisfying the properties above with $L_*$ as large as possible is called the ``extension slimming problem''.  
\end{Def}

One strategy for solving this problem is to use a greedy algorithm.
That is, start with $X_{s-1}$ and find $L_{s-1}\subset X_{s-1}$ away from the image of $\im(d_s)=k$.  Then, for each $i$ from $s-2$ to $1$, set
\[X_i'=X_i/d_{i+1}(L_{i+1})\]
And find $L_i'\subset X_i'$ such that
\[L_i'\cap \im(\overline{d}_{i+1}) = 0\]
where 
\[\overline{d}_{i+1}:X_{i+1}'\to X'\]
is the induced differential.  
Clearly $L_i$ can be constructed by lifting the $L_i'$.
To find the $L_i'$, we need to solve the following problem.  
\begin{Def}
  Let $X$ be an $R$-module and $I\subset X$ a submodule.
  The ``Submodule Avoidance Problem'' is to find a submodule $L\subset X$ as large as possible with $I\cap L=0$
\end{Def}

We will reduce this to the following problem, which we will show is NP-hard.

\begin{Def}
  Let $V$ be a $k$-vector space, and let $S_1,...,S_n\subset V$ be subspaces.
  The ``Subspace Union Avoidance Problem'' is to find a subspace $W\subset V$ with $W\cap S_i=0$ for each $i$
\end{Def}

\begin{Lemma}
  The submodule avoidance problem is at least as hard as the subspace union avoidance problem if $R$ is a $k$-algebra.
\end{Lemma}

\begin{proof}
  Let $V, S_1,...,S_n$ be a subspace union avoidance problem.
  Let $V_i$ be a copy of $V$ for each $i$.  Adopt the notation that if $v\in V$, then $v_i$ is the copy of the vector $v$ in $V_i$.  
  Let
  \[R=k[x_1,...,x_n]\]
  be a polynomial ring 
  Let   
  \[X=V\oplus \left(\bigoplus_{i=1}^n V_i\right)\]
  and let $R$ act on $X$ by
  \[x_jv_i = v_j\]
  Let
  \[\Phi : X \to V\]
  be defined by $v_i\mapsto v$.
  Finally, denote the copy of $S_i\subset V_i$ as $T_i$, and let 
  \[I=\bigoplus_i T_i\subset X\]
  
  It is clear that $W\subset V$ is a solution to the original subspace union avoidance problem if and only if the $R$-module generated by $W$ is a solution to the submodule avoidance problem.
  Likewise, $L$ is a solution to the submodule avoidance problem we created if and only if $\Phi(L)$ is a solution to the original subspace union avoidance problem, since if $v^{(0)},v^{(1)},...,v^{(n)}\subset V$ and $v=\sum_{i=0}^n(v^{(i)})_i\in X$, then $\Phi(v) =\sum_{i=0}^n v^{(i)}$ is in some $S_j$ if and only if $x_jv\in T_j$.  
  Since both ways of going between solutions preserve relative $k$-dimensions, the optimal $L$ has $\Phi(L)$ optimal as a solution to the original problem.  
Thus the submodule avoidance problem is at least as hard as the subspace union avoidance problem.
 
\end{proof}

\begin{Lemma}
  \label{sec:npcomplem}
  If $k=\F_p$, deciding if there is a solution to the subspace union avoidance problem of codimension $l$ is NP-hard, even if the $S_i$ are 1-dimensional, unless $l=1$ and $p=2$.  
\end{Lemma}

\begin{Cor}
  The submodule avoidance problem is NP-hard.
\end{Cor}

\begin{proof}[Proof of \ref{sec:npcomplem}:]
  Let $(G,E)$ be a graph and $V=k(G)$ be the free $k$-space on $G$.
  For each edge $uv\in E$, let 
  \[S_i=\langle u-v\rangle\subset V\]
  If there is a $W\subset V$ of codimension $l$, then there are $l$ linearly independent functionals $f_i:V\to \F_p$ such that the kernel of $f_i$ contains $W$ and furthermore $W$ is the intersection of these kernels.
  Notice that this means that for each edge $uv$ there is an $i$ with $f_i(u-v)\ne 0$, so $f_i(u)\ne f_i(v)$.
  Let
  \[f=\bigoplus f_i : V \to \F_p^l\]
  For each edge $uv\in E$, $f(u)\ne f(v)$, so, $f|_{G}$ is a $|\F_p^l|=p^l$-coloring on $(G,V)$.
  To decide if such a thing exists is NP-complete for $l>1$ for all $p$ and when $l>0$ for $p>2$.
\end{proof}
  

\subsection{Automated Differential Propagation}

We automatically propagate information about known differentials using propositional logic.
Consider the following context free grammar
\[
{\setlength\tabcolsep{4pt}
\begin{tabular}{>{$}l<{$}>{$}r<{$}>{$}l<{$}}
  \mbox{BoolNode} &\coloneqq& \mbox{True}\\
  & | & \mbox{False}\\
  & | & \mbox{BoolVar String}\\
  & | & \mbox{And (Set BoolNode)}\\
  & | & \mbox{Or (Set BoolNode)}\\
  & | & \mbox{XOr (XOrSet BoolNode)}\\
  & | & \mbox{Not BoolNode}\\
  & | & \mbox{EqZero SSNode}\\
  \mbox{SSNode}_r^{s,t} &\coloneqq& \mbox{SSConst}_r^{s,t}\\
  & | & \mbox{Differential}_r \mbox{ SSNode}_r^{s-r,t-r+1}\\
  & | & \mbox{Projection}_r\mbox{ SSNode}_2^{s,t}\\
  & | & \mbox{Sum (XOrSet SSNode}_r^{s,t})\\
  & | & \mbox{ScalerMult BoolNode SSNode}_r^{s,t}
\end{tabular}}
  \]
The notation ``And (Set BoolNode)'' means an And node which contains as its data an unordered set of BoolNodes.  An XOrSet is a Set in which insertion is defined by the rule
\begin{verbatim}
insert x set = if set contains x then remove x set else insert x set
\end{verbatim}
A BoolNode represents, equally, a Boolean value True or False or an element of $\F_2$.
A BoolVar node represents a Boolean valued indeterminate.  
An $\mbox{SSNode}_r^{s,t}$ represents an element of $E_r^{s,t}$.  
An $\mbox{SSConst}_r^{s,t}$ represents a known literal element of the group $E_r^{s,t}$.  
A $\mbox{Projection}_r$ node represents the projection of a node to the $r$ page, and is not necessarily defined.  
A ScalerMult node represents ``scalar'' multiplication of an SSNode by a BoolNode, seen as an element of $\F_2$.  
A node ``EqZero $s$'' is true if the value of $s$ is zero and false otherwise.


Our algorithm for computing differentials in $E_r^{s,t}(S,X)$ is as follows.  
We start out knowing the $E_2$ page of the Adams Spectral Sequence, the module structure over the spectral sequence for $X=S$, and the Steenrod Operations.
If $X\ne S$, we also know the differentials in $E_r^{s,t}(S,S)$, meaning we must run this algorithm with $X=S$ first.  
We will maintain a BoolNode which is a quantifier free propositional formula representing all known constraints on the Adams Spectral Sequence.
We will maintain a table of preferred bases for all known groups $E_r^{s,t}$, a table of all known differentials and all known projections, called the page, differential and projection table.  
Finally, we maintain a table of BoolVar substitutions, which we will populate as we learn more about the indeterminates we create, called the variable table.
Our work-loop will run as follows
\begin{enumerate}
\item Simplify constraint formula using rewrite rules, and reduce linear systems using Gaussian Elimination
\item Populate projection and page table using new information learned
\item Add new constraints using updated tables.  
\item Repeat if any tables modified.  
\end{enumerate}

\subsubsection{Rewriting and Simplification}

The rewrite rules we use are mostly standard: things like associativity, identity, nilpotence, etc.
One important thing is the way we treat ScalarMult nodes.
We have obvious rules like
\[\mbox{ScalarMult $b$ (Sum $x_1$ ... $x_n$)}\mapsto 
\mbox{Sum (ScalarMult $b$ $x_1$) ... (ScalarMult $b$ $x_n$)}\]
but also less obvious rules, like, if $x$ is an SSConst, then
\[\mbox{Sum (ScalarMult $b_1$ $x$) ... (ScalarMult $b_n$ $x$)} \mapsto
\mbox{ScalarMult (XOr $b_1$ ... $b_n$) $x$}\]
that is, we gather in the scalar and distribute in the module.  
The reason for this is so we can solve.  
If each $x_i$ is a distinct SSConst, we add the rule
\[\mbox{EqZero (Sum (ScalarMult $b_1$ $x_1$) ... (ScalarMult $b_n$ $x_n$))}
\mapsto \mbox{And (Not $b_1$) ... (Not $b_n$))}\]
and if $x$ is an SSConst,
\[\mbox{EqZero (ScalarMult $b$ $x$)} \mapsto \mbox{Not $b$}\]
For technical reasons, we add, 
\[\mbox{EqZero (Sum ... $x$ ...)}\mapsto \mbox{EqZero (Sum ... (ScalarMult True $x$)...)}\]
so that if $b\cdot x=x$, we can infer $b=True$.  


When building rewriting systems, one must chose how to deal with negations in XOr.
For us, a negated XOr node is an XOr node with True as a child.
To that end, we have the following rules
\[\mbox{Not (Xor ...)} \mapsto \mbox{XOr True ...}\]
\[\mbox{XOr ... (Not $b$) ...} \mapsto\mbox{XOr True ... $b$ ...}\]
And, for ``negated singletons''
\[\mbox{Xor True $b$}\mapsto \mbox{Not $b$}\]
We find this is helpful when we get to the solving phase.  

More interesting are our rules regarding BoolVars.  
As mentioned above, we maintain a variable table, which assigns a BoolVar $b$ to an arbitrary simplified BoolNode $x$ which does not contain $b$ as a descendant.  
In addition, each entry in this table has a ``timestamp'', which says the last time that entry was simplified.  
When the simplification algorithm reaches a $b$, it checks to see if $b$ is in the table.
If there is some $x$ associated to $b$, we check the timestamp to see if $x$ has been simplified since the last time something has changed, and if not we simplify it and update the timestamp.
Either way we rewrite $b$ to $x$.

BoolVars are created in various ways.
For instance, consider the node $\mbox{Differential}_r x$ for some SSConst $x\in E_r^{s,t}$, and suppose nothing is known about this differential.
If we know a basis of $E_r^{s+r,t+r-1}$, say $y_1,...,y_n$, we have
\[d_rx = \sum_i c_iy_i\]
for some $c_i\in \F_2$.  
We then rewrite this to 
\[\mbox{Sum (ScalarMult (BoolVar $c_1$) $y_1$) ... (ScalarMult (BoolVar $c_n$) $y_n$)}\]
Like the variable table, the differential table associates an SSConst $x$ to arbitrary SSNode $y$, and we add the above to the differential table upon its discovery.
Like the variable table, the differential table has timestamps and the entries within are simplified when they become out-of-date.
Thus if later we find more information about the $c_i$, our differential table will be updated accordingly.  

Often, our root is an And node, since it represents a conjunction of constraints.  
We say a BoolNode is ``toplevel'' if it is a direct child of the root And node, or if it is the root and not an And node.
If either a BoolVar $b$ is toplevel and not already in the variable table, we add an entry to the variable table simply associating $b$ to True.
Likewise for the negated situation, Not $b$, we associate $b$ to False.


\subsubsection{Linear Solving}

When no more simplification is possible using the strategies above, we attempt linear solving.
This entails gathering all the toplevel XOr nodes.  Let $\mathcal{V}$ be the free $\F_2$-vector space on the set of all possible BoolNodes.
These XOr nodes together determine a system of $\F_2$-linear equations over $\mathcal{V}$.
We use Tarski's principle component algorithm to split this system up into a direct sum of often many much smaller system, and use Gaussian Elimination to simplify.  
The resulting system is equivalent to the first, but simpler.
When doing the Gaussian Elimination, we order the basis vectors so that BoolVars come first, since they are preferred as pivots.  
For each equation of the form $b+x_1+...+x_n=0$ in the simplified system where the pivot $b$ is a BoolVar, add an entry to the variable table sending $b$ to the BoolNode representing $x_1+...+x_n$.
For the remaining nontrivial equations in the simplified system, simply put them back to the original formula.  
Note that sometimes it could be the case that $n=1$ and $x_1=1$ or $n=0$, in which case we sould just be setting $b$ to a Bool literal.  
If anything is accomplished during this step (that is, anything is added to the variable table), we start the simplification step over.
Otherwise, we go on to the next step.  

\subsubsection{Updating $E_r$ Tables}

We wish to update the tables containing preferred bases for each $E_r^{s,t}$.  
We first simplify everything that needs to be simplified in the differential and projection tables, and update the timestamps accordingly.
In the best case, for a given $E_{r-1}^{s,t}$, we know all the differentials into and out of this summand exactly, in which case we can simply pick projections onto
$E_{r}^{s,t}$ and basis for $E_r^{s,t}$.  
If, we know all of the cycles but not all of the differentials into $E_{r-1}^{s,t}$ (and therefor not all of the boundary relations), we can still often say something intelligent.  
First pick an ordered basis of literals for the cycles $x_1,..,,x_n$ to over approximate our basis.  The actual basis for $E_r^{s,t}$ will be some a subset of these.
If we know the differentials into $E_{r-1}^{s,t}$, the way we would pick relations is by saying if some $y$ in our basis of $E_{r-1}^{s-r,t-r+1}$ has
\[dy=\sum_{i=1}^ic_ix_i\ne 0\]
then, for the minimum $i$ with $c_i\ne 0$, add the relation $x_i=\sum_{j=i+1}^nc_ix_i$.  
We can implement logical ``if-then-else'' statements by 
\[\mbox{If $c$ Then $x$ Else $y$} ~:=~ \mbox{Sum} (\mbox{ScalarMult $c$ $x$}) (\mbox{ScalarMult (Not $c$) $y$})\]
Thus we can make our basis out of the cascading ``if-then-else'' statements on if the basis vectors in $E_{r-1}^{s-r,t-r+1}$ have a coefficient of $x_i$.
Since most $E_r^{s,t}$ are low dimensional (usually 1-dim or 2-dim, occasionally 3-dim in as far as we can compute Ext), the terms stay reasonably small.
Since often times we can know some but not all of the coefficients of the $x_i$, these statements will often simplify further to exactly the sort of reasoning a human might attempt: for instance, if either $dy=x$ or $dy=0$ for the only nonzero $y$ in $E_{r-1}^{s-r,t-r+1}$, we might say the basis element corresponding to $x$ is
\[\mbox{If $c$ Then $0$ Else $x$}~:=~\mbox{ScalarMult (Not $c$) $x$} \]
where $c$ is some BoolVar which is true if and only if $dy=x$.  

\subsubsection{Generating Constraints}

Our three main sources of constraints are the Leibniz Rule (Theorem \ref{sec:product}), the Steenrod Operations (Theorem \ref{sec:dsq1}, Theorem \ref{sec:dsq2}, Theorem \ref{sec:hdiffthm}), and the $J$-homomorphism  (Theorem \label{sec:jthm}).  
The Leibniz Rule constraints are easy to generate: for each new cycle $x\in E_r$ discovered in the previous round and each other known cycle $y$, write
\[d_r(x)y+d_r(y)x+d_r(xy)=0\]
The Steenrod Operation constraints follow much the same way, provided you can evaluate $Sq^ix$, or read them from a table.  

The $\im(J)$ constraints are more interesting.
Since we can calculate the $\im(J)$ directly on the $E_2$ page, we can generate a few constraints.  
Since the $\im(J)$ must last until $E_\infty$, we know these elements are always cycles and never boundaries.
Better yet, we know that if $x\notin\im(J)$ and $h_0x\in\im(J)$, then $x$ must at some point not be a cycle.  
Many differentials, especially near $t-s\equiv 7$, can be inferred just from these rules and Leibniz, so we take great advantage of this.  


  








