\section{The Hopf Invariant}

\label{sec:HopfInvariant}

The Hopf Invariant is a somewhat mysterious invariant of maps between spheres of certain sizes, however, the existence of maps whose Hopf Invariant is equal to 1 is closely related to beautiful and elementary algebraic facts, in particular, the existence of finite dimensional division algebras over $\R$.  
We can define the invariant as follows.  
Let $n$ be an integer
\[f:S^{4n-1}\to S^{2n}\]
Letting $f$ be the attaching map of a $4n$ cell, we can form the complex
\[S^{2n}\cup_f D^{4n}\]
It is easy to see, by cellular cohomology, that the reduced cohomology group is
\[H^*(S^{2n}\cup_f D^{4n})=\Sigma^{2n}\Z\oplus \Sigma^{4n}\Z\]
If we let $\alpha$ be the cohomology class in dimension $2n$ and $\beta$ be the dimension of the cohomology class in $4n$, we have that there is an integer $h$ with
\[\alpha\smile\alpha=h\beta\]
\begin{Def}
  Way say that $h$ is the Hopf Invariant of the (unstable) map $f\in \pi_{4n-1}(S^{2n})$.  This is only well defined up to sign, or choice of generator.  
\end{Def}

It is worth trying to describe a map of Hopf Invariant One.
Let
\[\eta : S^3\to S^2\]
be given as follows by considering $S^2=\C P^2=S^3/S^1$, also written as a fibration
\[S^1\to S^3\to S^2\]
This is known as the Hopf Fibration.  
Notice that 
\[S^2\cup_\eta D^4=\C P^3\]
 and since 
\[H^*(\C P^3) = \Z[\alpha]/(\alpha^3)\]
we have that the Hopf Invariant of $\eta$ is one.  

Because the preimage of any point is a circle, the preimage of a circle is a torus, though the preimage of a line is a mobius strip.  This is described in detail in \cite[Example~4.45]{HatcherAT}, and there are numerous visualizations which can be found, for instance, on YouTube (I even wrote one myself!).  


\begin{Lemma}
  The Hopf Invariant is a map of groups 
  \[H:\pi_{4n-1}(S^{2n})\to \Z\]
\end{Lemma}
\begin{proof}
  THE IDEA IS USE THE DEFINITION OF $+$ AS $\vee$ FOLLOWED BY THE COLLAPSING MAP. 
  I'LL DO THE DETAILS LATER!!!!!!
\end{proof}


As promised, there is a fantastic equivalence
\begin{Theorem}
  When $n$ is even, the following are equivalent:
  \begin{enumerate}
    \item There is an element of Hopf Invariant 1 in $\pi_{2n-1}(S^n)$.
    \item There is an $n$-dimensional division algebra over $\R$ (not necessarily commutative or associative).
    \item There is a map $\mu:S^{n-1}\times S^{n-1}\to S^{n-1}$ an a point $e\in S^{n-1}$ with $\mu(e,x)=\mu(x,e)=x$ for all $x\in S^{n-1}$.  We say that $S^{n-1}$ is an H-space in this case. 
  \end{enumerate}
\end{Theorem}
Note that when $n$ is odd, the hairy ball theorem makes (2) and (3) false.  


\begin{proof}
  To see that (2) implies (3), set
  \[\mu(x,y) = xy/||xy||\]
  where $||.||$ is the Euclidean norm in $\R^n$.  
  Since $\R^n$ is a division algebra, we know $xy\ne 0$ when $x\ne 0$ and $y\ne 0$, so this is well defined and continuous if $x$ and $y$ are on $S^{n-1}$.
  Also, we can assume that $1=(1,0,...,0)$ is the identity of the division algebra structure on $\R^n$, so $\mu$ has an identity element $e=1$.  

  To see that (3) implies (1), suppose we have a $\mu$ and write
  \[S^{n}=D^{n}_+\cup D^{n}_-\]
  where we identify the boundaries and
  \[S^{2n-1}=\partial(D^{n}\times D^{n})=(\partial D^{n})\times D^{n}\cup D^{n}\times(\partial D^{n}) \]
  And define
  \[f(x,y) = \left\{\begin{array}{cc} ||x||\mu(\frac{x}{||x||},y)\in D^n_+ & (x,y)\in (\partial D^{n})\times D^{n}\\
  ||y||\mu(x,\frac{y}{||y||})\in D^n_- & (x,y)\in D^n\times (\partial D^{n})\end{array}\right.\]
  Once can check that this is well defined and continuous.  
  We claim that the Hopf invariant of $f$ is $\pm 1$, or equivalently, that the cup product
  \[H^n(S^n\cup_f D^{2n})\otimes H^n(S^n\cup_f D^{2n}) \to H^{2n}(S^n\cup_f D^{2n})\]
  is surjective.  

  Let $\Phi:D^{2n}\to S^n\cup_f D^{2n}$ be the characteristic map of the $2n$ cell.  I claim we have the following diagram
  \begin{diagram}
    H^n(S^n\cup_f D^{2n})\otimes H^n(S^n\cup_f D^{2n}) & \rTo & H^{2n}(S^n\cup_f D^{2n})\\
    \uTo^\approx                                       &     &   \uTo^\approx\\
    H^n(S^n\cup_f D^{2n},D^n_+)\otimes H^n(S^n\cup_f D^{2n},D^n_-) & \rTo & H^{2n}(S^n\cup_f D^{2n},S^n)\\
    \dOnto^{\Phi^*\otimes\Phi^*}                                      &     &   \dTo^{\Phi^*}_\approx\\
    H^n(D^{n}\times D^{n},(\partial D^{n})\times D^{n})\otimes H^n(D^{n}\times D^{n},D^{n}\times(\partial D^{n})) & \rTo & 
    H^{2n}(D^{n}\times D^n,\partial(D^n\times D^n))\\
    \dTo^\approx & & \uTo^\approx\\
    H^n(D^n\times \{e\},(\partial D^{n})\times \{e\})\otimes H^n(\{e\}\times D^{n},\{e\}\times(\partial D^{n})) & \rTo^\approx &
    H^{2n}(D^n\wedge D^n,\partial(D^n\times D^n))
  \end{diagram}


  The top row is the cup product, which we want to show is a surjection.
  The second row is the relative cup product.  
  Since the left subcomplexes are both contractible, the left map is an isomorphism.
  The right map is an isomorphism because $S^n$ has no effect on $H^{2n}$
  The next row is a obtained by applying the characteristic map $\Phi$.  
  The right vertical map is an isomorphism because of the short exact sequence from the cofibration (the connecting maps must be zero!) 
  \[0\to H^*(S^{2n})\to H^*(S^{n}\cup_f D^{2n})\to H^*(S^{2n})\to 0 \]
  which implies that
  \[\Phi^*:H^*(S^{n}\cup_f D^{2n},S^{2})\to H^*(S^{2n})\]
  is an isomorphism.
  Finally, the bottom row is Kunneth, the bottom left vertical map is an isomorphism because it is a deformation retract and the bottom right vertical map is induced by a homeomorphism.  


Finally, to see that (1) implies (2), first prove that there only exists elements of Hopf Invariant One in $\pi_3(S^2),\pi_7(S^4)$ and $\pi_{15}(S^8)$, corresponding to the complex numbers, quaternions and octonions.  

\end{proof}


