\section{Spectra and the Stable Homotopy Category}

Using the Freudenthal Suspension Theorem \ref{sec:Freudenthal}, we can construct the topological invariant $\pi_*^s(X)$ of stable homotopy groups.  
The functor $\pi_*^s$ itself is somewhat awkward.
It would be nice if we could create a category in which it was representable, that is, where stable maps were simply maps.  
Stable homotopy in the category of spaces is in fact a rather awkward game, because one constantly must suspend maps to stay in the stable range.
To save ourselves from this awkwardness, we will work in a stable homotopy category, that is, a category of ``spectra''.  

Spectra are a sort of dimension-less generalization of topological spaces.  
There is a functor from spaces to spectra called $\Sigma^\infty$, 
with the homotopy classes of maps between $\Sigma^\infty S^0$ and $\Sigma^\infty Y$ being exactly $\pi_*^s(Y)$.  
Another somewhat strange thing happens; cohomology theories end up begin objects in the stable homotopy category.  
Cohomology theories end up on the same footing as $\Sigma^\infty$ as spaces, and these cohomology spectra represent the cohomology of spectra.  

The category has a number of improved formal properties over topological spaces.  
First of all, there is a ``desuspension'' functor $\Sigma^{-1}$ which is an inverse to the suspension functor $\Sigma$.  
This means for any two spectra $X$ and $Y$, $[X,Y]$ is an abelian group (see \ref{sec:HomotopyIntro}).  
Better yet, cofiberings and fiberings are the same, so when you form the Puppe-Barratt Sequence of a map, you can, 
for any spectrum $X$, apply the functor $[?,X]$ or $[X,?]$, and get a long exact sequence.  
Cohomology and stable homotopy are functors of that form in this category.  

Without further ado, let us define the stable homotopy category, which we do following \cite{AdamsStable}


\subsection{Spectra}

We will now define the objects in the stable homotopy category.  
\begin{Def}
  A Spectrum is a sequence of topological spaces $X=\{X_n\}$ for $n\in \Z$, along with structure maps 
  \[\epsilon_n:\Sigma X_n\to X_{n+1}\]
  You can always assume the structure maps are inclusions of subcomplexes.  
  A ``cell'' of a Spectrum is an equivalence class of cells $e^m$ in $X_n$, where two cells $e^m$ and $e^{m'}$ ($m<m'$) are equivalent if $e^m$ 
  becomes $e^{m'}$ after $m'-m$ applications of the structure map.  
  The dimension of a cell represented by $e^m$ in $X_n$ is $m-n$.  
\end{Def}

We promised a functor from spaces to spectra, which will be an important source of examples.  

\begin{Def}
  Let $X$ be a topological space.  Let $\Sigma^\infty X$ be a topological space with
  \[(\Sigma^\infty X)_n = \left\{\begin{array}{cc} 
  \Sigma^n X & n \ge 0\\
  \{*\} & n < 0
  \end{array}\right.\]
  We call this the ``suspension spectrum'' of $X$.
  We define 
  \[S=\Sigma^\infty S^0\]
\end{Def}


Let $K^n$ be a cohomology theory, and let it be represented
\[K^n(?)=[X,K_n]\]
for some space $K_n$ (for instance $K_n$ could be $K(G,n)$ for some $G$).  
We have
\[[K_n,K_n]\cong[\Sigma K_n,K_{n+1}]\]
The image of the identity makes the structure map, thus a generalized cohomology theory is a spectrum.  
We call the spectrum associated with $K(G,n)$ by the name $HG$.  


\subsection{Functions, Maps and Morphisms}

We now have two rich sources of examples of spectra, but objects do not a category make.  
We also need to define the morphisms of our category.  
We do this first by defining ``functions''.  
\begin{Def}
  A degree $i$ function between two spectra $f:X\to Y$ is a serious of maps of topological spaces 
  \[f_n:X_n\to Y_{n-i}\]
  that ``commute'' with suspension, that is
  \[\epsilon_{n-i}(\Sigma f_n)=f_{n+1}\epsilon_n\]
\end{Def}

The problem with functions is that there aren't enough of them.  
The requirement they be defined on every single space keeps use from being in a stable situation.  
For instance, let $\eta:S^3\to S^2$ be the Hopf Fibration \ref{sec:HopfInvariant}.  
We would want this to define a degree 1 function from $\eta:S\to S$ with $\eta_3=\eta$.  
This means that would have to be $\eta_n=\Sigma^{n-3}\eta$ for $n\ge 3$.  
But what can $\eta_2$ be?  Since is a map $\eta_2:S^2\to S^1$, $\eta_2$ must be null-homotopic, so the suspension of it would have to be null-homotopic, 
which the Hopf Fibration is not.  
For this reason, we must weaken the notion to make maps in this category.  

\begin{Def}
  A ``cofinal'' subspectra $K\subset X$ is a subspectra such that for each $n$ and each cell $
  e\in X_n$ there is an $i$ such that applying the structure map to $e$ $i$ times will land $e$ in $K_{i+n}$.  
\end{Def}

Note that the intersection of two cofinal subspectra is again cofinal.  

\begin{Def}
  A ``map'' $f:X\to Y$ between two spectra $X$ and $Y$ is a function (or an equivalence class of functions) defined on any cofinal subspectra $K\subset X$.  
  Two maps are considered equal if they are equal on the intersection of their domains.  
\end{Def}

Now we can make $\eta:S\to S$ a map, since it is defined on the cofinal subspectrum $K\subset$ with
\[K_n = \left\{\begin{array}{cc} 
S^n & n \ge 3\\
\{*\} & n < 3
\end{array}\right.\]

Finally, we can define morphisms
\begin{Def}
  Let $X$ and $Y$ be spectra.  Then $Cyl(X)$ is a spectra with 
  \[(Cyl(X))_n=I^+\wedge X_n\]
  where $I^+$ is the unit interval with disjoint basepoint.  
  Note that the obvious structure map $1\wedge \epsilon_n$ works as a structure map.  
  There are two natural injections $i_1,i_2 X\to Cyl(X)$.  
  Two maps $f,g$ if there is a map 
  \[H:Cyl(X)\to Y\]
  with $Hi_1=f, Hi_2=g$.  
  A ``morphism'' $f:X\to Y$ is a homotopy class of maps.  
  We let $[X,Y]_n$ be the set up homotopy classes of degree $n$ maps between spectra $X$ and $Y$.  
\end{Def}

We can now make the definition
\begin{Def}
  Let $X$ be a spectrum.  Define the homotopy group
  \[\pi_n(X)=[S,X]_n\]
\end{Def}
Here is the thing to notice.  
The homotopy groups in the stable category are exactly the stable homotopy groups, that is
\begin{Theorem}
  Let $X$ be a topological space.  Then
  \[\pi_n^s(X)=\pi_n(\Sigma^\infty(X))\]
\end{Theorem}

Make sure to convince yourself of this before moving on.  

We can define cohomology, but it is somewhat different than for spaces.  A cohomology theory is always a spectrum, but in fact any spectrum is sufficient to define cohomology.  
\begin{Def}
  Let $E$ and $X$ be spectra.  
  The $E$-cohomology of $X$ is given
  \[E^k(X)=[X,E]_k\]
\end{Def}
Notice that 
\[(HG)^k(\Sigma^\infty X)=H^k(X;G)\]
and
\[(H\F_2)^*(H\F_2)=\A\]
by Cor  \ref{sec:SteenrodAreCohom}.  


\subsection{Additive Category of Spectra}

We have the following self-evident consequence:
\begin{Lemma}
  If $X$ is a spectra and $K\subset X$ is cofinal, then the inclusion $K\to X$ is an isomorphism
\end{Lemma}

Notice that since the subspectrum of a spectrum defined by collapsing negative-indexed spaces to a point is cofinal, so it doesn't matter whether we consider spaces indexed by all integers or positive integers.  

\begin{Def}
  Let $X$ be a spectra.  Define $\Sigma X$ by
  \[(\Sigma X)_n=\Sigma X_n\]
  and use the obvious structure maps.
\end{Def}
Obviously this is functorial.  
Define $\mbox{shift}_+$ and $\mbox{shift}_-$ to be the obvious functors.  
The structure maps define a degree 0 map 
\[\epsilon:\Sigma X\to \mbox{shift}_+\]
The image of $\epsilon$ is obviously cofinal and $\Sigma X$ is isomorphic to its image, so $\epsilon$ is an isomorphism.  
But $\mbox{shift}_-$ is an obvious inverse to $\mbox{shift}_+$, so there is a functor $\Sigma^{-1}$ inverting $\Sigma$.  
Thus any spectrum $X\cong\Sigma^2 X'$, so $X$ is isomorphic to a spectrum where each $X_n$ is a double-suspension.  

Let $X$ and $Y$ be spectra and $K\subset X$ be a cofinal subspectrum on which $f,g:K\to Y$ be representing functions for maps $f$ and $g$.  
We can assume that for each $n$, $K_n$ is a double suspension, so we can form $(f+g)_n=f_n+g_n$.  
While the sum is only defined up to homotopy, representing maps can be picked to commute so that $f+g$ is a bonafide morphism in the stable homotopy category.  
This makes $[X,Y]_n$ an abelian group.  
Obviously composition in either direction is a bilinear group homomorphism, that is, function composition gives abelian group-maps
\[[X,Y]\otimes [Y,Z]\to [X,Z]\]

We can construct wedge products in the obvious way
\begin{Def}
  If $X$ and $Y$ are spectra, we can form $X\vee Y$ with
  \[(X\vee Y)_n=X_n\vee Y_n\]
  and the wedge product of the structure map, where we use that
  \[\Sigma(X\vee Y)=(\Sigma X)\vee (\Sigma Y)\]
  for reduced suspensions.  The same construction works for infinite products.  
\end{Def}
This is obviously a coproduct, that is, for any spaces $X,Y,W$, we naturally have
\[[X\vee Y,W]\cong [X,W]\oplus [Y,W]\]


Consider the sequence of spectra
\[X\to X\vee Y\to Y\]
given by inclusion and then projection.  The sequence
\[[W,X]\to [W, X\vee Y]\to [W,Y]\]
is clearly a short exact sequence, naturally split by the map induced from the inclusion $Y\to X\vee Y$.  
Thus we have, for any space $W$
\[[W,X\vee Y]\cong [W,X]\oplus [W,Y]\]
This is the universal property of products, so $X\vee Y$ is a product as well as a coproduct.  

Finally, since $\{*\}$ has the property that $\Sigma\{*\}=\{*\}$ (again, recall we have been using reduced suspension), there is a zero spectrum $\{*\}=\Sigma^\infty\{*\}$.  


\subsection{Fibrations and Cofibrations}

Let $i:A\to X$ be an inclusion of subcomplexes for (unstable) spaces.  
We can then form the mapping cylinder and cofibration sequence
\[A\xrightarrow{i} X\xrightarrow{j} CA\cup_i X\]
Let $W$ be any space, and consider
\[[CA\cup_i X,W]\xrightarrow{j^*} [X,W]\xrightarrow{i^*}[A,W]\]
Obviously $(ij)^*=0$, since it includes $A$ into its contractible cone.  
Also, if a map $g:X\to W$ has $i^*(g)=0$, then $g$ restricted to $A$ is nullhomotopic, but a nullhomotopy is just a map $CA\to W$ extending $g$, so $g$ can be extended to $CA\cup_i X$.  
Thus the sequence above is exact.  

Now, $j$ is an inclusion of a subcomplex, thus we can continue the sequence 
\[A\xrightarrow{i} X\xrightarrow{j} CA\cup_i X\xrightarrow{k} CX\cup_j CA\cup_i X\cong \Sigma A\]
if we continued the sequence we would get a map homotopic to $-\Sigma(i)$ going to $\Sigma X$, and so on.  
Each three terms is the inclusion of a subcomplex and a mapping cone, so is exact.  
Finally, there is a natural homotopy equivalence $CA\cup_i X\to X/A$ and isomorphism $[\Sigma X,Y]_n\cong [X,Y]_{n-1}$
We summarize in a lemma
\begin{Lemma}[Cofibration Sequence of Spaces]
  \label{sec:cofibspace}
  If $A\to X$ is an inclusion of a subcomplex of (unstable) spaces.  Then there is a long exact sequence for any space $W$
  \[...\to [X/A,W]_n\to [X,W]_n\to [A,W]_n\to [X/A,W]_{n+1}\to ...\]
\end{Lemma}

Quotients by subcomplexes commute with suspenison, so given spectra $A\subset X$ which is an inclusion of subcomplexes, we can form $X/A$ in the obvious way,
and of course this is equivalent to a spectra $X\cup CA$ with $(X\cup CA)_n=C_n\cup CA_n$.  
Now, Once again, a nullhomotpy from $g:X\to Y$ is a map $CX\to Y$ with which restricts to $g$.  
Thus, the argument above yields 

\begin{Lemma}[Cofibration Sequence of Spectra]
  If $A\to X$ is an inclusion of a spectrum which is an inclusion of subcomplexes on each space, then Lemma \ref{sec:cofibspace} holds.  
\end{Lemma}

\begin{Lemma}[Fibration Sequences of Spectra]
  If $A\to X$ is a cofibration, then 
  \[...\to [W,A]_n\to [W,X]_n\to [W,X/A]_n\to [W,A]_{n-1}\to ...\]
\end{Lemma}

\begin{proof}
  Obviously $A\to A/X$ is null homotopic, so let $g:W\to X$ be a map which becomes nullhomotopic in $X/A$.  
  We get the following diagram
  Consider the following diagram
  \begin{diagram}
    A & \rTo^i & X & \rTo^j & CA\cup X&\rTo^k &\Sigma A&\rTo^{-\Sigma i}&\Sigma X\\
    \uDashto^{l'} &     &\uTo^g & &    \uTo^h          &     & \uTo^l  &   &   \uTo^{\Sigma g}\\
    W        & \rTo^1 & W       & \rTo& CW  & \rTo & \Sigma W & \rTo^{-1} & \Sigma W
  \end{diagram}
  We get $h$ from the nullhomotpy $ig$ and $l$ from attaching another copy of $h$.  
  Let $l=\Sigma l'$.  Then we have $\Sigma (il')=\Sigma g$, so $il'=g$, so the $g$ can be compressed to $A$ and the first point is exact.  
  The other exactness points follow from the symmetry in the sequence.  
\end{proof}

\begin{Def}
  A triple $(A,X,C)$ where $A\to X$ is a cofibration and $C$ is the cofiber is called a"distinguished triangle''.   
\end{Def}

\subsection{Smash Products}

Our category is equipped with a smash product.
The construction is notorious for being as confusing as it is unnecessary.  Frank Adams himself said of them,  
``In order to operate the machine, it is not necessary to raise the bonnet'', and we will take this approach as well.  
A construction can be found in \cite[Ch~4]{AdamsStable}
The smash product should be seen as a generalization of the smash product of spaces and similar to a tensor product on modules.  

\begin{Theorem}
  Let $X,Y$ be topological spaces.  Then there is a topological space $X\wedge Y$ with the properties that
  \begin{enumerate}
    \item $S\wedge X=X$
    \item If $X$ and $Y$ are spaces, $\Sigma^\infty(X\wedge Y)=(\Sigma^\infty X)\wedge (\Sigma^\infty Y)$
    \item $(X\wedge Y)\wedge Z\cong X\wedge (Y\wedge Z)$.  
      Thus we just write $X\wedge Y\wedge Z$
    \item  $X\wedge Y\cong Y\wedge X$
    \item  $\Sigma(X\wedge Y)\cong(\Sigma X)\wedge Y\cong X\wedge (\Sigma Y)$
    \item $[W,Y\wedge Z]=[W,Y]\otimes [W,Z]$
    \item $[Y\wedge Z,W]=[Y,W]\otimes [Z,W]$
  \end{enumerate}
\end{Theorem}


This completes our brief tour of spectra, as these are all of the results and constructions we will need to construct the Adams Spectral Sequence.  


